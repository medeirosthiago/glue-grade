\chapter{MATERIAIS E MÉTODOS}\label{ch:materiais-metodos}
Após a revisão bibliográfica de outros estudos e os fundamentos teóricos necessários para a mineração de dados utilizando Python, torna-se importante definir as ferramentas, tecnologias e procedimentos necessários para o desenvolvimento do projeto.

Este capítulo apresenta os materiais e métodos utilizados para a realização do processo de \textit{data mining}, onde, na seção~\ref{sec: tec-ferramenta} são apresentadas as tecnologias e ferramentas que serão utilizadas durante o estudo. Serão abordados quais as bibliotecas que a linguagem Python disponibiliza para a análise e mineração de dados e, também, como acessar a API do \textit{Twitter}. Esta será esclarecida também neste capítulo, após a explicação do conceito de API e o protocolo OAuth.

A seção~\ref{sec: metodologia} irá concluir o capítulo apresentando as etapas de \textit{data mining} com o intuito de evidenciar o processo para a obtenção de conhecimento útil.

\section{TECNOLOGIAS E FERRAMENTAS}\label{sec: tec-ferramenta}
Tecnologias e ferramentas para a criação e prototipagem dos algoritmos.

%% Bibliotecas Python
\subsection{Bibliotecas da Linguagem Python}\label{sec:bib_python}
Um dos grandes diferenciais da linguagem Python é o seu enorme conjunto de bibliotecas para soluções de diversos problemas.

A seguir serão apresentadas as bibliotecas necessárias para a mineração de dados, através das quais é possível coletar, limpar, transformar, realizar operações e apresentar resultados proveniente dos dados da rede social \textit{Twitter}. Para evitar repetições da palavra "biblioteca", o termo "pacote"\space também será utilizado com o mesmo significado no restante desta dissertação.
 
%%% sub: NumPy
\subsubsection{\textit{Biblioteca NumPy}}
\textit{NumPy} é o pacote fundamental para computação científica em Python. É o acrônico para \textit{Numerical Python}. Esta biblioteca provê:

\begin{itemize}
    \item \textit{ndarray} que é um objeto de matriz multidimensional;
    \item Funções que permitem realizar operações vetoriais ou operações matemáticas entre matrizes sem a necessidade de programar \textit{loops};
    \item Ferramentas para a leitura e escrita em conjuntos de dados matriciais;
    \item Operações de álgebra linear, transformada de Fourier e geração de números aleatórios;
    \item Ferramentas para a integração em outras linguagens de programação como C, C++ e Fortran.
\end{itemize}

Além da capacidade de rápido processamento em matrizes que o \textit{NumPy} oferece ao Python, um dos principais objetivos em relação a análise de dados é que serve como um "container"\space para os dados serem passado por algoritmos. Para dados numéricos, as matrizes de \textit{NumPy} são muito mais eficientes para a ordenação e manipulação de dados do que qualquer outra estrutura embutida em Python. Igualmente, bibliotecas escritas em linguagens de baixo nível, como C ou Fortran, podem operar dados gravados em matrizes da \textit{NumPy} sem precisar da cópia de qualquer dado \cite{python-analysis}.

A biblioteca \textit{NumPy} por si só, não provê uma funcionalidade de alto-nível para a análise de dados. Tendo um conhecimento sobre as matrizes de \textit{NumPy} e matrizes orientadas a computação (\textit{array-oriented computing}) irá facilitar o uso de outras ferramentas, como \textit{pandas}, com mais efetividade.

Para aplicações voltadas para a análise de dados, esta biblioteca possui grande funcionalidade em setores como:

\begin{itemize}
    \item Criação rápida de matrizes para a interação e limpeza de dados, separação e filtragem, transformação e outros tipos de operações computacionais;
    \item Algoritmos comuns para matrizes como ordenação, operações únicas e definidas;
    \item Eficiente descrição estatística e agregação/sumarização de dados;
    \item Alinhamento de dados e manipulação de dados relacionais para operações de junção e imerção (\textit{join} e \textit{merge}) de conjuntos de dados heterogeneos;
    \item Expressar lógicas de condições através de expressões matriciais ao invés de laços de repetições e condições como \textit{while, for, if-elif-else};
    \item Agrupamento de manipulação de dados (agregação, transformação, aplicação de funções).
\end{itemize}

Enquanto \textit{NumPy} oferece o fundamento computacional para essas operações, é preferível utilizar a biblioteca \textit{pandas} como base para a mineração de dados (especialmente de dados estruturados ou dados tabulados), devido a sua interface rica e de alto-nível no qual permite as tarefas com dados mais concisas e simples.

%%% sub: pandas
\subsubsection{Biblioteca \textit{pandas}}\label{pandas}
A biblioteca \textit{pandas} atrai o maior interesse de cientistas quanto a mineração de dados. Ela possui estruturas de dados de alto-nível e ferramentas de manipulação desenvolvidas para facilitar e agilizar a análise de dados em Python. \textit{pandas} é desenvolvida sob a biblioteca \textit{NumPy} e viabiliza o uso em aplicações centradas nesta. A seguir são expostas algumas soluções que a biblioteca disponibiliza \cite{python-analysis}:

\begin{itemize}
    \item Estrutura de dados com eixos rotulados suportam o alinhamento de dados automáticos ou explícitos. Isso evita erros comuns resultantes de dados desalinhados e dados indexados de formas diferentes provenientes de outras fontes de dados;
    \item A mesma estrutura de dados consegue manusear tanto dados de séries temporais como dados não-temporais;
    \item Operações e reduções aritméticas é passado para metadados (eixos rotulados);
    \item Manipulação flexível de dados em falta;
    \item \textit{Merge} (fundir) e outras operações relacionais encontradas em bancos de dados relacional.
\end{itemize}

Esta biblioteca possui duas estrutura de dados principais: \textit{Series} e \textit{DataFrame}. Estas estruturas não são uma solução universal para todos os problemas, mas provê uma base sólida e de fácil manipulação para a maioria das aplicações com mineração de dados.

Uma \textit{Series} é um tipo de \textit{array} ou uma matriz unidimensional, similar a um \textit{array} que possui uma matriz de dados (qualquer tipo de dado da biblioteca \textit{NumPy}) e um outro vetor associado a dados rotulados, chamados de \textit{index} (índice). Uma simples \textit{Series} é formado por uma única matriz de dados, conforme a Figura~\ref{pandas-series}.

\begin{figure}[h!]
	\centering
	\fbox{\includegraphics[width=.5\textwidth]{Cap4/imagens/pandas-series}}
	\caption{Exemplo de uma \textit{Series}}
	\vspace{-0.3cm}
	\legend{FONTE: \citeonline{python-analysis}}
	\label{pandas-series}
\end{figure}

\textit{DataFrame} representa uma tabela, uma estrutura de dados do tipo planilha, que possui uma coleção ordenada de colunas, onde cada uma delas pode ter um tipo de valor diferente (numérico, \textit{string}, \textit{boolean}, etc.). O \textit{DataFrame} possui um índice para linhas e também para colunas. Pode ser interpretado como um dicionário de \textit{Series}. De uma maneira geral, o dado é armazenado como um ou mais blocos bi-dimensionais ao invés de uma lista, dicionário, ou outro tipo de coleção de matriz unidimensional \cite{python-analysis}.

Existem várias maneiras diferentes de se criar um \textit{DataFrame}, entretanto uma forma comum é um dicionário de dimensões iguais, conforme a Figura~\ref{pandas-dataframe} e a Figura~\ref{pandas-dataframe2}, ou uma matriz \textit{NumPy}.

\begin{figure}[h!]
	\centering
	\fbox{\includegraphics[width=.9\textwidth]{Cap4/imagens/pandas-dataframe}}
	\caption{Criação de um \textit{DataFrame}}
	\vspace{-0.3cm}
	\legend{FONTE: \citeonline{python-analysis}}
	\label{pandas-dataframe}
\end{figure}

\begin{figure}[h!]
	\centering
    \fbox{\includegraphics[width=.36\textwidth]{Cap4/imagens/pandas-dataframe2}}
	\caption{Conteúdo de um \textit{DataFrame} pelo interpretador \textit{IPython}}
	\vspace{-0.3cm}
	\legend{FONTE: \citeonline{python-analysis}}
	\label{pandas-dataframe2}
\end{figure}

%%% sub: matplotlib
\subsubsection{Biblioteca \textit{matplotlib}}
O pacote \textit{matplotlib} é desenvolvido para a geração de gráficos bidimensionais a partir de \textit{arrays}. Gráficos comuns podem ser criados com alta qualidade a partir de simples comandos, inspirados nos comandos gráficos do MATLAB, exemplo ilustrado na Figura~\ref{matplotlib-fig}.

Quando usado em conjunto com ferramentas GUI (\textit{IPython}, por exemplo), esta biblioteca possui recursos interativos como zoom e visão panorâmica. Além disto, suporta várias ferramentas GUI \textit{backend}, nos diversos sistemas operacionais suportados pelo Python, e permitem exportar gráficos em diversos formatos: PDF, SVG, JPG, PNG, BMP, GIF, etc.

\textit{matplotlib} também possui várias ferramentas adicionais, como o \textit{mplot3d} para plotar gráficos tridimensionais, e o \textit{basemap} para mapeamentos e projeções.

\begin{figure}[h!]
  \includegraphics[width=1\textwidth]{Cap4/imagens/matplotlib}
  \caption{Exemplo de um gráfico gerado pelo \textit{matplotlib}}
  \vspace{-0.3cm}
  \legend{FONTE: \citeonline{matplotlib}}
  \label{matplotlib-fig}
\end{figure}


%%% sub: SciPy
\subsubsection{Biblioteca \textit{SciPy}}
\textit{SciPy} é uma coleção de pacotes que abordam uma série de soluções para diferentes domínios na computação científica. Na lista a seguir são apresentados exemplos desses pacotes \cite{python-analysis}:

\begin{itemize}
	\item \textit{scipy.integrate}: rotinas de integração numéricas e soluções de equações diferenciais;
	\item \textit{scipy.linalg}: rotinas de álgebra linear e decomposição de matrizes;
	\item \textit{scipy.optimize}: funções otimizadoras (minimizadoras) e algoritmos de busca em raíz;
	\item \textit{scipy.signal}: ferramentas para processamento de sinais;
	\item \textit{scipy.sparse}: matrizes esparsas e soluções de sistemas lineares esparsos;
	\item \textit{scipy.special}: agregador do \textit{SPECFUN}, uma biblioteca do Fortran que implementa várias funções matemáticas, como exemplo, a função gama;
	\item \textit{scipy.stats}: funções estatísticas, variáveis contínuas e discretas, testes estatísticos e outros modelos estatísticos;
	\item \textit{scipy.weave}: ferramenta para usar códigos \textit{inline} de C++ para acelerar a computação de matrizes.
\end{itemize}


%%% sub: IPython
\subsubsection{Interpretador \textit{IPython}}
O interpretador \textit{IPython} teve seu desenvolvimento iniciado em 2001, com o intuito de ser um interpretador interativo para a linguagem Python. Desde a sua criação o \textit{IPython} evoluiu grandemente, ao ponto de ser considerada uma das mais importantes ferramentas para computação científica em Python. Essa biblioteca não oferece nenhuma ferramenta para análise de dados ou análise computacional em si, sendo designada para maximizar a produtividade, tanto na interação computacional como no desenvolvimento de softwares. Oferece um fluxo de visualização de um modo \textit{execute-explore} ao invés do típico modelo \textit{edit-compile-run} de muitas outras linguagens de programação. Ela também provê uma pequena integração com o \textit{shell} e sistemas de arquivos. Como a maior parte da programação focada na mineração de dados envolve exploração, tentativa, erro e iteração, \textit{IPython}, em quase todos os casos, irá facilitar este tipo de trabalho \cite{python-analysis}.

Hoje, o projeto \textit{IPython}, mantido pela empresa \textit{Jupyter}, engloba muito mais do que apenas um interpretador \textit{shell} para Python. Ele também inclui um console gráfico interativo, o \textit{IPython Notebook}, que provê ao usuário uma experiência de caderno (\textit{notebook-like}) através de um navegador \textit{web}, conforme Figura~\ref{ipython-fig}, e dispõe de um mecanismo de processamento paralelo. Assim como muitas outras ferramentas desenvolvidas para programadores, é extremamente customizável \cite{mining-social-web}. \\ \\ \\ \\ \\ \\ \\

\begin{figure}[h!]
  \centering
  \fbox{\includegraphics[width=0.95\textwidth]{Cap4/imagens/jupyter}}
  \caption{Exemplo de uma página \textit{web} do \textit{IPython Notebook}}
  \vspace{-0.3cm}
  \legend{FONTE: Elaborado pelo autor}
  \label{ipython-fig}
\end{figure}

%%
% API
\subsection{Interface de Programação de Aplicações - API}\label{subsec: api}
API é uma sigla para \textit{Application Programming Interface} e basicamente é uma tecnologia que permite um pedaço de \textit{software} se comunicar com outro pedaço de \textit{software}. Existem vários tipos de API e é comumente referenciado a outras tecnologias. Por exemplo, para o desenvolvimento deste trabalho será utilizado a API do \textit{Twitter}. 

Uma API é composta por uma série de funções acessíveis somente por programação, e que permitem utilizar características do \textit{software} menos evidentes ao utilizador tradicional.


%% 
% REST API
\subsubsection{Arquitetura REST}
Abreviação para Transferência de Estado Representacional (REST), é um estilo arquitetural baseado em recursos e nas representações desses recursos. Enfatiza a escalabilidade na interação entre componentes, a generalidade de interfaces, a implantação independente dos componentes de um sistema, o uso de componentes intermediários visando a redução na latência de interações, o reforço na segurança e o encapsulamento de sistemas legados. A REST ignora os detalhes da implementação de componente e a sintaxe de protocolo com o objetivo de focar nos papéis dos componentes, nas restrições sobre sua interação com outros componentes e na sua interpretação de elementos de dados significantes \cite{rest}.

REST foi um termo criado por \citeonline{rest}, onde ele modela um estilo de arquitetura para a construção de serviços \textit{web} consistentes e coesos. O estilo da arquitetura REST é baseado em recursos e nos estados desses recursos.

A funcionalidade de uma REST API é similar ao funcionamento de uma página \textit{web}, onde o usuário efetua uma requisição a um servidor \textit{web}, utilizando o protocolo HTTP, e recebe dados como resposta.

Um recurso é qualquer conteúdo ou informação que é exposto na Internet, podendo ser um documento, vídeo clip, até processos de negócio ou dispositivos. Para utilizar um recurso é necessário ser capaz de identificá-lo na rede e de ter meios para manipulá-lo. Tem-se então, o \textit{Uniform Resource Identifiers} (URI) para este propósito. Um URI unicamente identifica um recurso e, ao mesmo tempo, o torna endereçável ou capaz de ser manipulado utilizando um protocolo, como o HTTP. O URI de um recurso se distingue dos de qualquer outro recurso e é através do próprio URI que ocorrem as interações com o recurso \cite{rest-book}.

Recursos devem possuir pelo menos um identificador para ser endereçável, e cada identificador é associado com uma ou mais representações, que é uma transformação ou uma visão do estado do recurso em um instante de tempo. Essa visão é codificada em um ou mais formatos transferíveis, tal como XHTML, Atom, texto simples, XML, YML, JSON, JPG, MP3, entre outros  \cite{rest-book}.

Os recursos provêm o conteúdo ou objeto com o qual se quer interagir e para atuar sobre eles é utilizado os métodos de HTTP. Os métodos HTTP na arquitetura REST podem ser referenciados como Verbos, uma vez que representam ações sobre os recursos \cite{rest-book}.


\subsection{Protocolo de Autenticação - OAuth}
Protocolos de autenticação são capazes de, simplesmente, autenticar a parte que está se conectando, ou ainda de autenticar a parte que está conectando, assim como se autenticar para ele.

Neste trabalho será utilizado apenas o protocolo OAuth 1.0 para o acesso aos dados do \textit{Twitter}. É possível também, realizar a autenticação utilizando a versão mais atual, OAuth 2.0, mas será apenas referenciado, neste trabalho, para a melhor compreensão do funcionamento do protocolo.

OAuth é uma sigla para "\textit{open authorization}", ou autorização aberta, e provê um meio para que usuários autorizem uma aplicação acessar dados, com alguma finalidade, através de uma API, sem que os usuários precisem passar credenciais como nome de usuário e senha. De um modo geral, usuários são capazes de controlar o nível de acesso para estas aplicações e revogar este controle a qualquer momento \cite{mining-social-web}.

\subsubsection{Protocolo OAuth 1.0a}
OAuth 1.0a é um protocolo que permite que um cliente (\textit{client}) \textit{web} tenha acesso a um recurso protegido pelo seu dono em um servidor. Esta definição se dá através da RFC 5849. Que são documentos técnicos desenvolvidos e mantidos pelo Internet Enginnering Task Force (IETF), instituição que especifica os padrões que serão implementados e utilizados em toda a Internet.

A razão para a existência dessa tecnologia é para evitar problemas de usuários (donos dos recursos) compartilhar suas senhas com aplicações \textit{web}.

A versão OAuth 1.0a não permite que credenciais sejam trocadas utilizando uma conexão \textit{Secure Socket Layer} (SSL) através de um protocolo HTTPS. Por esse motivo, muitos desenvolvedores achavam tedioso o trabalho devido aos vários detalhes envolvidos em encriptação.

SSL é um padrão global para tecnologia de segurança. Tem como função principal criar um canal criptografado entre um servidor \textit{web} e um navegador (\textit{browser}) para garantir que todos os dados transmitidos sejam seguros e sigilosos.

Uma aplicação que está requerindo acesso é conhecida como \textit{client}, em alguns momentos chamado de \textit{consumer}, a rede social ou o serviço que contém os recursos protegidas é nomeado como \textit{server} (também chamado de \textit{provider}) e o usuário que concede o acesso é o \textit{resource owner} (dono do recurso, tradução livre). Com estes elementos, as três participações que envolvem o processo e a interação que estes elementos possuem é conhecida como \textit{"three-legged-flow"} ou de uma maneira mais coloquial, a \textit{OAuth dance}. Estas são as etapas fundamentais que envolvem a \textit{OAuth dance} que, como resultado, permite ao \textit{client} o acesso a recursos protegidos, conforme listado a seguir \cite{mining-social-web}:

\begin{enumerate}
	\item O \textit{client} obtém um \textit{token} de requisição do servidor de serviço (aplicação);
	\item O dono do recurso autoriza o \textit{token} de requisição;
	\item O \textit{client} troca o \textit{token} de requisição por um \textit{token} de acesso;
	\item O \textit{client} usa o \textit{token} de acesso para acessar os recursos protegidos com a consideração do dono do recurso.
\end{enumerate}

Para credenciais particulares, um \textit{client} começa com uma \textit{consumer key} e um \textit{consumer secret} e no fim do processo de \textit{OAuth dance}, termina com um \textit{token} de acesso e \textit{token} de acesso secreto que pode ser usado para acessar recursos protegidos.

\subsubsection{Protocolo OAuth 2.0}
Enquanto o protocolo OAuth 1.0a permite uma autorização útil para o acesso a aplicações \textit{web}, o OAuth 2.0 foi originalmente destinado a simplificar, significantemente, a implementação detalhada para desenvolvedores de aplicações \textit{web}, baseando-se completamente no SSL para aspectos de segurança e para satisfazer uma vasta quantidade de casos de uso. Esses casos de uso variaram desde suporte para dispositivos móveis à necessidades empresariais e, consequentemente, às necessidades de um termo mais futuro, da "Internet das Coisas"\space \cite{mining-social-web}.

Diferentemente da implementação OAuth 1.0a, que consiste de um rígido conjunto de etapas, a implementação do OAuth 2.0, definido através do RFC 6749, pode variar de acordo com a particularidade do caso de uso. Um decorrer típico da execução do OAuth 2.0 tem a vantagem do SSL e, essencialmente, contém apenas poucos redirecionamentos que, acompanhada de em alto-nível, não possui tanta diferença em relação ao processo anterior que envolvem um ciclo do OAuth 1.0a.

%%
% Twitter
\subsection{Rede Social \textit{Twitter}}
Para definir o que seria o \textit{Twitter}, \citeonline{mining-social-web} aborda as seguintes necessidades que uma tecnologia social precisa disponibilizar à uma pessoa:

\begin{itemize}
	\item Permitir que a pessoa seja ouvida;
	\item Permitir que a pessoa satisfaça suas curiosidades;
	\item Permitir isso de um modo fácil e acessível;
	\item Permitir isso agora.
\end{itemize}

Essas observações são, de um modo geral, verdadeiramente humano. Pessoas possuem o desejo de compartilhar ideias e experiências, que as permitam se conectarem com outras pessoas, para serem ouvidas, e se sentirem parte ou digna de importância \cite{mining-social-web}.

Os dois últimos itens enfatizam a vontade de não ter alguma dificuldade para satisfazer as curiosidades ou realizar algum trabalho específico.

Uma maneira de descrever o \textit{Twitter} é, então, um serviço de \textit{microblog} que permite uma breve comunicação entre pessoas. Mensagens de no máximo 140 caracteres explicam a "breve comunicação", que normalmente correspondem a pensamentos ou ideias sobre um determinado assunto. Em outras palavras, \textit{Twitter} é um serviço global de trocas de mensagens, extremamente rápido e gratuito.

%Besides the macro-level possibilities for marketing and advertising—which are always lucrative with a user base of that size—it’s the underlying network dynamics that created the gravity for such a user base to emerge that are truly interesting, and that’s why Twitter is all the rage. While the communication bus that enables users to share short quips at the speed of thought may be a necessary condition for viral adoption and sustained engagement on the Twitter platform, it’s not a sufficient condition. The extra ingredient that makes it sufficient is that Twitter’s asymmetric following model satisfies our curi‐ osity. It is the asymmetric following model that casts Twitter as more of an interest graph than a social network, and the APIs that provide just enough of a framework for struc‐ ture and self-organizing behavior to emerge from the chaos.
%In other words, whereas some social websites like Facebook and LinkedIn require the mutual acceptance of a connection between users (which usually implies a real-world connection of some kind), Twitter’s relationship model allows you to keep up with the latest happenings of any other user, even though that other user may not choose to follow you back or even know that you exist. Twitter’s following model is simple but exploits a fundamental aspect of what makes us human: our curiosity. Whether it be an infatuation with celebrity gossip, an urge to keep up with a favorite sports team, a keen interest in a particular political topic, or a desire to connect with someone new, Twitter provides you with boundless opportunities to satisfy your curiosity.
%
%Think of an interest graph as a way of modeling connections between people and their arbitrary interests. Interest graphs provide a profound number of possibilities in the data mining realm that primarily involve measuring correlations between things for the objective of making intelligent recommendations and other applications in machine learning. For example, you could use an interest graph to measure correlations and make recommendations ranging from whom to follow on Twitter to what to purchase online to whom you should date. To illustrate the notion of Twitter as an interest graph, con‐ sider that a Twitter user need not be a real person; it very well could be a person, but it could also be an inanimate object, a company, a musical group, an imaginary persona, an impersonation of someone (living or dead), or just about anything else.
%For example, the @HomerJSimpson account is the official account for Homer Simpson, a popular character from The Simpsons television show. Although Homer Simpson isn’t a real person, he’s a well-known personality throughout the world, and the @Homer‐ JSimpson Twitter persona acts as an conduit for him (or his creators, actually) to engage his fans. Likewise, although this book will probably never reach the popularity of Homer Simpson, @SocialWebMining is its official Twitter account and provides a means for a community that’s interested in its content to connect and engage on various levels. When you realize that Twitter enables you to create, connect, and explore a community of interest for an arbitrary topic of interest, the power of Twitter and the insights you can gain from mining its data become much more obvious.
%There is very little governance of what a Twitter account can be aside from the badges on some accounts that identify celebrities and public figures as “verified accounts” and basic restrictions in Twitter’s Terms of Service agreement, which is required for using the service. It may seem very subtle, but it’s an important distinction from some social websites in which accounts must correspond to real, living people, businesses, or entities of a similar nature that fit into a particular taxonomy. Twitter places no particular re‐ strictions on the persona of an account and relies on self-organizing behavior such as following relationships and folksonomies that emerge from the use of hashtags to create a certain kind of order within the system.



\subsubsection{API \textit{Twitter}}\label{api-twitter}

%Twitter might be described as a real-time, highly social microblogging service that allows users to post short status updates, called tweets, that appear on timelines. Tweets may include one or more entities in their 140 characters of content and reference one or more places that map to locations in the real world. An understanding of users, tweets, and timelines is particularly essential to effective use of Twitter’s API, so a brief intro‐ duction to these fundamental Twitter Platform objects is in order before we interact with the API to fetch some data. We’ve largely discussed Twitter users and Twitter’s asymmetric following model for relationships thus far, so this section briefly introduces tweets and timelines in order to round out a general understanding of the Twitter plat‐ form.
%Tweets are the essence of Twitter, and while they are notionally thought of as the 140 characters of text content associated with a user’s status update, there’s really quite a bit more metadata there than meets the eye. In addition to the textual content of a tweet itself, tweets come bundled with two additional pieces of metadata that are of particular note: entities and places. Tweet entities are essentially the user mentions, hashtags, URLs, and media that may be associated with a tweet, and places are locations in the real world
%
%that may be attached to a tweet. Note that a place may be the actual location in which a tweet was authored, but it might also be a reference to the place described in a tweet.
%To make it all a bit more concrete, let’s consider a sample tweet with the following text:
%@ptwobrussell is writing @SocialWebMining, 2nd Ed. from his home office in Franklin, TN. Be #social: http://on.fb.me/16WJAf9
%The tweet is 124 characters long and contains four tweet entities: the user mentions @ptwobrussell and @SocialWebMining, the hashtag #social, and the URL http:// on.fb.me/16WJAf9. Although there is a place called Franklin, Tennessee that’s explicitly mentioned in the tweet, the places metadata associated with the tweet might include the location in which the tweet was authored, which may or may not be Franklin, Tennessee. That’s a lot of metadata that’s packed into fewer than 140 characters and illustrates just how potent a short quip can be: it can unambiguously refer to multiple other Twitter users, link to web pages, and cross-reference topics with hashtags that act as points of aggregation and horizontally slice through the entire Twitterverse in an easily searchable fashion.
%Finally, timelines are the chronologically sorted collections of tweets. Abstractly, you might say that a timeline is any particular collection of tweets displayed in chronological order; however, you’ll commonly see a couple of timelines that are particularly note‐ worthy. From the perspective of an arbitrary Twitter user, the home timeline is the view that you see when you log into your account and look at all of the tweets from users that you are following, whereas a particular user timeline is a collection of tweets only from a certain user.
%For example, when you log into your Twitter account, your home timeline is located at https://twitter.com. The URL for any particular user timeline, however, must be suffixed with a context that identifies the user, such as https://twitter.com/SocialWebMining. If you’re interested in seeing what a particular user’s home timeline looks like from that user’s perspective, you can access it with the additional following suffix appended to the URL. For example, what Tim O’Reilly sees on his home timeline when he logs into Twitter is accessible at https://twitter.com/timoreilly/following.
%An application like TweetDeck provides several customizable views into the tumultuous landscape of tweets, as shown in Figure 1-1, and is worth trying out if you haven’t journeyed far beyond the Twitter.com user interface.
%
%Whereas timelines are collections of tweets with relatively low velocity, streams are samples of public tweets flowing through Twitter in realtime. The public firehose of all tweets has been known to peak at hundreds of thousands of tweets per minute during events with particularly wide interest, such as presidential debates. Twitter’s public fire‐ hose emits far too much data to consider for the scope of this book and presents inter‐ esting engineering challenges, which is at least one of the reasons that various third- party commercial vendors have partnered with Twitter to bring the firehose to the masses in a more consumable fashion. That said, a small random sample of the public timeline is available that provides filterable access to enough public data for API de‐ velopers to develop powerful applications.
%The remainder of this chapter and Part II of this book assume that you have a Twitter account, which is required for API access. If you don’t have an account already, take a moment to create onem and then review Twitter’s liberal terms of service, API docu‐ mentation, and Developer Rules of the Road. The sample code for this chapter and Part II of the book generally don’t require you to have any friends or followers of your own, but some of the examples in Part II will be a lot more interesting and fun if you have an active account with a handful of friends and followers that you can use as a basis for social web mining. If you don’t have an active account, now would be a good time to get plugged in and start priming your account for the data mining fun to come.

%O acesso a API se dá através da criação de uma aplicação pela página \textit{web} de desenvolvimento do \textit{Twitter}. Após a criação da aplicação é fornecido ao usuário informações para o acesso utilizando o protocolo OAuth, será informado uma chave da API da aplicação, uma chave secreta, o \textit{token} de usuário OAuth e credenciais secretas do usuário OAuth.
%
%Dispondo de todas as credenciais do protocolo OAuth o acesso à API acontece utilizando a biblioteca \textit{tweepy}, especificada na subseção~\ref{sec:bib_python}.

% Bibliotecas Twitter
\subsubsection{Bibliotecas Para o Consumo de Dados da API do \textit{Twitter}}
%Esta biblioteca provê ao Python a API do \textit{Twitter}. Através da utilização do protocolo OAuth 1.0a, é possível acessar diversos campos como \textit{Profile}, \textit{Group}, \textit{Company}, \textit{Jobs}, \textit{Search}, \textit{Share}, \textit{Network} e requisições REST APIs \cite{tweepy}.


%%
% METODOLOGIA
\section{METODOLOGIA E DESENVOLVIMENTO}\label{sec: metodologia}
%O processo de desenvolvimento da solução segue uma série de princípios de conjunto de boas práticas e etapas do \textit{data mining}, para melhor estruturar e obter, não só o resultado esperado, mas também para que todo o processo ocorra de forma coerente e padronizada.

%Onde e como será realizada a pesquisa; Tipo de pesquisa, população (universo da pesquisa), amostragem, instrumentos de coleta, forma como pretende tabular e analisar os dados

%Nesta seção são apresentadas as principais metodologias utilizadas neste trabalho. Na seção XX …
\subsection{Iterativo e Incremental}


%%
%% Etapas do Data Mining
%\subsection{Etapas Para a Mineração de Dados do \textit{Twitter}}
%Abordado previamente, o acesso aos dados do \textit{Twitter} acontece através da utilização de sua API e a interpretação dos dados se dá pela aplicação da técnica de \textit{clustering}, normalização de dados e computação de similaridade. 

















