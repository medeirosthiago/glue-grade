\chapter{INTRODUÇÃO}\label{ch:introducao}

Redes sociais se tornaram um termo comum e uma chave fundamental para o estilo de vida moderno. Hoje em dia, a maioria das pessoas, independente de idade, sexo, crença, utilizam uma ou mais redes sociais. A princípio, esses ambientes \textit{on-line} focavam-se na comunicação, por exemplo; a possibilidade de se comunicar com alguém distante e tornar esse diálogo pessoal, seguro e, de alguma forma, próximo, ajudando na popularização desse tipo de tecnologia. No decorrer dos anos e com o avanço tecnológico, diferentes tipos de redes sociais surgiram com ideias semelhantes ou extremamente diferentes, não sendo apenas para a comunicação, mas para outros fins como o compartilhamento de mídias, localização, críticas, \textit{mini-blogs}, perguntas e respostas, negócios, profissão, música, artes, venda e troca de produtos, entre outros.

\textit{Facebook}, \textit{Twitter}, \textit{LinkedIn}, \textit{Google+} e, muito comum entre desenvolvedores, o \textit{GitHub} são exemplos populares de redes sociais. Logo, possuem grande número de usuários e diversas interações que estes realizam a cada momento, gerando uma quantidade gigantesca de dados. Esses dados são informações sobre pessoas, comportamentos, gostos, marcas e vários outros tipos de conteúdo. Devido a diversidade e a vasta quantidade desse tipo de informação, algumas redes sociais as utilizam para o aprimoramento de conteúdo ou, então, para o comércio de dados para empresas, por exemplo; de publicidade e marketing, que fazem a mineração desses dados para encontrar padrões de seus usuários e, assim, conseguir aumentar suas vendas, reduzir riscos e, até mesmo, gerar novas tendências.

A mineração de dados, também conhecida como \textit{data mining}, é o processo de analisar dados em diferentes perspectivas e transformar em informação útil. Hoje em dia o \textit{data mining} é usado por companhias com grande foco em varejo, finanças, comunicação e marketing. Para conseguirem determinar as relações de fatores internos como preço, posição de produto, ou habilidade de recurso humano, e fatores externos como indicadores econômicos, competições e população demográfica de clientes \cite{mining-social-web}.

Esta análise de dados consiste em visualizar informações em diferentes maneiras e formas, plotando gráficos e planilhas. Neste momento, novas informações aparecerão permitindo alguma previsão ou predição desse conteúdo. As observações levarão a uma reflexão que resultará em possibilidades ou probabilidades concretas para se exercer uma atividade. No primeiro momento, essas informações são amorfas e, após a análise, se transformará em ideias \cite{han}.

Para que essas ideias se tornem um trabalho futuro é preciso capturá-las e interpretá-las através de um modelo de extração de conhecimento. Esse modelo, geralmente, é um processo que apresenta etapas que vão desde o armazenamento dos dados em estudos e passa por processos matemáticos, estatísticos e computacionais, com o objetivo de extrair informação útil. Um modelo, então, é muito mais que apenas a descrição dos dados, incorpora o entendimento de todo o processo da origem dos dados até a competência deles. Logo, ele consegue fazer previsões sobre os conhecimentos analisados \cite{han}.

Para conseguir fazer melhores previsões é preciso desenvolver métodos mais sofisticados antes de formular um modelo relevante. Com isso, a dificuldade aumenta e, então, é necessário implementar um modelo computacional que consiga obter possíveis resultados através do reconhecimento desses dados.

Dados é um termo, deliberadamente vago, que agrega várias formas comuns de dados, como por exemplo matrizes (vetores multidimensionais), tabelas ou planilhas, onde cada coluna pode ter um tipo diferente de informação (caracteres, numéricos, data, entre outros). Essas tabelas podem se relacionar através de colunas chaves apresentada no modelo relacional de \apudonline{codd}{data-command-line}.

Certamente que todos esses exemplos citados não demonstram a totalidade e nem toda a abordagem para a palavra dados. Não é sempre que grande percentual de um conjunto de dados pode ser transformados em uma forma estruturada, onde é possível ser analisados e modelados.

Cientistas de dados precisam visualizar os dados com o objetivo de produzir resultados claros e serem capazes de informar ao seus mantenedores sobre a situação atual e a qualquer momento. Este é o verdadeiro valor que um cientista nessa área precisa prover.

Para a análise e a interação de dados, computação exploratória e visualização de dados, a linguagem de programação Python vai, inevitavalmente, ser comparada a muitas outras, tanto no domínio de software livre, como também, com linguagens e ferramentas comerciais, como R, MATLAB, SAS, Stata e outros. Atualmente, o Python possui bibliotecas que se tornaram fortes alternativas para a tarefa de manipulação de dados. Combinado com o poder de programação que a linguagem tem, é uma excelente escolha como linguagem para a construção de aplicações centradas em dados \cite{python-analysis}.

Em muitas organizações, é comum realizar pesquisas, prototipar e testar novas ideias utilizando mais de um domínio específico de linguagem computacional, como MATLAB ou R e, posteriormente, estas ideias viram parte de um sistema de produção maior, escrito, por exemplo, em Java, C\#, ou C++. \citeonline{kaldero}, afirma que Python não é somente uma linguagem adequada para a pesquisa e prototipagem, mas também para o desenvolvimento de sistemas.

Devido a esta solução de apenas uma única linguagem, as organizações podem se beneficiar tendo cientistas e tecnólogos usando o mesmo conjunto de ferramentas programáticas. Portanto, Python é a ferramenta escolhida pela maioria desses profissionais. Essa escolha se deve, não somente a alta produtividade que a linguagem fornece, mas também por ela ser uma ferramenta comum a diferentes times e organizações \cite{kaldero}. 

Python é uma linguagem de programação livre e multiplataforma, possui uma excelente documentação e está sobre cuidado de uma enorme comunidade, onde é possível obter ajuda e melhores soluções para problemas durante a codificação. Tem como grande vantagem a facilidade de aprendizado, porque foi desenvolvida para ser simples e descomplicada. É uma linguagem interpretada, dinâmicamente tipada, com grande precisão e sintaxe eficiente. Tem grande popularidade para analisar dados devido ao enorme poder que suas bibliotecas possuem (\textit{NumPy}, \textit{SciPy}, \textit{pandas}, \textit{matplotlib}, \textit{IPython}). A linguagem apresenta alta produtividade para prototipação, desenvolvimento de sistemas menores e reaproveitáveis.

A mineração de dados busca, então, extrair dos dados o conhecimento útil para algum objetivo específico. Entretanto, a tarefa de extração de conhecimento é complexa devido a multidisciplinaridade envolvida no seu processo de extração e também por não ter um modelo de mineração genérico para a busca de informação útil. Para isso, é necessário o uso de ferramentas que viabilizam essas tarefas. Logo, Python dispõe de um conjunto de bibliotecas para a análise e mineração de dados extremamente poderosas e com uma curva de aprendizado curta graças a sintaxe clara e descomplicada que a linguagem fornece.


\section{JUSTIFICATIVA}\label{sec:justificativa}

\textit{LinkedIn} é uma rede social que possui foco em relacionamento profissional e negócios. De um modo ilustrativo, \textit{LinkedIn} se parece com um evento privado, onde os participantes possuem normas para vestimenta e estão extremamente bem vestidos, tentando demonstrar os valores específicos e expertises que possuem para trazer ao mercado profissional. Usuários desta rede estão principalmente interessados em oportunidades de negócios que este meio provê como uma forma arbitrária de socialização e irá, necessariamente, trazer detalhes sobre relações, históricos profissionais e outros.

Hoje, o \textit{LinkedIn} possui mais de 400 milhões de usuários e um crescimento de 40\%, a cada ano, de novos integrantes. Existem mais de 3 milhões de páginas de companhias e empresas, permitindo a conexão de 225 milhões de profissionais em sua rede. A cada dia, 200 conversas acontecem por minuto. Essas informações foram realizadas à dois anos atrás e demonstram um pouco do tamanho de informações que esta rede social possui \cite{linkedin-about}.

A mineração de dados do \textit{LinkedIn} ajuda no conhecimento de fortes conexões, novas habilidades a serem adquiridas, na comparação entre perfis e a busca para um cargo em específico, assim como a busca a uma companhia específica.

Para que os dados se tornem uma informação útil é preciso saber como coletar, processar, modelar e visualizar. Portanto, este estudo é tratado como uma área interdisciplinar que depende de uma arquitetura de \textit{software} apropriada, técnicas de processamento massivo de dados, algoritmos de redução de dimensionalidade, modelagem estatística e computacional, visualização de dados, entre outros.


\section{OBJETIVOS}\label{sec:objetivos}

\subsection{Objetivo Geral}

Este trabalho tem como objetivo principal desenvolver soluções e algoritmos utilizando a linguagem Python para analisar a rede social \textit{LinkedIn}, tendo como foco a predição de um perfil de um profissional de tecnologia da informação, a verificação de fortes conexões e construção de um perfil mais influente na rede.

\subsection{Objetivos Específicos}\label{subsec:objetivos_especificos}

\begin{itemize}
    \item Apresentar a API (interface de programação de aplicações) do \textit{LinkedIn};
    \item Apresentar a linguagem Python e as bibliotecas que ela possui para a análise de dados;
    \item Estudo e manipulação de APIs de redes sociais e seus métodos de autenticação;
    \item Implementar algoritmos para a mineração de dados;
    \item Encontrar padrões com a utilização de algoritmos de aprendizado de máquina;
    \item Apresentar os testes e resultados obtidos com a execução de algoritmos.
\end{itemize}

\section{CRONOGRAMA DE ATIVIDADES}\label{subsec:cronograma}

As atividades a serem executadas no decorrer do projeto visando o êxito do mesmo, estão listados a seguir e especificados em meses na Tabela~\ref{cronograma}:

\begin{itemize}
  \item Estudo e Pesquisa: aquisição dos conhecimentos pertinentes e necessários para o desenvolvimento do projeto;
  \item Análise de Requisitos: levantamento dos requisitos do projeto;
  \item Geração de Documentação: desenvolvimento das documentações para especificação do projeto;
  \item Implementação: desenvolvimento dos códigos para a análise de dados;
  \item Testes: execução dos testes que irão garantir a qualidade das informações a serem geradas;
  \item Elaboração de Artigos: parte do tempo destinado ao projeto será para desenvolver artigos visando a publicação em eventos da área;
  \item Apresentação de Resultados: etapas destinadas à apresentação dos resultados parciais e finais.
\end{itemize}


\renewcommand{\arraystretch}{1.5}

\begin{table}[h!]
  \centering
  \caption{\textsc{Cronograma de execução}}
  \vspace{-0.3cm}
  \legend{\small{FONTE: Elaborado pelo autor}}
  \begin{tabular}{ | l | c | c | c | c | c | c | c | c | c | }
    \hline
    \textbf{Meses} & \textbf{1} & \textbf{2} & \textbf{3} & \textbf{4} & \textbf{5} & \textbf{6} & \textbf{7} & \textbf{8} & \textbf{9} \\ \hline
    Estudo e Pesquisa & X & X & X & X & X & X & X & X &  \\ \hline
    Análise de Requisitos & X & X & X & X & X & X & X & X &  \\ \hline
    Geração do Documento & X & X & X & X & X & X & X & X & X \\ \hline
    Implementação &  &  &  & X & X & X & X & X & X \\ \hline
    Testes &  &  &  & X & X & X & X & X & X \\ \hline
    Elaboração de Artigos &  &  & X & X & X &  &  & X & X \\ \hline
    Apresentação de Resultados &  &  &  &  & X &  &  &  & X \\
    \hline
  \end{tabular}
  \label{cronograma}
\end{table}


\section{ORGANIZAÇÃO DO TRABALHO}\label{sec:organizacao-trabalho}

Além deste capítulo introdutório, este trabalho é composto de mais seis capítulos.

No capítulo 2 é apresentado as citações de trabalhos que foram referências bibliográficas para este estudo.

Os fundamentos teóricos, como os conceitos de \textit{data mining}, e base para o entendimento do tema proposto estão descritos no capítulo 3.

As bibliotecas de Python utilizadas para a mineração de dados é exposta no capítulo 4. Também é apresentado a API do \textit{LinkedIn}, etapas de \textit{data mining} e outros materiais e metodologias utilizados para execução deste trabalho.

No capítulo 5 é demonstrado as fases de desenvolvimento de algoritmos para a implementação do \textit{data mining} e \textit{machine learning}.

Os resultados obtidos e a apresentação de planilhas e gráficos das soluções desenvolvidas são apresentados no capítulo 6.

Por fim, a conclusão do trabalho se dá no capítulo 7, sendo nesse abordado e analisado as dificuldades, além de determinadas as possibilidade para trabalhos posteriores.
