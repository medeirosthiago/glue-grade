%-----------------------------------------------------------------
% Modelo de Tese/Dissertação
% Autora: Fabiana Frata Furlan Peres 
% 2014 
%-----------------------------------------------------------------
%% ADAPTADO DE:
%% modelo de Documento TCC da Unioeste   
%%
%% que por sua vez foi ADAPTADO DE: 
%% abtex2-modelo-trabalho-academico.tex, v-1.6 laurocesar
%% Copyright 2012-2013 by abnTeX2 group at http://abntex2.googlecode.com/ 
% ------------------------------------------------------------------------
% abnTeX2: Modelo de Trabalho Academico (tese de doutorado, dissertacao de
% mestrado e trabalhos monograficos em geral) em conformidade com 
% ABNT NBR 14724:2011: Informacao e documentacao - Trabalhos academicos - Apresentacao
%
% ------------------------------------------------------------------------
%DECLARAÇÃO DO TIPO DE DOCUMENTO, TAMANHO DA FOLHA E FONTE
%-----------------------------------------------------------------
\documentclass[12pt,oneside,a4paper,english,french,spanish]{abntex2}

%-----------------------------------------------------------------
%DEFINIÇÃO DOS PACOTES UTILIZADOS
%-----------------------------------------------------------------
\usepackage{cmap}				% Mapear caracteres especiais no PDF
\usepackage{lmodern}			% Usa a fonte Latin Modern			
\usepackage[T1]{fontenc}		% Selecao de codigos de fonte.
\usepackage[utf8]{inputenc}		% Codificacao do documento (conversão automática dos acentos)
\usepackage{lastpage}			% Usado pela Ficha catalográfica
\usepackage{indentfirst}		% Indenta o primeiro parágrafo de cada seção.
\usepackage{color}				% Controle das cores
\usepackage{graphicx}			% Inclusão de gráficos

\usepackage{lipsum}			

\usepackage{amsmath}
\usepackage{amssymb}






% Configuração para código:
%
% Adição de códigos
\usepackage{listings}
\usepackage{color}

\definecolor{mygreen}{rgb}{0,0.6,0}
\definecolor{mygray}{rgb}{0.5,0.5,0.5}
\definecolor{mymauve}{rgb}{0.58,0,0.82}

\lstset{ %
	backgroundcolor=\color{white},   % choose the background color; you must add \usepackage{color} or \usepackage{xcolor}
	basicstyle=\footnotesize\ttfamily,        % the size of the fonts that are used for the code
	breakatwhitespace=false,         % sets if automatic breaks should only happen at whitespace
	breaklines=true,                 % sets automatic line breaking
	captionpos=b,                    % sets the caption-position to bottom
	commentstyle=\color{mygreen},    % comment style
	deletekeywords={...},            % if you want to delete keywords from the given language
	escapeinside={\%*}{*)},          % if you want to add LaTeX within your code
	extendedchars=true,              % lets you use non-ASCII characters; for 8-bits encodings only, does not work with UTF-8
	frame=single,	                   % adds a frame around the code
	keepspaces=true,                 % keeps spaces in text, useful for keeping indentation of code (possibly needs columns=flexible)
	keywordstyle=\color{blue},       % keyword style
	language=Python,                 % the language of the code
	otherkeywords={*,...},           % if you want to add more keywords to the set
	numbers=left,                    % where to put the line-numbers; possible values are (none, left, right)
	numbersep=5pt,                   % how far the line-numbers are from the code
	numberstyle=\tiny\color{mygray}, % the style that is used for the line-numbers
	rulecolor=\color{black},         % if not set, the frame-color may be changed on line-breaks within not-black text (e.g. comments (green here))
	showspaces=false,                % show spaces everywhere adding particular underscores; it overrides 'showstringspaces'
	showstringspaces=false,          % underline spaces within strings only
	showtabs=false,                  % show tabs within strings adding particular underscores
	stepnumber=1,                    % the step between two line-numbers. If it's 1, each line will be numbered
	stringstyle=\color{red},     % string literal style
	tabsize=1,	                   % sets default tabsize to 2 spaces
	title=\lstname                   % show the filename of files included with \lstinputlisting; also try caption instead of title
}

     
\sloppy

%
%\lstset{language=Python}
%colocar dentro do documento begin-end


% Fonte "Arial"
\usepackage{helvet}
\renewcommand{\familydefault}{\sfdefault}

% Pacotes de citações
\usepackage[alf,abnt-emphasize=bf,bibjustif]{abntex2cite}	% Citações padrão ABNT
% ------------------------
% CONFIGURAÇÕES DA APARENCIA DO PDF FINAL
% ------------------------ 
\definecolor{blue}{RGB}{41,5,195}% alterando o aspecto da cor azul
% informações do PDF
\makeatletter
\hypersetup{     	
		pdftitle={\@title}, 
		pdfauthor={\@author},
    	pdfsubject={\imprimirpreambulo},
	    pdfcreator={LaTeX with abnTeX2},
		pdfkeywords={abnt}{latex}{abntex}{abntex2}{trabalho acadêmico}, 
		colorlinks=true,       		% false: boxed links; true: colored links
    	linkcolor=black,          	% color of internal links
    	citecolor=black,        		% color of links to bibliography
    	filecolor=magenta,      		% color of file links
		urlcolor=blue,
		bookmarksdepth=4
}
\makeatother

% ---
% INFORMAÇÕES REFERENTE A DISSERTAÇÃO
% ---
\titulo{TCC sobre Java}
\autor{Jorge Silva \\ Silva Matheus}
\orientador{Msc. Nome do Orientador}
\coorientador{Prof. Esp. Nome do Coorientador}
\data{2016}
 % DADOS para CAPA e FOLHA DE ROSTO


% !TeX encoding = UTF-8

\preambulo{Trabalho de conclus{\~ a}o de curso apresentado como requisito obrigat{\' o}rio para obten{\c c}{\~ a}o do t{\' i}tulo de Bacharel em Ci{\^ e}ncia da Computa{\c c}{\~ a}o da Faculdade Anglo-Americano de Foz do Igua{\c c}u.}
\local{Foz do Igua{\c c}u}
\instituicao{Faculdade Anglo-Americano de Foz do Igua{\c c}u}


% ---
% CAPA
% ---
\renewcommand{\imprimircapa}{%
  \begin{capa}
  	\begin{center}
  		\textbf{\large{\MakeUppercase{\imprimirinstituicao}}}
  	\end{center}
    \center

    {\normalsize\MakeUppercase{\imprimirautor}}

    \vspace*{8cm}
    \ABNTEXchapterfont\bfseries\large\MakeUppercase{\imprimirtitulo}
    \vfill
    
    {\normalfont\small\MakeUppercase{\imprimirlocal}}

    {\normalfont\small\imprimirdata}
    
    \vspace*{1cm}
  \end{capa}
}

\makeatletter
% ---
% FOLHA DE ROSTO
% ---
\renewcommand{\folhaderostocontent}{
		\center	
		{\normalsize\MakeUppercase{\imprimirautor}}
	
	    \vspace*{8cm}		
		
		{\large\MakeUppercase{\imprimirtitulo}}
	
		\vspace*{3cm}	

		\abntex@ifnotempty{\imprimirpreambulo}{
			\hspace{.3\textwidth}
			\begin{minipage}{.55\textwidth}
				{\footnotesize \SingleSpacing
				\imprimirpreambulo				
				\SingleSpacing				
				{\imprimirorientadorRotulo~\imprimirorientador\par}
				\SingleSpacing	
		        \abntex@ifnotempty{\imprimircoorientador}{
		             {\imprimircoorientadorRotulo~\imprimircoorientador} } }
			\end{minipage}
		
			\vspace*{\fill}
		}
		
		{\normalfont\small\MakeUppercase{\imprimirlocal}}
		
		{\normalfont\small\imprimirdata}		
}


% ---
% OUTRAS CONFIGURAÇÕES
% ---
\setlength{\parindent}{1.3cm} 		% Tamanho do parágrafo
\setlength{\parskip}{0.2cm}  		% Controle do espaçamento entre um parágrafo e outro

% Controle do espaçamento entre a legenda e a fonte (e nota) de figuras e tabelas}:
\setlength{\abovecaptionskip}{0 mm}
\setlength{\belowcaptionskip}{1 mm}

% Define os tamanhos das margens do documento
% ---
\setlrmarginsandblock{3cm}{2cm}{*}
\setulmarginsandblock{3cm}{2cm}{*}
\checkandfixthelayout

	
% Altera o tamanho das fontes dos capítulos e dos apêndices
% ---
\renewcommand{\ABNTEXchapterfont}{\bfseries}
\renewcommand{\ABNTEXchapterfontsize}{\large}
\renewcommand{\ABNTEXsectionfontsize}{\normalfont}

% ---
% SEPARAÇÃO DE PALAVRAS
% ---
%\hyphenation{}
\hyphenation{teorias}
\hyphenation{baseados}
\hyphenation{MATLAB}
\hyphenation{Pyramid}
\hyphenation{Mar-cação}
\hyphenation{lin-guagem}
\hyphenation{matplotlib}
\hyphenation{sociali-zação}
\hyphenation{bi-bli-oteca}
\hyphenation{Notebook}
\hyphenation{profis-si-onais}
\hyphenation{ar-quivos}
\hyphenation{usu-ário}
\hyphenation{fer-ramentas}
\hyphenation{materi-al}
\hyphenation{re-almen-te}
\hyphenation{LinkedIn}
\hyphenation{proces-samento}
\hyphenation{neces-sidades}
\hyphenation{relaci-onados}

% ---
% CONFIGURACAO DO ESTILO DE PÁGINA TEXTUAL 
% ---
% criar um novo estilo de cabeçalhos e rodapés
\makepagestyle{abnt-anglo}
  %%cabeçalhos
  \makeevenhead{abnt-anglo}{\ABNTEXfontereduzida\thepage}{}{}
  \makeoddhead{abnt-anglo}{}{}{\ABNTEXfontereduzida\thepage}
  %\makeheadrule{textualCC-Anglo}{\textwidth}{\normalrulethickness} %linha
  %% rodapé
  \makeevenfoot{abnt-anglo}{}{}{} 
  \makeoddfoot{abnt-anglo}{}{}{}
\pagestyle{abnt-anglo}

% ---
% utfpr-style
%\makepagestyle{abnt_ufpr}
%
%\makeoddhead{abntchapfirst}{}{}{\ABNTEXfontereduzida\thepage}
%
%\pagestyle{abnt_ufpr}
%
%\newtheorem{theorem}{Teorema}[chapter]
%\newtheorem{definition}[theorem]{Defini\c{c}\~{a}o}
%\newtheorem{proposition}[theorem]{Proposi\c{c}\~{a}o}

\makeatother

\usepackage{perpage} %the perpage package
\MakePerPage{footnote} %the perpage package command

% ---
% CONFIGURAÇÃO DE LISTAS DE CONTEÚDO
% ---
% lista de gráficos
% -
\newcommand{\graficoname}{GRÁFICO}
\newcommand{\listofgraficosname}{LISTA DE GRÁFICOS}

\newfloat[chapter]{grafico}{loq}{\graficoname}
\newlistof{listofgraficos}{loq}{\listofgraficosname}
\newlistentry{grafico}{loq}{0}

\counterwithout{grafico}{chapter}		% configurações para atender às regras da ABNT em listas
\renewcommand{\cftgraficoname}{\graficoname\space}
\renewcommand*{\cftgraficoaftersnum}{\hfill--\hfill}

% lista de códigos
% -
\renewcommand{\lstlistingname}{CÓDIGO}
\renewcommand{\lstlistlistingname}{LISTA DE CÓDIGOS}

\begingroup\makeatletter				% configurações para atender às regras da ABNT em listas
\let\newcounter\@gobble\let\setcounter\@gobbletwo
  \globaldefs\@ne \let\c@loldepth\@ne
  \newlistof{listings}{lol}{\lstlistlistingname}
  \newlistentry{lstlisting}{lol}{0}
\endgroup

\renewcommand{\cftlstlistingaftersnum}{\hfill--\hfill}

\let\oldlstlistoflistings\lstlistoflistings
\renewcommand{\lstlistoflistings}{%
   \begingroup%
   \let\oldnumberline\numberline%
   \renewcommand{\numberline}{\lstlistingname\space\oldnumberline}%
   \oldlstlistoflistings%
   \endgroup}

%\setlength{\parindent}{1.3cm} 			% Tamanho do parágrafo
%\setlength{\parskip}{0.0cm}  			% Controle do espaçamento entre um parágrafo e outro

% Altera o tamanho das fontes dos capítulos e dos apêndices
% -
\renewcommand{\ABNTEXchapterfont}{\normalfont\fontseries{b}\selectfont}
\renewcommand{\ABNTEXchapterfontsize}{\normalsize}
\renewcommand{\ABNTEXpartfont}{\fontseries{b}\selectfont\selectfont}
\renewcommand{\ABNTEXpartfontsize}{\normalsize}
\renewcommand{\ABNTEXsectionfont}{\normalfont\selectfont}
\renewcommand{\ABNTEXsectionfontsize}{\normalsize}
\renewcommand{\ABNTEXsubsectionfont}{\normalfont\selectfont}
\renewcommand{\ABNTEXsubsectionfontsize}{\normalsize}
\renewcommand{\ABNTEXsubsubsectionfont}{\normalfont\selectfont}
\renewcommand{\ABNTEXsubsubsectionfontsize}{\normalsize}
\renewcommand{\ABNTEXsubsubsubsectionfont}{\normalfont\itshape\selectfont}
\renewcommand{\ABNTEXsubsubsubsectionfontsize}{\normalsize}

% ---
% CONFIGURAÇÃO DO SUMÁRIO
% ---
% Sumário
\renewcommand*{\cftsectionfont}{\normalfont}
\renewcommand*{\cftsubsubsectionfont}{\normalfont}
\renewcommand*{\cftsubsectionfont}{\normalfont}
\renewcommand*{\cftparagraphfont}{\normalfont\itshape}

% ---
% Modifica o espaçamento no sumário
% Nao ha espacos, para as entradas de capitulos
% ---
\setlength{\cftbeforechapterskip}{\onelineskip}
\setlength{\cftbeforesectionskip}{0pt}
\setlength{\cftbeforesubsectionskip}{0pt}
\setlength{\cftbeforesubsubsectionskip}{0pt}
\setlength{\cftbeforeparagraphskip}{0pt}

%\setlength{\cftbeforeparagraphskip}{0pt}
%\setlength{\cftbeforesubsectionskip}{0pt}
%\setlength{\cftbeforesectionskip}{0pt}
%\setlength{\cftbeforesubsubsectionskip}{0pt}
%\setlength{\cftbeforechapterskip}{0pt}

% ---
% CAPITALIZAÇÃO DE LISTAS
% ---
\renewcommand{\fontename}{FONTE}

\addto\captionsbrazil{
	\renewcommand{\bibname}{REFER\^ENCIAS}
}
\addto\captionsbrazil{\renewcommand{\listadesiglasname}{LISTA DE ABREVIATURAS}}
\addto\captionsbrazil{\renewcommand{\listfigurename}{LISTA DE ILUSTRAÇÕES}}
\addto\captionsbrazil{\renewcommand{\listtablename}{LISTA DE TABELAS}}
%\addto\captionsbrazil{%
%  \renewcommand*{\lstlistlistingname}{LISTA DE CÓDIGOS}%
%  \renewcommand*{\lstlistingname}{CÓDIGO}%
%}
\addto\captionsbrazil{\renewcommand{\contentsname}{SUMÁRIO}}
\addto\captionsbrazil{\renewcommand\appendixtocname{APÊNDICES}\renewcommand\appendixpagename{APÊNDICES}}
\addto\captionsbrazil{\renewcommand{\figurename}{FIGURA}}
\addto\captionsbrazil{\renewcommand{\tablename}{TABELA}}

% ---
% LAYOUT PARA ELEMENTOS TEXTUAIS
% ---
%\makepagestyle{abnt_ufpr}

%\makeoddhead{abntchapfirst}{}{}{\ABNTEXfontereduzida\thepage}

%\pagestyle{abnt_ufpr}

% ---
% AMBIENTES
% ---
%\newtheorem{teo}{Teorema}[chapter]
%\newtheorem{cor}[teo]{Corol\'{a}rio}
%\newtheorem{lem}[teo]{Lema}
%\newtheorem{prop}[teo]{Proposi\c{c}\~{a}o}
%\newtheorem{defn}[teo]{Defini\c{c}\~{a}o}
%\newtheorem{Ex}[teo]{Exemplo}
%\newtheorem{obs}[teo]{Observa\c{c}\~{a}o}
%\newtheorem{prob}[teo]{Problema}
%\newtheorem{conc}[teo]{Conclusão}
%\newenvironment{dem}{\smallskip \noindent{\bf Demonstra\c{c}\~{a}o}: }
%{\hfill $\Box$\hspace{0in}\medskip}

% ---
% COMANDOS FREQUENTES
% ---
%\newcommand{\eq}{\begin{equation}}
%	\newcommand{\ee}{\end{equation}}
%\newcommand{\R}{{\mathbb R}}
%\newcommand{\N}{{\mathbb N}}
%\newcommand{\K}{{\mathbb K}}
%\newcommand{\Q}{{\mathbb Q}}
%\newcommand{\Z}{{\mathbb Z}}
%\newcommand{\V}{{\mathbb V}}
%\newcommand{\D}{{\mathcal{D}}}
%\newcommand{\C}{{\mathbb C}}
%\newcommand{\di} {\displaystyle}
%\newcommand{\I}{{\displaystyle\int_{0}^{T} \displaystyle\int_{0}^1 }}
%\newcommand{\Ia}{{\displaystyle\int_{0}^{1} \displaystyle\int_{0}^T }}
%\newcommand{\Ii}{{\displaystyle\int_{0}^{t} \displaystyle\int_{0}^1 }}
%\newcommand{\Ib}{{\displaystyle\int_{0}^{1} \displaystyle\int_{0}^t }}

%\makeatother % Configurações DO DOCUMENTO

\makeindex
% ------------------------
% Início do documento
% ------------------------ 
\begin{document}
\frenchspacing % Retira espaço extra obsoleto entre as frases.

% ----------------------------------------------------------
% ELEMENTOS PRÉ-TEXTUAIS
% ----------------------------------------------------------
\imprimircapa
\imprimirfolhaderosto* % (o * indica que haverá a ficha bibliográfica)

%\begin{fichacatalografica}

	\vspace*{15cm}       %  Posição  vertical

	\hrule %  Linha  horizontal

	\begin{center}       %  Minipage  Centralizado

	\begin{minipage}[c]{12.5cm}  %  Largura
	
	SobreNome, Nome1 Nome2 % Nome de referência. Por ex. Silva, João Paulo
 
	\hspace{0.5cm}  \imprimirtitulo~/~\imprimirautor~--~\imprimirlocal,  \imprimirdata.
	
	\hspace{0.5cm}  \pageref{LastPage}  p.  :  il.\\

	\hspace{0.5cm}  \imprimirorientadorRotulo ~\imprimirorientador\\

	\hspace{0.5cm}

\parbox[t]{\textwidth}{\imprimirtipotrabalho ~--~ Universidade Estadual do Oeste do Paraná (UNIOESTE). Centro de Engenharias e Ciências Exatas (CECE). Curso de Ciência da Computação, \imprimirdata.}\\

\hspace{0.5cm}
	1.  Palavra-chave1.
	2.  Palavra-chave2.
	I.  Orientador.
	II.  Universidade Estadual do Oeste do Paraná (UNIOESTE).
	III. Centro de Engenharias e Ciências Exatas (CECE).
	IV. Curso de Ciência da Computação.
	V.  \imprimirtitulo\\
	
	\hspace{8.75cm}  CDU  02:141:005.7\\

	\end{minipage}
	\end{center}
	\hrule
\end{fichacatalografica}

\begin{folhadeaprovacao} %[TERMO DE APROVAÇÃO]
\begin{center}
	\vspace*{1cm}  
  	\large\textbf{TERMO DE APROVAÇÃO}
  	
  	\vspace*{1cm} %minha mudança
  	%\vspace*{2cm}
  	{\large\textbf\imprimirautor}

   \vspace*{1cm} %minha modif
   %\vspace*{2cm}
    {\large\textbf\imprimirtitulo}   
 \end{center}     
  
	
	\hspace{.4\textwidth}
	\SingleSpace\noindent\normalsize{Trabalho de conclusão de curso apresentado como requisito obrigatório para obtenção do título de Bacharel em Ciência da Computação da Faculdade Anglo-Americano de Foz do Iguaçu, pela seguinte banca examinadora:}
%	\noindent\paragraph
   
 %  \end{center}
    
  
   %\vspace*{1.5cm}
   \vspace*{0.5cm}  %minha modif
   \assinatura{{\imprimirorientador}\\Faculdade Anglo-Americano\\(Orientador)}
   \assinatura{Prof. Banca 2\\Faculdade Anglo-Americano}
   \assinatura{Prof. Banca 3\\Faculdade Anglo-Americano} 
   \vspace*{2.5cm}
   \begin{center}
   	%{\imprimirlocal, \ \imprimirdata}
   	{Foz do Iguaçu, 30 de novembro de 2015}
   \end{center}
   
 
\end{folhadeaprovacao}
\begin{dedicatoria}
   \vspace*{\fill}
   \begin{flushright}
   	\textit{A todos os que tentam a mais de meia década se graduar, \\ Eu consegui!}
   	
   \end{flushright}
\end{dedicatoria}

% ---
% Agradecimentos
% ---
\begin{agradecimentos}[AGRADECIMENTOS]

Agradecimentos para serem agradecidos...

\end{agradecimentos}


% ---
% Epígrafe
% ---
\begin{epigrafe}
    \vspace*{\fill}
	\begin{flushright}
		\textit{"Apenas que... Busquem conhecimento." \\
		(E.T. Bilu)}
	\end{flushright}
\end{epigrafe}


\vspace*{-0.65cm}
% resumo em português
\begin{resumo}[RESUMO]	
Este trabalho tem como objetivo analisar e minerar os dados provenientes da rede social \textit{Twitter}, com a finalidade de encontrar e apresentar informações sobre os usuários que realizaram \textit{tweets} e os resultados encontrados na rede social, durante a votação do Congresso Brasileiro, para a continuidade do processo de Impeachment da presidente Dilma Rousseff. Para a realização desta atividade será utilizada a linguagem de programação Python como ferramenta principal para o estudo e implementação prática de métodos de aprendizado de máquina para o reconhecimento e apresentação dos dados.


 \vspace{\onelineskip}
    
 \noindent
 \textbf{Palavras-chaves}: Dados. Data Mining. Twitter. Python.
 % 4 palavras separadas por . (ponto)
\end{resumo}

\vspace*{-0.65cm}
\begin{resumo}[ABSTRACT]
 \begin{otherlanguage*}{english}
This paper aims to analyze and mine the data from the social network \textit{LinkedIn}, in order to find patterns in Information Technology professional's profiles. For this activity will be used Python as the main programming language as well as a tool for the study and practical implementation of machine learning methods for data recognition and presentation.
   
   \vspace{\onelineskip}
 
   \noindent 
   \textbf{Keywords}: Data. Data Mining. LinkedIn. Python.
 \end{otherlanguage*}
\end{resumo}


\pdfbookmark[0]{\listfigurename}{lof}% inserir lista de ilustrações
\listoffigures*
\cleardoublepage

\pdfbookmark[0]{\listtablename}{lot}% inserir lista de tabelas
\listoftables*
\cleardoublepage

% ---
% SIGLAS
% ---
\begin{siglas}
	\item[API] \textit{Application Programming Interface} - Interface de Programação de Aplicação
	\item[BMP] \textit{Windows Bitmap}
	\item[CGI] \textit{Common Gateway Interface} - Interface Comum de Entrada\footnote{Tradução do autor}
	\item[CSV] \textit{Comma-Separated Values} - Valores Separados Por Vírgula\footnotemark[1]
	\item[DBA] \textit{Database Administrator} - Administrador de Banco de Dados
	\item[FTP] \textit{File Transfer Protocol} - Protocolo de Transferência de Arquivos
	\item[GIF] \textit{Graphics Interchange Format} - Formato Para Intercâmbio de Gráficos\footnotemark[1]
	\item[GUI] \textit{Graphical User Interface} - Interface Gráfica do Usuário
	\item[HTTP] \textit{Hypertext Transfer Protocol} - Protocolo de Transferência de Hipertexto
	\item[HTTPS] \textit{Hyper Text Transfer Protocol Secure} - Protocolo de Transferência de Hipertexto Seguro
	\item[IETF] \textit{Internet Engineering Task Force}
	\item[IMAP] \textit{Internet Message Access Protocol} - Protocolo de Acesso a Mensagem da Internet
	\item[IP] \textit{Internet Protocol} - Protocolo de Internet
	\item[JPG] \textit{Joint Photographic Experts Group}
	\item[JSON] \textit{JavaScript Object Notation} - Notação de Objeto JavaScript\footnotemark[1] 
	\item[KDD] \textit{Knowledge Discovery From Data} - Descoberta de Conhecimento por Dados
	\item[NLTK] \textit{Natural Language Toolkit} - Ferramentas de Linguagem Natural\footnotemark[1]
	\item[PDF] \textit{Portable Document Format} - Formato de Documento Portátil\footnotemark[1]
	\item[PNG] \textit{Portable Network Graphics} - Rede Portável de Gráficos\footnotemark[1]
	\item[POP] \textit{Post Office Protocol} - Protocolo dos Correios
	\item[RFC] \textit{Request for Comments} - Pedido Para Comentários
	\item[RPC] \textit{Remote Procedure Call} - Chamada Remota de Procedimento\footnote{Tradução do autor}
	\item[SMTP] \textit{Simple Mail Transfer Protocol} - Protocolo de Transferência de Correio Simples
	\item[SSL] \textit{Secure Sockets Layer} - Camada Segura de Sockets
	\item[SVG] \textit{Scalable Vector Graphics} - Gráficos Vetoriais Escaláveis
	\item[TCP] \textit{Transmission Control Protocol} - Protocolo de Controle de Transmissão
	\item[URI] \textit{Uniform Resource Identifier} - Identificador Uniforme de Recursos
	\item[URL] \textit{Uniform Resource Locator} - Localizador Padrão de Recursos
	\item[XHTML] \textit{eXtensible Hypertext Markup Language} - Linguagem de Marcação de Hipertexto Extensiva
	\item[XML] \textit{eXtensible Markup Language} - Linguagem de Marcação Extensiva
	\item[YML] \textit{Yet Another Markup Language} - Uma Outra Linguagem de Marcação\footnotemark[2]
\end{siglas}

  % inserir lista de abreviaturas e siglas

%\input{Configuracoes/simbolos}% inserir lista de símbolos

\vspace*{0cm}
\pdfbookmark[0]{\contentsname}{toc}% inserir o sumario
\tableofcontents*
\cleardoublepage


% ----------------------------------------------------------
% ELEMENTOS TEXTUAIS
% ----------------------------------------------------------
\textual

% !TeX encoding = UTF-8

\chapter{INTRODUÇÃO}\label{ch:introducao}

Redes sociais se tornaram um termo comum e uma chave fundamental para o estilo de vida moderno. Hoje em dia, ....


\section{JUSTIFICATIVA}\label{sec:justificativa}
A rede social \textit{Twitter} é um excelente ponto de partida para a mineração de dados em redes sociais, ....

A rede social possui um total de 289 milhões de usuários ativos no mundo inteiro, totalizando 58 milhões de \textit{tweets} por dia \cite{statistics}.


\section{OBJETIVOS}\label{sec:objetivos}

\subsection{Objetivo Geral} 
Este trabalho tem como objetivo principal utilizar técnicas e algoritmos de \textit{data mining}, para a análise e mineração de dados provenientes da rede social \textit{Twitter}, utilizando os recursos e bibliotecas que a linguagem de programação Python possui.

\subsection{Objetivos Específicos}\label{subsec:objetivos_especificos}
\begin{itemize}
	\item Identificar os conceitos sobre KDD e \textit{data mining};
	\item Descrever as técnicas de \textit{data mining};
	\item Explorar as funcionalidades das bibliotecas de mineração e visualização da linguagem Python;
	\item Examinar e utilizar a API da rede social \textit{Twitter} para a coleta de dados;
	\item Encontrar padrões em dados provenientes do \textit{Twitter};
	\item Compreender e aplicar técnicas para apresentação e visualização de informações geográficas encontradas nos dados coletados;
	\item Apresentar testes e resultados obtidos da análise e mineração dos dados.
\end{itemize}

\section{CRONOGRAMA DE ATIVIDADES}\label{subsec:cronograma}

As atividades a serem executadas no decorrer do projeto visando o êxito do mesmo, estão listados a seguir e especificados em meses na \autoref{cronograma}:

\begin{table}[h!]
	\centering
	\caption{Cronograma}
	\includegraphics[width=1\textwidth]{tabela}
	\fonte{Autor}
	\label{crono}
\end{table}

\begin{itemize}
  \item Estudo e Pesquisa: aquisição dos conhecimentos pertinentes e necessários para o desenvolvimento do projeto;
  \item Análise de Requisitos: levantamento dos requisitos do projeto;
  \item Geração do Documento: desenvolvimento das documentações para especificação do projeto;
  \item Implementação: desenvolvimento dos códigos para a análise de dados;
  \item Testes: execução dos testes que irão garantir a qualidade das informações a serem geradas;
  \item Elaboração de Artigos: parte do tempo destinado ao projeto será para desenvolver artigos visando a publicação em eventos da área;
  \item Apresentação de Resultados: etapas destinadas à apresentação dos resultados parciais e finais.
\end{itemize}



\begin{table}[h!]
	\centering
	\caption{Cronograma de execução}
	\includegraphics[width=1\textwidth]{tabela}
	\fonte{Autor}
	\label{cronograma}
\end{table}


\section{ORGANIZAÇÃO DO TRABALHO}\label{sec:organizacao-trabalho}

Além deste capítulo \autoref{crono}, este trabalho é composto de mais seis capítulos.

O Capítulo 2 apresenta os trabalhos que são referências para este estudo.

Os fundamentos teóricos, como os conceitos de \textit{data mining} e base para o entendimento do tema proposto, estão descritos no Capítulo 3.

No Capítulo 5 são apresentadas as fases do desenvolvimento ....

Os resultados obtidos e a apresentação de planilhas e gráficos das soluções desenvolvidas são apresentados no Capítulo 6.

Por fim, a conclusão deste trabalho se dá no Capítulo 7, onde são abordadas e analisadas as dificuldades, além de determinar as possibilidades para trabalhos futuros.


% ------------------------
% Capítulos da Monografia
% ------------------------
\chapter{REVISÃO BIBLIOGRÁFICA}\label{ch:rev-bibs}

Alguns trabalhos serviram como ajuda e inspiração para este estudo. Porém durante o período de busca por bibliografias a respeito de \textit{data mining}, pouco material foi encontrado quanto a mineração em redes sociais. Ainda sim, alguns estudos possuem um destaque e é necessário citá-los, devido a utilização de ferramentas específicas em \textit{data mining} e, também, aos resultados obtidos neste processo.

De acordo com \citeonline{credito-bancario}, um dado se transforma em informação quando ganha um significado para seu utilizador, caso contrário, continua sendo simplesmente um dado.

Em seu estudo, \citeonline{credito-bancario} aborda duas técnicas de \textit{data mining}: Árvores de Decisão e Redes Neurais, para realizar a análise de crédito bancário. Estas técnicas permitem fazer o reconhecimento de padrões e também diagnosticar novos casos. Através deste modo, os analistas têm condições de diagnosticar os novos clientes, quanto ao merecimento de crédito ou não. Estas técnicas de mineração de dados são ferramentas que podem ser utilizadas pelo especialista para auxiliá-lo nas tomadas de decisão, nunca porém poderão por si só, substituir a figura do especialista no contexto da análise de crédito.

Semelhante as técnicas abordadas para a análise de crédito bancário, \citeonline{diag-medico} realizaram um estudo na área médica, onde a posse e uso de ferramentas que auxiliem na tarefa de classificação de pacientes em prováveis ictéricos com câncer ou ictéricos com cálculo, pode ser crucial. Em uma tentativa de otimizar todo o processo do diagnóstico, minimizando riscos e custos e, por outro lado, maximizando a eficácia nos resultados, utilizaram técnicas de  \textit{data mining} como um processo para extração de informações valiosas. O trabalho consistiu na análise de 118 históricos de pacientes utilizando Árvores de Decisão e Regras de Classificação como ferramenta. Os autores afirmam que os métodos de \textit{data mining} apresentam a vantagem de deixar claro ao usuário quais são os atributos que estão discriminando os padrões e de que forma a mesma está ocorrendo, isso é uma característica altamente desejável em qualquer técnica de reconhecimento de padrões.

O reconhecimento de padrões permite a obtenção de conhecimento da frequência com que determinadas seções de uma página \textit{web} são acessadas e quais são os serviços mais procurados. Isso possibilita que empresas consigam descobrir o perfil de seus usuários e, com base nesse acontecimento, ofertar serviços e atendimento personalizado. Essa afirmação foi resultado do estudo de \citeonline{logs-web}, onde realizaram a mineração de dados em \textit{logs} de acesso a servidores \textit{web}. Para o desenvolvimento do trabalho utilizaram regras de associação para a descoberta e representação de padrões frequentes em conjuntos de dados. Isso propiciou a identificação de padrões de comportamento de usuários da Internet ao navegarem por \textit{websites}. Também concluem que outras ferramentas de mineração podem ser aplicadas, visando aumentar a flexibilidade de manipulação dos atributos específicos para o ambiente \textit{Web}.

Como visto, o tema sobre \textit{data mining} possui uma abordagem comum a setores e áreas diferentes, o que caracteriza a enorme quantidade de dados que ainda não se tem informação útil para seus utilizadores. Por mais que as técnicas utilizadas dentre essa diversidade de áreas sejam semelhantes, o desenvolvimento e aplicação das técnicas são particulares para cada caso. Neste contexto, esta dissertação modela, de uma forma mais simplificada, o reconhecimento de padrões para a predição dos dados extraídos da rede social \textit{Twitter}.












\chapter{FUNDAMENTAÇÃO TEÓRICA}\label{ch:fundaments-teorico}
A mineração de dados é um assunto totalmente interdisciplinar podendo ser definido de diversas maneiras. Até mesmo o termo \textit{data mining} não representa realmente todos os componentes desta área. \citeonline{han} exemplificam esta questão comentando sobre a mineração de ouro através da extração de rocha e areia, que é chamado de mineração de ouro e não mineração de rochas ou mineração de areia. Analogamente, a mineração de dados deveria se chamar "mineração de conhecimento através de dados", que infelizmente é um termo um tanto longo. Entretanto, uma referência mais curta, como "mineração de conhecimento", pode não enfatizar a mineração de uma enorme quantidade de dados. Apesar disso, a mineração é um termo que caracteriza o processo de encontrar uma pequena quantia de uma preciosa pepita em uma grande quantidade de matéria bruta. Nesse sentido, um termo impróprio contendo ambos \textit{"data"} e \textit{"mining"} se tornou popular e, como consequência, muitos outros nomes similares surgiram: \textit{knowledge mining from data}, \textit{knowledge extraction}, \textit{data/pattern analysis}, \textit{data archaeology} e \textit{data dredging}.

Na Seção~\ref{sec:kdd} será abordado os conceitos de descoberta de conhecimento em base de dados e a sua diferença em relação ao \textit{data mining}. Logo, na Seção~\ref{sec:data-mining}, é apresentado os métodos e a concepção de \textit{data mining}. Alguns desses métodos tem como prática o uso de aprendizado de máquina que será definido na Seção~\ref{sec:machine-learning}. A Seção~\ref{sec:python} irá caracterizar a linguagem de programação Python e qual a sua vantagem em utilizá-la para a mineração de dados.

%%
% KDD
\section{DESCOBERTA DE CONHECIMENTO EM BASE DE DADOS E \textit{DATA MINING}}\label{sec:kdd}
Muitas pessoas tratam a mineração de dados como um sinônimo para outro termo muito popular, descoberta de conhecimento em base de dados (\textit{knowledge discovery from data}) - KDD, enquanto outros referenciam \textit{data mining} como apenas uma etapa no processo de descoberta de conhecimento em base de dados. O processo de KDD é demonstrado através da  Figura~\ref{kdd-fig} como uma sequência interativa e iterativa dos seguintes passos: \\ \\ \\ \\ \\ \\

\begin{figure}
	\centering
	\includegraphics[width=1\textwidth]{Cap3/imagens/kdd}
	\caption{\textsc{Etapas do processo de KDD}}
	\vspace{-0.3cm}
	\legend{\small{FONTE: Adaptado de \citeonline{han}}}
	\label{kdd-fig}
\end{figure}

\begin{enumerate}
	\item \textit{Data cleaning} (Limpeza de dados);
	\item \textit{Data integration} (Integração de dados);
	\item \textit{Data selection} (Seleção de dados);
	\item \textit{Data transformation} (Transformação de dados);
	\item \textit{Data mining} (Mineração de dados);
	\item \textit{Pattern evaluation} (Avaliação de padrões);
	\item \textit{Knowledge presentation} (Apresentação de conhecimento).
\end{enumerate}


De acordo com \apudonline{brachman}{fayyad2}, as etapas são interativas porque envolvem a cooperação da pessoa responsável pela análise de dados, cujo conhecimento sobre o domínio orientará a execução do processo. Por sua vez, a interação deve-se ao fato de que, com frequência, esse processo não é executado de forma sequencial, mas envolve repetidas seleções de parâmetros e conjuntos de dados, aplicações das técnicas de \textit{data mining} e posterior análise dos resultados obtidos, a fim de refinar os conhecimentos extraídos.

KDD refere-se ao processo global de descobrimento de conhecimento útil em bases de dados. \textit{Data mining} é um passo particular neste processo aplicação de algoritmos específicos para extrair padrões (modelos) de dados. Os passos adicionais no processo KDD, como integração de dados, limpeza de dados, seleção de dados, incorporação de conhecimento anterior apropriado e interpretação formal dos resultados de mineração assegura aquele conhecimento útil que é derivado dos dados. A aplicação cega de métodos de \textit{data mining} pode ser uma atividade perigosa que conduz a descoberta de padrões sem sentido \cite{navega}. 

O KDD evoluiu e continua evoluindo da interseção de pesquisas em campos como bancos de dados, aprendizado de máquinas (\textit{machine learning}), reconhecimento de padrões, estatísticas, inteligência artificial, aquisição de conhecimento para sistemas especialistas, visualização de dados, descoberta científica, recuperação de informação e computação de alto-desempenho. Aplicações de KDD incorporam teorias, algoritmos e métodos de todos estes campos \cite{credito-bancario}.


%%
% Data Mining
\section{\textit{DATA MINING}}\label{sec:data-mining}
Apesar do conceito de \textit{data mining}, na maioria das vezes, ser utilizado pelas indústrias, mídias e centros de pesquisa para se referir ao processo de descoberta de conhecimento considerado em sua globalidade, o termo \textit{data mining} poderá ser usado também para indicar o quinto estágio do KDD, sendo um processo essencial na descoberta e extração de padrões de dados. \citeonline{han}, adotam uma visão mais abrangente para a funcionalidade de mineração de dados: \textit{data mining} é o processo de descoberta de padrões interessantes e conhecimentos de um vasto conjunto de dados. A fonte dos dados pode ser banco de dados, \textit{data warehouses}, a Internet, outros repositórios de informações, ou dados correntes em sistemas dinâmicos.

Uma das definições, talvez, mais importante de \textit{data mining} foi elaborada por \citeonline{fayyad} "...o processo não-trivial de identificar, em dados, padrões válidos, novos, potencialmente úteis e ultimamente compreensíveis".

\textit{Data mining} ou mineração de dados, pode ser entendido, então como o processo de extração de informações, sem conhecimento prévio, de algum conjunto de dados e seu uso para tomada de decisões. A mineração de dados se define através de processos automatizados de captura e análise deste conjunto de dados com a finalidade de extrair algum significado, podendo descrever características do passado, como também para predizer futuras tendências \cite{conceito-data-mining}.

Diversos métodos são usados em \textit{data mining} para encontrar respostas ou extrair conhecimento interessante. Esses podem ser obtidos através dos seguintes métodos:

\begin{itemize}
	\item Classificação: associa ou classifica um item a uma ou várias classes. Os objetivos dessa técnica envolvem a descrição gráfica ou algébrica das características diferenciais das observações de várias populações. A ideia principal é derivar uma regra que possa ser usada para classificar, de forma otimizada, uma nova observação a uma classe já rotulada;
	
	\item Modelos de Relacionamento entre Variáveis: associa um item a uma ou mais variáveis de predição de valores reais, conhecidas como variáveis independentes ou exploratórias. Nesta etapa se destacam algumas técnicas estatísticas como regressão linear simples, múltipla e modelos lineares por transformações, com o objetivo de verificar o relacionamento funcional entre duas variáveis quantitativas, ou seja, constatar se há uma relação funcional entre X e Y;
	
	\item Análise de Agrupamento (\textit{Cluster}): associa um item a uma ou várias classes (ou \textit{clusters}). Os \textit{clusters} são definidos por meio do agrupamento de dados baseados em modelos probabilísticos ou medidas de similaridade. Analisar \textit{clusters} é uma técnica com o objetivo de detectar a existência de diferentes grupos dentro de um determinado conjunto de dados e, caso exista, determinar quais são eles;
	
	\item Sumarização: determina uma descrição compacta para um determinado subconjunto, por exemplos, medidas de posição e variabilidade. Nesta etapa se aplica algumas funções mais sofisticadas envolvendo técnicas de visualização e a determinação de relações funcionais entre variáveis. Estas funções são usadas para a geração automatizada de relatórios, sendo responsáveis pela descrição compacta de um conjunto de dados;
	
	\item Modelo de Dependência: descreve dependências significativas entre variáveis. Estes modelos existem em dois níveis: estruturado e quantitativo. O nível estruturado demonstra, através de gráficos, quais variáveis são localmente dependentes. O nível quantitativo especifica o grau de dependência, utilizando alguma escala numérica;
	
	\item Regras de Associação: determinam relações entre campos de um banco de dados. Esta relação é a derivação de correlações multivariadas que permitam auxiliar as tomadas de decisão. Medidas estatísticas, como correlação e testes de hipóteses apropriados, revelam a frequência de uma regra no universo dos dados minerados;
	
	\item Análise de Séries Temporais: determina características sequênciais, como dados com dependência no tempo. Tem como objetivo modelar o estado do processo extraindo e registrando desvios e tendências no tempo. As séries são compostas por quatro padrões: tendência, variações cíclicas, variações sazonais e variações irregulares. Existem vários modelos estatísticos que podem ser aplicados a essas situações.
\end{itemize}

A maioria destes métodos são baseados em técnicas de aprendizado de máquina (\textit{machine learning}), reconhecimento de padrões e estatística. Essas técnicas vão desde estatística multivariada, como análise de agrupamentos e regressões, até modelos mais atuais de aprendizagem, como redes neurais, lógica difusa e algoritmos genéticos \cite{conceito-data-mining}.

Devido aos vários métodos estatísticos que são aplicados no processo de \textit{data mining}, \cite{fayyad} mostram uma relevância da estatística para o processo de extração de conhecimentos ao afirmar que essa ciência provê uma linguagem e uma estrutura para quantificar a incerteza resultante quando se tenta deduzir padrões de uma amostra a partir de uma população.

%%
% Machine Learning
\section{\textit{MACHINE LEARNING}}\label{sec:machine-learning}
Abstratamente, pode-se pensar em \textit{machine learning}, ou aprendizado de máquina, como um conjunto de ferramentas e métodos que tentam inferir padrões e extrair \textit{insights} de um porção daquilo que se é observado no mundo. Por exemplo, ao tentar ensinar um computador a reconhecer os códigos postais escritos nos envelopes, os dados podem consistir em fotografias dos envelopes, além de um registro do código postal a que cada envelope estava endereçado. Ou seja, dentro de um contexto, é possível selecionar um registro de ações de certos objetos, aprender com este registro e, em seguida, criar um modelo dessas atividades que irão informar a compreensão deste contexto futuramente \cite{machine-hacker}.

Na prática, isto requer dados e, em aplicações atuais, isso, muitas vezes, significa uma grande quantidade de dados (talvez vários \textit{terabytes}). A maioria das técnicas de aprendizagem automática considera a disponibilidade de tais dados como algo inquestionável, o que significa novas oportunidades para a sua aplicação, em função da quantidade de dados que são produzidos como um produto de administrar companhias modernas.

\textit{Machine learning} é a intersecção entre ciência da computação, engenharia, estatística e ainda outras disciplinas. É possível ser aplicada em várias áreas desde políticas a geociência. É uma ferramenta que pode ser utilizada para a solução de vários problemas. Qualquer campo que precisa interpretar e agir sobre dados pode ser beneficiar do uso de técnicas de \textit{machine learning} \cite{machine-hacker}.

A prática de engenharia está em utilizar a ciência para resolver um problema. Em engenharia, é comum resolver um problema determinista, em que a solução dada por humanos sempre resolve o problema. Se desenvolver um software para controlar uma máquina de venda automática, é melhor que esta trabalhe sempre, independentemente do dinheiro depositado ou dos botões pressionados. Muitos problemas existem quando a solução não é determinista. Isto é, ou não se sabe o suficiente sobre o problema ou não se tem poder computacional suficiente para delinear adequadamente o problema. Para esses problemas, precisa-se de estatísticas. 

Uma das tarefas de \textit{machine learning} é a classificação. Na classificação, o trabalho é prever em que classe deve uma porção de dados ser enquadrada. Outra tarefa é a regressão. A regressão é a previsão de um valor numérico. Classificação e regressão são exemplos de aprendizado supervisionado. Este conjunto de problemas é conhecido como supervisionado, porque está dizendo o que o algoritmo deve prever \cite{machine-learning}.

O oposto de aprendizagem supervisionada é um conjunto de tarefas conhecidas como aprendizado não supervisionado. Em aprendizado não supervisionado, não há nenhum rótulo ou valor alvo dado para os dados. A tarefa através da qual se agrupa itens semelhantes é conhecida como \textit{clustering}. Na aprendizagem não supervisionada, também pode-se querer encontrar valores estatísticos que descrevem os dados. Isso é conhecido como estimativa da densidade. Outra tarefa do aprendizado não supervisionado está em reduzir dados com várias funcionalidades até se chegar a um número reduzido, em que seja possível visualizá-lo em duas ou três dimensões \cite{machine-learning}.

%%
% Python
\section{PYTHON}\label{sec:python}
Python é uma linguagem de programação orientada a objetos, interpretada, interativa. Incorpora módulos, excessões, tipagem dinâmica alta. Possuindo uma sintaxe clara e simples, o que facilita o aprendizado para novos desenvolvedores, assim como a rápida leitura e interpretação para usuários mais experientes. A linguagem dispõe de interfaces para várias chamadas de sistemas (\textit{system calls}) e bibliotecas, também para vários sistemas de janelas, e é extensível a outras linguagens de programação como C ou C++. É também usada como uma linguagem de extensão para aplicações que precisam de uma interface programática \cite{python-doc}.

Outra característica de Python é a portabilidade, podendo ser utilizada em diversos sistemas operacionais como variantes do Unix, em sistemas Mac e também em PCs sob MS-DOS, Windows, Windows-NT, e OS/2.

Python é uma linguagem de programação de alto-nível que pode ser aplicada em soluções para diversas classes diferentes. A linguagem possui uma vasta quantidade de bibliotecas que atende a áreas como o processamento de \textit{strings} (expressões regulares, Unicode, cálculo de diferença entre arquivos), protocolos de Internet (HTTP, FTP, SMTP, XML-RPC, POP, IMAP, CGI \textit{programming}), engenharia de software (testes unitários, registro de logs, \textit{profiling}, análise de código Python), e interfaces para sistemas operacionais (\textit{system calls}, sistemas de arquivos, TCP/IP sockets) \cite{python-doc}.

A sintaxe bastante expressiva e a abundância de suas bibliotecas tornam Python uma ótima linguagem para se obter resultados em várias questões. Algumas de suas utilidades é apresentada conforme a seguinte lista:

\begin{itemize}
	\item Escrita de \textit{scripts}: Python é uma ótima linguagem para a criação de \textit{scripts}. É possível usar \textit{scripts} para analisar arquivos de texto, gerar amostra de entradas para testar programas, coletar conteúdos de páginas \textit{web} utilizando a biblioteca \textit{Beautiful Soup}, dentre outras atividades;
	\item Desenvolvimento \textit{backend} para aplicações \textit{web}: É possível criar APIs (\textit{Application Programming Interface}, apresentado no Capítulo~\ref{ch:materiais-metodos}) e interagir com banco de dados. \textit{Frameworks} mais utilizados inclui \textit{Django}, \textit{Flask} e \textit{Pyramid};
	\item Análise e visualização de dados: Conforme o foco deste trabalho, bibliotecas como \textit{pandas}, \textit{NumPy} e recursos semelhantes a outras ferramentas como R e MATLAB estão dispostas através da biblioteca \textit{SciPy};
	\item \textit{matplotlib} e \textit{Seaborn} são mecanismos que possibilitam a visualização dos dados.
\end{itemize}

\textit{Dictionaries} em Python são estruturas de dados similares ao JSON, que permitem ordenar dados através de um modelo chave-valor. Devido então a sintaxe intuitiva que a linguagem possui e seu excelente ecossistema de bibliotecas, é possível acessar APIs e manipular dados com mais facilidade \cite{mining-social-web}. 













\chapter{MATERIAIS E MÉTODOS}\label{ch:materiais-metodos}
Após a revisão bibliográfica de outros estudos e os fundamentos teóricos necessários para a mineração de dados utilizando Python, torna-se importante definir as ferramentas, tecnologias e procedimentos necessários para o desenvolvimento do projeto.

Este capítulo apresenta os materiais e métodos utilizados para a realização do processo de \textit{data mining}, onde, na Seção~\ref{sec: tec-ferramenta} é apresentado as ferramentas e tecnologias que serão utilizadas durante o estudo. Serão abordados quais as bibliotecas que o Python disponibiliza para a análise e mineração de dados e também como acessar a API do \textit{LinkedIn}, e essa será esclarecida também neste capítulo após a explicação do conceito de API e o protocolo OAuth.

A Seção~\ref{sec: metodologia} irá concluir o capítulo apresentando as etapas de \textit{data mining} com o intuito de evidenciar o processo para a obtenção de conhecimento útil.

\section{TECNOLOGIAS E FERRAMENTAS}\label{sec: tec-ferramenta}
Tecnologias e ferramentas para a criação e prototipagem dos algoritmos.

%% Bibliotecas Python
\subsection{Bibliotecas do Python}\label{sec:bib_python}
Um dos grandes diferenciais da linguagem Python é o seu enorme conjunto de bibliotecas para soluções de diversos problemas.

A seguir serão apresentadas as bibliotecas necessárias para a mineração de dados, através das quais é possível coletar, limpar, transformar, realizar operações e apresentar resultados proveniente dos dados da rede social \textit{LinkedIn}.
 
%%% sub: NumPy
\subsubsection{\textbf{\textit{NumPy}}}
\textit{NumPy} é o pacote fundamental para computação científica. É o acrônico para \textit{Numerical Python}. Esta biblioteca provê:

\begin{itemize}
    \item \textit{ndarray} que é um objeto de matriz multidimensional;
    \item Funções que permitem realizar operações vetoriais ou operações matemáticas entre matrizes sem a necessidade de programar \textit{loops};
    \item Ferramentas para a leitura e escrita em conjuntos de dados matriciais;
    \item Operações de álgebra linear, transformada de Fourier e geração de números aleatórios;
    \item Ferramentas para a integração em outras linguagens de programação como C, C++ e Fortran.
\end{itemize}

Além da capacidade de rápido processamento em matrizes que o \textit{NumPy} oferece ao Python, um dos principais objetivos em relação a análise de dados é que serve como um "container" para os dados serem passado por algoritmos. Para dados numéricos, as matrizes de \textit{NumPy} são muito mais eficientes para a ordenação e manipulação de dados do que qualquer outra estrutura embutida em Python. Igualmente, bibliotecas escritas em linguagens de baixo nível, como C ou Fortran, podem operar dados gravados em matrizes da \textit{NumPy} sem precisar da cópia de qualquer dado \cite{python-analysis}.

A biblioteca \textit{NumPy} por si só, não provê uma funcionalidade de alto-nível para a análise de dados. Tendo um conhecimento sobre as matrizes de \textit{NumPy} e matrizes orientadas a computação (\textit{array-oriented computing}) irá facilitar o uso de outras ferramentas, como \textit{pandas}, com mais efetividade.

Para aplicações voltadas para a análise de dados, esta biblioteca possui grande funcionalidade em setores como:

\begin{itemize}
    \item Criação rápida de matrizes para a interação e limpeza de dados, separação e filtragem, transformação e outros tipos de operações computacionais;
    \item Algoritmos comuns para matrizes como ordenação, operações únicas e definidas;
    \item Eficiente descrição estatística e agregação/sumarização de dados;
    \item Alinhamento de dados e manipulação de dados relacionais para operações de junção e imerção (\textit{join} e \textit{merge}) de conjuntos de dados heterogeneos;
    \item Expressar lógicas de condições através de expressões matriciais ao invés de laços de repetições e condições como \textit{while, for, if-elif-else};
    \item Agrupamento de manipulação de dados (agregação, transformação, aplicação de funções).
\end{itemize}

Enquanto \textit{NumPy} oferece o fundamento computacional para essas operações, é preferível utilizar a biblioteca \textit{pandas} como base para a mineração de dados (especialmente de dados estruturados ou dados tabulados) devido a sua interface rica e de alto-nível no qual permite as tarefas com dados mais concisas e simples.

%%% sub: pandas
\subsubsection{\textbf{\textit{pandas}}}
\textit{pandas} é a biblioteca de maior interesse para a mineração de dados. Ela possui estruturas de dados de alto-nível e ferramentas de manipulação desenvolvidas para facilitar e agilizar a análise de dados em Python. \textit{pandas} é desenvolvida sob a biblioteca \textit{NumPy} e viabiliza o uso em aplicações centradas nesta. A seguir são expostas algumas soluções que a biblioteca disponibiliza \cite{python-analysis}:

\begin{itemize}
    \item Estrutura de dados com eixos rotulados suportam o alinhamento de dados automáticos ou explícitos. Isso evita erros comuns resultantes de dados desalinhados e dados indexados de formas diferentes provenientes de outras fontes de dados;
    \item A mesma estrutura de dados consegue manusear tanto dados de séries temporais como dados não-temporais;
    \item Operações e reduções aritméticas é passado para metadados (eixos rotulados);
    \item Manipulação flexível de dados em falta;
    \item \textit{Merge} (fundir) e outras operações relacionais encontradas em bancos de dados relacional.
\end{itemize}

Esta biblioteca possui duas estrutura de dados principais: \textit{Series} e \textit{DataFrame}. Estas estruturas não são uma solução universal para todos os problemas, mas provê uma base sólida e de fácil manipulação para a maioria das aplicações com mineração de dados.

Uma \textit{Serie} é um tipo de \textit{array} ou uma matriz unidimensional, similar a um \textit{array} que possui uma matriz de dados (qualquer tipo de dado da biblioteca \textit{NumPy}) e um outro vetor associado a dados rotulados, chamados de \textit{index} (índice). Uma simples \textit{Series} é formado por uma única matriz de dados conforme a Figura~\ref{pandas-series}.

\begin{figure}[h!]
	\centering
	\fbox{\includegraphics[width=.5\textwidth]{Cap4/imagens/pandas-series}}
	\caption{\textsc{Exemplo de uma \textit{Series}}}
	\vspace{-0.3cm}
	\legend{\small{FONTE: \citeonline{python-analysis}}}
	\label{pandas-series}
\end{figure}

\textit{DataFrame} representa uma tabela, uma estrutura de dados do tipo planilha, que possui uma coleção ordenada de colunas, onde cada uma delas pode ter um tipo de valor diferente (numérico, \textit{string}, \textit{boolean}, etc.). O \textit{DataFrame} possui um índice para linhas e também para colunas. Pode ser interpretado como um dicionário de \textit{Series}. De uma maneira geral, o dado é armazenado como um ou mais blocos bi-dimensionais ao invés de uma lista, dicionário, ou outro tipo de coleção de matriz unidimensional \cite{python-analysis}.

Existem várias maneiras diferentes de se criar um \textit{DataFrame}, entretanto uma forma comum é um dicionário de dimensões iguais, conforme a Figura~\ref{pandas-dataframe} e Figura~\ref{pandas-dataframe2},  ou uma matriz \textit{NumPy}.

\begin{figure}[h!]
	\centering
	\fbox{\includegraphics[width=.9\textwidth]{Cap4/imagens/pandas-dataframe}}
	\caption{\textsc{Criação de um \textit{DataFrame}}}
	\vspace{-0.3cm}
	\legend{\small{FONTE: \citeonline{python-analysis}}}
	\label{pandas-dataframe}
\end{figure}

\begin{figure}[h!]
	\centering
    \fbox{\includegraphics[width=.36\textwidth]{Cap4/imagens/pandas-dataframe2}}
	\caption{\textsc{Conteúdo de um \textit{DataFrame} pelo interpretador \textit{IPython}}}
	\vspace{-0.3cm}
	\legend{\small{FONTE: \citeonline{python-analysis}}}
	\label{pandas-dataframe2}
\end{figure}

%%% sub: matplotlib
\subsubsection{\textbf{\textit{matplotlib}}}
\textit{matplotlib} é uma biblioteca desenvolvida para a geração de gráficos bidimensionais a partir de \textit{arrays}. Gráficos comuns podem ser criados com alta qualidade a partir de simples comandos, inspirados nos comandos gráficos do MATLAB, exemplo ilustrado na Figura~\ref{matplotlib-fig}.

Quando usado em conjunto com ferramentas GUI (\textit{IPython}, por exemplo), esta biblioteca possui recursos interativos como zoom e visão panorâmica. \textit{matplotlib} suporta várias ferramentas GUI \textit{backend}, nos diversos sistemas operacionais suportados pelo Python, e permitem exportar gráficos em diversos formatos: PDF, SVG, JPG, PNG, BMP, GIF, etc.

\textit{matplotlib} também possui várias ferramentas adicionais, como o \textit{mplot3d} para plotar gráficos em tridimensionais e \textit{basemap} para mapeamentos e projeções.

\begin{figure}[h!]
  \includegraphics[width=1\textwidth]{Cap4/imagens/matplotlib}
  \caption{\textsc{Exemplo de um gráfico gerado pelo \textit{matplotlib}}}
  \vspace{-0.3cm}
  \legend{\small{FONTE: \citeonline{matplotlib}}}
  \label{matplotlib-fig}
\end{figure}


%%% sub: SciPy
\subsubsection{\textbf{\textit{SciPy}}}
\textit{SciPy} é uma coleção de pacotes que abordam uma série de soluções para diferentes domínios na computação científica. Na lista a seguir são apresentados exemplos desses pacotes \cite{python-analysis}:

\begin{itemize}
	\item \textit{scipy.integrate}: rotinas de integração numéricas e soluções de equações diferenciais;
	\item \textit{scipy.linalg}: rotinas de álgebra linear e decomposição de matrizes;
	\item \textit{scipy.optimize}: funções otimizadoras (minimizadoras) e algoritmos de busca em raíz;
	\item \textit{scipy.signal}: ferramentas para processamento de sinais;
	\item \textit{scipy.sparse}: matrizes esparsas e soluções de sistemas lineares esparsos;
	\item \textit{scipy.special}: agregador do \textit{SPECFUN}, uma biblioteca do Fortran que implementa várias funções matemáticas, como exemplo, a função gama;
	\item \textit{scipy.stats}: funções estatísticas, variáveis contínuas e discretas, testes estatísticos e outros modelos estatísticos;
	\item \textit{scipy.weave}: ferramenta para usar códigos \textit{inline} de C++ para acelerar a computação de matrizes.
\end{itemize}


%%% sub: IPython
\subsubsection{\textbf{\textit{IPython}}}
\textit{IPython} foi desenvolvido com o intuito de ser um interpretador interativo para o Python que teve início em 2001. Desde a sua criação o \textit{IPython} evoluiu grandemente, ao ponto de ser considerada uma das mais importantes ferramentas para computação científica em Python. Essa biblioteca não oferece nenhuma ferramenta para análise de dados ou análise computacional em si, sendo designada para maximizar a produtivadade tanto na interação computacional como no desenvolvimento de softwares. Oferece um fluxo de visualização de um modo \textit{execute-explore} ao invés do típico modelo \textit{edit-compile-run} de muitas outras linguagens de programação. Ela também provê uma pequena integração com o \textit{shell} e o sistema de arquivos de sistema operacional. Como a maior parte da programação focada na mineração de dados envolve exploração, tentativa e erro, e iteração, \textit{IPython}, em quase todos os casos, irá facilitar este tipo de trabalho \cite{python-analysis}.

Hoje, o projeto \textit{IPython} engloba muito mais do que apenas um interpretador \textit{shell} para Python. Ele também inclui um console gráfico interativo, o \textit{IPython Notebook}, que provê ao usuário uma experiência de caderno (\textit{notebook-like}) através de um navegador \textit{web}, conforme Figura~\ref{ipython-fig}, e dispõe de um mecanismo de processamento paralelo. Assim como muitas outras ferramentas desenvolvidas para programadores, é extremamente customizável \cite{mining-social-web}. \\ \\ \\ \\ \\ \\ \\

\begin{figure}[h!]
  \centering
  \fbox{\includegraphics[width=0.9\textwidth]{Cap4/imagens/ipython-notebook}}
  \caption{\textsc{Exemplo de uma página \textit{web} do \textit{IPython Notebook}}}
  \vspace{-0.3cm}
  \legend{\small{FONTE: \citeonline{python-analysis}}}
  \label{ipython-fig}
\end{figure}

% python-linkedin
\subsubsection{\textbf{\textit{python-linkedin}}}
Esta biblioteca provê ao Python a API do \textit{LinkedIn}. Através da utilização do protocolo OAuth 2.0, é possível acessar diveros campos como \textit{Profile}, \textit{Group}, \textit{Company}, \textit{Jobs}, \textit{Search}, \textit{Share}, \textit{Network} e requisições REST APIs \cite{python-linkedin}.

%%
% API
\subsection{Interface de Programação de Aplicações - API}\label{subsec: api}
API é uma sigla para \textit{Application Programming Interface} e basicamente é uma tecnologia que permite um pedaço de \textit{software} se comunicar com outro pedaço de \textit{software}. Existem vários tipos de API e é comumente referenciado a outras tecnologias. Por exemplo para o desenvolvimento deste trabalho será utilizado a API do \textit{LinkedIn}. 

Uma API é composta por uma série de funções acessíveis somente por programação, e que permitem utilizar características do \textit{software} menos evidentes ao utilizador tradicional.


%% 
% REST API
\subsubsection{\textbf{Arquitetura REST}}
REST foi um termo criado por \citeonline{rest}, onde ele modela um estilo de arquitetura para a construção de serviços \textit{web} consistentes e coesos. O estilo da arquitetura REST é baseado em recursos e nos estados desses recursos.

Abreviação para Transferência de Estado Representacional, REST, é um estilo arquitetural baseado em recursos e nas representações desses recursos. Enfatiza a escalabilidade na interação entre componentes, a generalidade de interfaces, a implantação independente dos componentes de um sistema, o uso de componentes intermediários visando a redução na latência de interações, o reforço na segurança e o encapsulamento de sistemas legados. O REST ignora os detalhes da implementação de componente e a sintaxe de protocolo com o objetivo de focar nos papéis dos componentes, nas restrições sobre sua interação com outros componentes e na sua interpretação de elementos de dados significantes \cite{rest}.

A funcionalidade de uma REST API é similar ao funcionamento de uma página \textit{web}, onde o usuário efetua uma requisição a um servidor \textit{web}, utilizando o protocolo HTTP, e recebe dados como resposta.

Um recurso é qualquer conteúdo ou informação que é exposto na Internet, podendo ser um documento, vídeo clip, até processos de negócio ou dispositivos. Para utilizar um recurso é necessário ser capaz de identificá-lo na rede e de ter meios para manipulá-lo. Tem-se então o \textit{Uniform Resource Identifiers} (URI) para este propósito. Um URI unicamente identifica um recurso e, ao mesmo tempo, torna ele endereçável ou capaz de ser manipulado utilizando um protocolo, como o HTTP. O URI de um recurso se distingue dos de qualquer outro recurso e é através do próprio URI que ocorrem as interações com o recurso \cite{rest-book}.

Recursos devem possuir pelo menos um identificador para ser endereçável, e cada identificador é associado com uma ou mais representações. Uma representação é uma transformação ou uma visão do estado do recurso em um instante de tempo. Essa visão é codificada em um ou mais formatos transferíveis, tal como XHTML, Atom, texto simples, XML, YML, JSON, JPG, MP3, entre outros  \cite{rest-book}.

Os recursos provêm o conteúdo ou objeto com o qual se quer interagir e para atuar sobre eles é utilizado os métodos de HTTP. Os métodos HTTP na arquitetura REST podem ser referenciados como Verbos, uma vez que representam ações sobre os recursos \cite{rest-book}.


\subsection{Protocolo de Autenticação - OAuth}
Protocolos de autenticação são capazes de simplesmente autenticar a parte que está se conectando, ou ainda de autenticar a parte que está conectando assim como se autenticar para ele.

Neste trabalho será utilizado apenas o protocolo OAuth 2.0 para o acesso aos dados do \textit{LinkedIn}. É possível também realizar a autenticação utilizando a versão mais antiga, OAuth 1.0a, mas será apenas referenciado, neste trabalho, para a melhor compreensão do funcionamento do protocolo.

OAuth é uma sigla para "\textit{open authorization}", ou autorização aberta, e provê um meio para que usuários autorizem uma aplicação acessar dados, com alguma finalidade, através de uma API sem que os usuários precisem passar credenciais como nome de usuário e senha. De um modo geral, usuários são capazes de controlar o nível de acesso para estas aplicações e revogar este controle a qualquer momento \cite{mining-social-web}.

\subsubsection{\textbf{OAuth 1.0a}}
OAuth 1.0a é um protocolo que permite que um cliente (\textit{client}) \textit{web} tenha acesso a um recurso protegido pelo seu dono em um servidor. Esta definição se dá através da RFC 5849. Que são documentos técnicos desenvolvidos e mantidos pelo Internet Enginnering Task Force (IETF), instituição que especifica os padrões que serão implementados e utilizados em toda a Internet.

A razão para a existência dessa tecnologia é para evitar problemas de usuários (donos dos recursos) compartilhar suas senhas com aplicações \textit{web}.

A versão OAuth 1.0a não permite que credenciais sejam trocadas utilizando uma conexão \textit{Secure Socket Layer} (SSL) através de um protocolo HTTPS, por esse motivo muitos desenvolvedores achavam tedioso o trabalho devido aos vários detalhes envolvidos em encriptação.

SSL é um padrão global para tecnologia de segurança. Tem como função principal criar um canal criptografado entre um servidor \textit{web} e um navegador (\textit{browser}) para garantir que todos os dados transmitidos sejam seguros e sigilosos.

Uma aplicação que está requerindo acesso é conhecida como \textit{client}, em alguns momentos chamado de \textit{consumer}, a rede social ou o serviço que contém os recursos protegidas é nomeado como \textit{server} (também chamado de \textit{provider}) e o usuário que concede o acesso é o \textit{resource owner} (dono do recurso, tradução livre). Com estes elementos, três participações envolvem o processo e a interação que estes elementos possuem é conhecida como \textit{"three-legged-flow"} ou de uma maneira mais coloquial, a \textit{OAuth dance}. Estas são as etapas fundamentais que envolvem a \textit{OAuth dance} que, como resultado, permite ao \textit{client} o acesso a recursos protegidos, conforme listado a seguir \cite{mining-social-web}:

\begin{enumerate}
	\item O \textit{client} obtêm um \textit{token} de requisição do servidor de serviço (aplicação);
	\item O dono do recurso autoriza o \textit{token} de requisição;
	\item O \textit{client} troca o \textit{token} de requisição por um \textit{token} de acesso;
	\item O \textit{client} usa o \textit{token} de acesso para acessar os recursos protegidos com a consideração do dono do recurso.
\end{enumerate}

Para credenciais particulares, um \textit{client} começa com um uma \textit{consumer key} e um \textit{consumer secret} e no fim do processo de \textit{OAuth dance} termina com um \textit{token} de acesso e \textit{token} de acesso secreto que pode ser usado para acessar recursos protegidos.

\subsubsection{\textbf{OAuth 2.0}}
Enquanto o protocolo OAuth 1.0a permite uma autorização útil para o acesso a aplicações \textit{web}, OAuth 2.0 foi originalmente destinado a simplificar significantemente a implementação detalhada para desenvolvedores de aplicações \textit{web}, baseando-se completamente no SSL para aspectos de segurança e satisfazer uma vasta quantidade de casos de uso. Esses casos de uso variaram desde suporte para dispositivos móveis à necessidades empresariais e, consequentemente as necessidades de um termo mais futuro, da "Internet das Coisas"\space \cite{mining-social-web}.

Diferentemente da implementação OAuth 1.0a, que consiste de um rígido conjunto de etapas, a implementação do OAuth 2.0, definido através do RFC 6749, pode variar de acordo com a particularidade do caso de uso. Um decorrer típico da execução do OAuth 2.0 tem a vantagem do SSL e essencialmente contém apenas poucos redirecionamentos que, acompanhada de em alto-nível, não possui tanta diferença em relação ao processo anterior que envolvem um ciclo do OAuth 1.0a.

%%
% Linkedin
\subsection{LinkedIn}
\textit{LinkedIn} é uma rede social diferenciada, pois é voltada para perfis profissionais. Os usuários geralmente utilizam a rede como um currículo \textit{online}, procuram se relacionar com novos contatos para aumentar a sua rede de conhecimentos profissionais e também para encontrar e conhecer novas oportunidades tanto de ocupação quanto de negócio.

A mineração de dados através da rede \textit{LinkedIn} se dá utilizando a API que ela dispõe. Para isso é necessário possuir uma conta na rede social.

A grande parte das análise realizadas neste projeto acontece utilizando arquivos \textit{comma-separated values} (CSV), que é possível baixá-lo.

Na Subseção~\ref{api-linkedin} é apresentado a API do \textit{LinkedIn} e como se obter dados através dela.

\subsubsection{\textbf{API LinkedIn}}\label{api-linkedin}
O acesso a API se dá através da criação de uma aplicação pela página \textit{web} de desenvolvimento do \textit{LinkedIn}. Após a criação da aplicação é fornecido ao usuário informações para o acesso utilizando o protocolo OAuth, será informado uma chave da API da aplicação, uma chave secreta, o \textit{token} de usuário OAuth e credenciais secretas do usuário OAuth.

Dispondo de todas as credenciais do protocolo OAuth o acesso a API acontece utilizando a biblioteca \textit{python-linkedin}, especificada na Subseção~\ref{sec:bib_python}.

Os dados disponibilizados pela API está limitado basicamente ao o que o usuário consegue usufruir da experiência através da página \textit{web}. Restrições como a visualização de "amigos de amigos" \space também existem, demonstrando ser uma API um tanto restrita. Essa restrição é proposital e importante para a política do \textit{LinkedIn}. Portanto é possível fazer buscas de pessoas, companhias, empregos, conexões, grupos e outros elementos que a API disponibiliza \cite{mining-social-web}.

Conforme descrito anteriormente, é possível fazer o \textit{download} dos dados, visíveis ao usuário, em formato CSV. Podendo ser especificados em títulos de trabalho ou contatos através de uma opção de exportação presente no menu da página do \textit{LinkedIn}.

%%
% METODOLOGIA
\section{METODOLOGIA E DESENVOLVIMENTO}\label{sec: metodologia}
O processo de desenvolvimento da solução segue uma série de princípios de conjunto de boas práticas e etapas do \textit{data mining}, para melhor estruturar e obter, não só o resultado esperado, mas também para que todo o processo ocorra de forma coerente e padronizada.

%%
% Etapas do Data Mining
\subsection{Etapas Para a Mineração de Dados do \textit{LinkedIn}}
Abordado previamente, o acesso aos dados do \textit{LinkedIn} acontece através da utilização de sua API. A análise necessária para a interpretação dos dados se dá pela aplicação da técnica de \textit{clustering}, normalização de dados e computação de similaridade. 

\subsubsection{\textbf{\textit{Clustering}}} 
Um aprendizado não supervisionado que está presente em várias ferramentas de \textit{data mining}. A técnica de \textit{clustering} envolve em pegar uma coleção de itens e particioná-los em pequenas coleções, conhecidos como \textit{cluster}, de acordo com alguma heurística que será usado para comparar com outros itens da coleção.

Técnicas para \textit{clustering} são partes fundamentais para um processo de mineração de dados. Uma implementação simples de \textit{clustering} pode criar experiências de usuário incrivelmente convincentes para alcançar resultados. Também pode ser aplicado a bases de texto utilizando algoritmos de \textit{text mining}, onde o algoritmo procura agrupar textos que falem sobre o mesmo assunto e separar textos de conteúdo diferentes.

Por exemplo, caso queira considerar a localização geográfica dos contatos na rede \textit{LinkedIn}, é necessário realizar um \textit{cluster} nas conexões através de um determinado número de regiões com o intuito da melhor compreensão de oportunidades econômicas disponíveis.

\subsubsection{\textbf{Normalização de Dados}}
Quando se recupera dados provenientes de uma API, é muito comum que esses não estarão no formato desejado para a sua análise. No caso do \textit{LinkedIn}, pode ser que usuários procurem por vagas de emprego escrevendo os nomes das vagas de alguma outra maneira sem ser o nome correto da vaga. Como exemplo, uma vaga para administrador de banco de dados pode ser buscado através da sigla "DBA", que em inglês significa \textit{Database Administrator}.

A normalização de dados, procura então resolver esses problemas padronizando situações específicas que irá facilitar a análise posterior dos dados.

\subsubsection{\textbf{Computação de Similaridade}}
Após a normalização dos itens, é preciso verificar a similaridade entre eles. Podendo ser vagas de emprego, nomes de empresas, interesses profissionais, indicação geográfica, ou qualquer outro campo digitado na busca como um texto livre. Para isso é necessário definir uma heurística que conseguirá aproximar a similaridade entre dois valores quaisquer. Em algumas situações a similaridade heurística será um tanto óbvia, porém em outros casos será complicada. Por exemplo, comparar o tempo de carreira entre duas pessoas pode ser simples como uma operação de soma ou subtração. Mas comparar um elemento profissional, como "atitude de liderança" \space de uma maneira automatizada pode ser um desafio.

\section{CONSIDERAÇÕES FINAIS}
A utilização das bibliotecas que Python oferece para a mineração de dados permite que o desenvolvimento das soluções se tornem mais rápidos e efetivos. Assim como a utilização da API do \textit{LinkedIn} para a obtenção dos dados e as etapas do processo de \textit{data mining}. Boas práticas, protocolos e tecnologias são aproveitadas no andamento deste trabalho. Considerando os conceitos até aqui abordados, é possível compreender de forma clara os tópicos seguintes.






















\chapter{IMPLEMENTAÇÃO}\label{ch:implementacao}

\section{INTRODUÇÃO}
\section{ALGUMA COISA}
\section{ALGUMA COISA2}



















\chapter{CONCLUSÕES E SUGESTÕES PARA TRABALHOS FUTUROS}\label{ch:conclusao}

A proposta dessa dissertação foi de utilizar técnicas e algoritmos de \textit{data mining} para analisar e minerar os dados provenientes da rede social \textit{Twitter}, utilizando os recursos e bibliotecas da linguagem de programação Python. Para tanto, foi necessário o estudo de conceitos de KDD e \textit{data mining}, implementar técnicas utilizando bibliotecas da linguagem em estudo e apresentar informações úteis para possíveis interpretações.

Durante o estudo de \textit{data mining}, foi determinado que a sua definição consiste em um processo de descoberta de padrões interessantes e conhecimentos de um vasto conjunto de dados. Sendo necessário, então, ferramentas que auxiliem na coleta, limpeza, seleção e apresentação dos dados.

Para a coleta e limpeza dos dados a linguagem Python se tornou extremamente útil, ao permitir o desenvolvimento de um código que se autenticou com o \textit{Twitter} e a utilização da API de \textit{streaming} para coletar os dados que continham a \textit{hashtag} \#ImpeachmentDay.




%- Resgatar o objetivo
%
%- Comentar as ferramentas estudadas
%
%- Comentar as ferramentas utilizadas
%
%- Breve resumo dos resultados
%
%- Pontos positivos e negativos (O fato de não ter o perfil real)
%
%- Pensar em outra API (Aplicação para previsão de resultados) ou rede social, baseando-se numa pergunta específica

%\bookmarksetup{startatroot}

% ----------------------------------------------------------
% ELEMENTOS PÓS-TEXTUAIS
% ----------------------------------------------------------
\postextual

\bibliography{Configuracoes/citacoes} % Referências bibliográficas

%\begin{apendicesenv}% Apêndices: inserir se necessário
%\partapendices
%  % ----------------------------------------------------------
\chapter{TENSORES} \label{Tensor}
% ----------------------------------------------------------

%\lipsum[50]

Na Física e nas Engenharias as grandezas podem ser classificadas em três grandes grupos:

\begin{enumerate}
	
	\item Grandezas escalares: são aquelas que necessitam de somente uma informação para classificá-las. Como exemplos destas grandezas tem-se a temperatura, comprimento, massa, tempo, volume, energia, entre outros.
	\item Grandezas vetoriais: são as que necessitam de três informações para caracterizá-las (módulo, direção e sentido). Como exemplos destas grandezas pode-se destacar o deslocamento, a força, o fluxo, a corrente elétrica e a aceleração.
	\item Grandezas Tensoriais: são aquelas que para serem bem representadas necessitam de, pelo menos, nove informações. Destacam-se como exemplos o tensor de tensão, o tensor de deformação, o tensor inercial e as rotações. 
\end{enumerate}

A transformação linear que tem a capacidade de transformar um dado vetor $ \vec{a} $ em um outro vetor $ \vec{b} $, ou seja,
\begin{equation}
	\vec{a} = \textbf{T} \vec{b}
\end{equation}
é chamada de tensor de segunda ordem ou simplesmente de tensor, cuja notação é $ \textbf{T} $. Assim, $ \textbf{T} $ é uma transformação linear, pois
\begin{equation}
	\textbf{T} ( \alpha \vec{a} + \beta \vec{b}) = \alpha \textbf{T}( \vec{a}) + \beta \textbf{T} ( \vec{b}).
\end{equation}

Para um conjunto de vetores unitários $ e_{1}, e_{2}, e_{3} $, na direção das componentes de um sistema de coordenadas retangulares, suas transformações podem ser escritas na forma
\begin{equation}
	\textbf{T} e_{1} = \textbf{T} \left[ 
	\begin{array}{c}
		1\\
		0\\
		0\\
	\end{array}
	\right] = \textbf{T} (1 e_{1} + 0 e_{2} + 0 e_{3}) 
\end{equation} 
ou
\begin{equation} \label{TFe1}
	\textbf{T} e_{1} = T_{11} e_{1} + T_{21} e_{2} + T_{31} e_{3},
\end{equation}
\begin{equation} \label{TFe2}
	\textbf{T} e_{2} = T_{12} e_{1} + T_{22} e_{2} + T_{32} e_{3},
\end{equation}
\begin{equation}\label{TFe3}
	\textbf{T} e_{3} = T_{13} e_{1} + T_{23} e_{2} + T_{33} e_{3}.
\end{equation}

Utilizando a notação indicial esta transformação pode ser generalizada como
\begin{equation} \label{indicialT}
	\textbf{T} e_{i} = T_{ji} e_{j}.
\end{equation}

Considerando as transformações apresentadas em (\ref{TFe1}), (\ref{TFe2}) e (\ref{TFe3}), o tensor \textbf{T} pode ser representado matricialmente da seguinte forma
\begin{equation} \label{matrizT}
	\textbf{T} = [T] =  \left[ 
	\begin{array}{ccc}
		T_{11} & T_{12} & T_{13} \\
		T_{21} & T_{22} & T_{23} \\
		T_{31} & T_{32} & T_{33} \\
	\end{array}
	\right].  
\end{equation} 

Assim como na transformação linear, os tensores também apresentam algumas propriedades. São elas:
\begin{enumerate}
	\item[a)] A soma entre dois tensores \textbf{T} e \textbf{S}
	\begin{equation}
		( \textbf{T} + \textbf{S}) \vec{a} = \textbf{T} \vec{a} + \textbf{S} \vec{a}, 
	\end{equation}
	que, em notação indicial, é escrito como
	\begin{equation}
		W_{ij} = T_{ij} + S_{ij}.
	\end{equation}
	\item[b)] O produto de dois tensores, caracteriza-se por
	\begin{equation}
		( \textbf{T} \textbf{S}) \vec{a} = \textbf{T} ( \textbf{S} \vec{a})
	\end{equation}
	ou
	\begin{equation}
		(TS)_{ij} = T_{im} S_{mj}.
	\end{equation}
	\item[c)] Transposição de um tensor $( \textbf{T} ^{T})$
	\begin{equation}
		\vec{a} \cdot \textbf{T} \vec{b} = \vec{b} \cdot \textbf{T} ^{T} \vec{a}.
	\end{equation}
	\item[d)] O traço de um tensor \textbf{T} é representado em notação indicial por
	\begin{equation}
		tr \textbf{T} = T_{ij} \delta _{ij},
	\end{equation}
	onde
	\begin{equation}
		\delta _{ij} = \left\{
		\begin{array}{rcl}
			1 & se & i = j\\
			0 & se & i \neq j
		\end{array} \right.,
	\end{equation}
	é o delta de Kronecker.
	\item[e)] Um tensor \textbf{T} pode ser decomposto em uma parte simétrica, $ \textbf{T} ^{S} $, e uma parte antissimétrica, $ \textbf{T} ^{A} $, ou seja,
	\begin{equation}
		\textbf{T} = \textbf{T} ^{S} + \textbf{T} ^{A},
	\end{equation}
	\begin{equation}
		\textbf{T} ^{S} = \dfrac{ \textbf{T} + \textbf{T} ^{T}}{2},
	\end{equation}
	e
	\begin{equation}
		\textbf{T} ^{A} = \dfrac{ \textbf{T} - \textbf{T} ^{T}}{2}.
	\end{equation}
\end{enumerate}

No cálculo tensorial, se $ \textbf{T} = \textbf{T} (t) $ for um tensor de segunda ordem dependente do tempo, então
\begin{equation}
	\dfrac{d \textbf{T}}{dt} = \lim_{ \Delta t \to 0} \dfrac{ \textbf{T} ( t+ \Delta t) - \textbf{T} (t)}{ \Delta (t)}
\end{equation}
e
\begin{equation}
	\dfrac{d}{dt} [ \textbf{T} + \textbf{S}] = \dfrac{d \textbf{T}}{dt} + \dfrac{d \textbf{S}}{dt}.
\end{equation}

Se $ \alpha(t)$ for um escalar dependente do tempo, então
\begin{equation}
	\dfrac{d}{dt}  [ \alpha(t) \textbf{T}] = \dfrac{d \alpha(t)}{dt} \textbf{T} + \alpha(t) \dfrac{d \textbf{T}}{dt}.
\end{equation}

Se $ [ \textbf{T} \textbf{S}]$ for o produto entre dois tensores, então
\begin{equation}
	\dfrac{d}{dt} [ \textbf{T} \textbf{S}] = \dfrac{d \textbf{T}}{dt} \textbf{S} + \textbf{T} \dfrac{d \textbf{S}}{dt}. 
\end{equation}

Se $ [ \textbf{T} \vec{a}]$ representa a transformação de um vetor $ \vec{a}$, então
\begin{equation}
	\dfrac{d}{dt} [ \textbf{T} \vec{a}] = \dfrac{d \textbf{T}}{dt} \vec{a} + \textbf{T} \dfrac{d \vec{a}}{dt}.
\end{equation}

Se $[ \textbf{T} ^{T}]$ representar um tensor transposto, então
\begin{equation}
	\dfrac{d}{dt} [ \textbf{T} ^{T}] =
	\left[
	\begin{array}{c}
		\dfrac{d \textbf{T}}{dt}\\
	\end{array} \right] ^{T} .
\end{equation}

Para um campo vetorial o divergente de um vetor velocidade $ \vec{v}$ é calculado, segundo \citeonline{Malvern}, por
\begin{equation}
	\mbox{div} \vec{v} = tr [ { \nabla} \vec{v}] = { \nabla} \cdot \vec{v}
\end{equation} 
onde $ { \nabla} $ é o operador gradiente. Assim, tem-se que, em relação às suas componentes,
\begin{equation}
	\mbox{div} \vec{v} = \dfrac{ \partial v_{1}}{ \partial x_{1}} + \dfrac{ \partial v_{2}}{ \partial x_{2}} +\dfrac{ \partial v_{3}}{ \partial x_{3}}.
\end{equation}

Para um campo tensorial, o divergente deste campo é calculado pela equação
\begin{equation}
	(\mbox{div} \textbf{T}) \cdot \vec{a} = \mbox{div} ( \textbf{T} ^{T} \vec{a}) - tr ( \textbf{T} ^{T} ( { \nabla} \vec{a}))
\end{equation}
ou, em notação indicial,
\begin{equation}
	\mbox{div} \textbf{T} = \dfrac{ \partial T_{im}}{ \partial x_{m}} \vec{e} _{i}.
\end{equation}

Em certos problemas da engenharia que envolvem pequenos deslocamentos faz-se necessário conhecer e descrever suas deformações. Quando estas deformações são muito pequenas dá-se o nome de deformações infinitesimais \cite{Lai}.

Considerando a Figura \ref{fig:campodeform}, em um dado instante $ t_{0} $, os pontos $ P(t_{0}) $ e $ Q(t_{0}) $ apresentam um vetor distância $ \vec{dX} $. Para este instante $ t_{0} $ diz-se que os pontos estão na configuração $ B_{0} $. Após um certo instante de tempo $ t $ ocorre o deslocamento dos pontos $ P $ e $ Q $, sendo chamados agora de $ P(t) $, $ Q(t) $  e a nova configuração de $ B_{t} $, o que infere uma mudança em $  \vec{dX} $ passando a ser chamado de $  \vec{dx} $. Esta variação de deslocamento, por ser infinitesimal, pode ser considerada como um filamento. Logo, o filamento  $  \vec{dx} $ é dado por:
\begin{equation} \label{filamento}
	\vec{dx} = \vec{dX} + ( { \nabla} \vec{u}) \vec{dX},
\end{equation}
onde
\begin{equation}
	{ \nabla} \vec{u} = \left[
	\begin{array}{ccc}
		\dfrac{ \partial u_{1}}{ \partial x_{1}} & \dfrac{ \partial u_{1}}{ \partial x_{2}} & \dfrac{ \partial u_{1}}{ \partial x_{3}}\\
		
		\dfrac{ \partial u_{2}}{ \partial x_{1}} & \dfrac{ \partial u_{2}}{ \partial x_{2}} & \dfrac{ \partial u_{3}}{ \partial x_{3}}\\
		
		\dfrac{ \partial u_{3}}{ \partial x_{1}} & \dfrac{ \partial u_{3}}{ \partial x_{2}} & \dfrac{ \partial u_{3}}{ \partial x_{3}}\
	\end{array} \right],
\end{equation}
é o tensor gradiente de deslocamentos.

\begin{figure}[H]
	\centering
	\includegraphics[scale=1]{figuras/campo_de_deformacao.jpg}
	\caption{\textsc{Deformação infinitesimal}}
	\vspace{-0.1cm}
	\legend{FONTE: \citeonline{Lai}}
	\label{fig:campodeform}
\end{figure}

%\begin{figure}
%\caption{DEFORMAÇÃO INFINITESIMAL}
%\small{Fonte: LAI, 2010}
%\label{fig:campodeform}
%\centering
%\includegraphics[scale=1]{campo_de_deformacao.jpg}
%\end{figure}	


O filamento $ \vec{dx} $ na equação (\ref{filamento}) pode ser escrita como 
\begin{equation}
	\vec{dx} = [ \textbf{I} + ( { \nabla} \vec{u})] \vec{dX} 
\end{equation}
onde \textbf{I} é o  tensor identidade. Logo, 
\begin{equation} \label{filamento_dx}
	\vec{dx} = \textbf{F} \vec{dX},
\end{equation}
sendo
\begin{equation} \label{tensordeform}
	\textbf{F} = \textbf{I} + ( { \nabla}  \vec{u}).
\end{equation}
Mas
\begin{equation}
	\textbf{F} ^T \cdot \textbf{F} = ( \textbf{I} + { \nabla}  \vec{u}) ^T \cdot ( \textbf{I} + { \nabla}  \vec{u}) 
\end{equation}
o que resulta em
\begin{equation} \label{FtranspF}
	\textbf{F} ^T \cdot \textbf{F} = \textbf{I} + ( { \nabla}  \vec{u}) + ( { \nabla}  \vec{u}) ^T +  ( { \nabla}  \vec{u}) ^T ( { \nabla}  \vec{u}).
\end{equation}

Considerando que a magnitude do vetor $  \vec{u} $ é muito pequena, $  \Vert \vec{u} \Vert < < 1  $ , então       
\begin{equation}
	( { \nabla}  \vec{u}) ^T ( { \nabla}  \vec{u}) \approx 0,
\end{equation}
que, substituindo na equação (\ref{FtranspF}), resulta
\begin{equation}
	\textbf{F} ^T \cdot \textbf{F} = \textbf{I} + ( { \nabla}  \vec{u}) + ( { \nabla}  \vec{u}) ^T .
\end{equation}

Assumindo
\begin{equation}
	\textbf{E} = \dfrac{ ( { \nabla}  \vec{u}) + ( { \nabla}  \vec{u}) ^T}{2}
\end{equation}
como sendo um tensor simétrico de $ ( { \nabla}  \vec{u})   $, então
\begin{equation}
	\textbf{F} ^T \cdot \textbf{F} = \textbf{I} + 2 \textbf{E},
\end{equation}      
sendo \textbf{E} chamado de tensor de deformações infinitesimais.
Assim com foi feito nas equações (\ref{indicialT}) e (\ref{matrizT}), o tensor \textbf{E} também pode ser escrito na forma matricial
\begin{equation}
	[E] =  \left[ 
	\begin{array}{ccc}
		E_{11} & E_{12} & E_{13} \\
		E_{21} & E_{22} & E_{23} \\
		E_{31} & E_{32} & E_{33} \\
	\end{array}
	\right] _{X_{1} X_{2} X_{3}}.  
\end{equation}

No entanto, se as deformações ocorrerem somente nas direções principais, ou seja, nas direções dos autovetores  do tensor \textbf{E}, então
\begin{equation}
	[E] =  \left[ 
	\begin{array}{ccc}
		E_{1} & 0 & 0 \\
		0 & E_{2} & 0 \\
		0 & 0 & E_{3} \\
	\end{array}
	\right]  
\end{equation}
e, neste caso, haverá na transformação uma preservação dos ângulos, sendo chamada de uma transformação pura \cite{Lai}.

\begin{figure}[H]
	\centering
	\includegraphics[scale=1]{figuras/deformacoes.jpg}
	\caption{\textsc{Campo de deslocamentos}}
	\vspace{-0.1cm}
	\legend{FONTE: \citeonline{Lai}}
	\label{fig:campodesl}
\end{figure}

%\begin{figure}
%\caption{CAMPO DE DESLOCAMENTOS}
%\label{fig:campodesl}
%\centering
%\includegraphics[scale=1]{deformacoes.jpg}
%\end{figure}


A Figura \ref{fig:campodesl} apresenta o deslocamento de um dado corpo rígido no instante $ t_{0} $ e após um instante $ t $. Nota-se, em relação à sua configuração inicial, que este corpo sofreu dois movimentos, sendo um de rotação e o outro de deformação. Assim, sejam \textbf{U} e \textbf{V} tensores simétricos e \textbf{R} um tensor ortogonal próprio. Tomando como base a equação (\ref{tensordeform}), o tensor \textbf{F} pode ser escrito como
\begin{equation} \label{Caucy_dir}
	\textbf{F} = \textbf{R} \textbf{U}
\end{equation}    
e
\begin{equation} \label{Caucy_esq}
	\textbf{F} = \textbf{V} \textbf{R}.
\end{equation}


%-----------------------------------------------------
%para trabalhar com referencias para equações
%incluir o label com um nome para a equacao e referenciar no texto com ~\ref{label} 
%\begin{equation}\label{nome}
%\textbf{F} = \textbf{V} \textbf{R}.
%\end{equation}
%-----------------------------------------------------

As equações (\ref{Caucy_dir}) e (\ref{Caucy_esq}) são conhecidas como Teorema de decomposição polar e os tensores \textbf{U} e \textbf{V} como tensores de \textit{Strech} à direita e à esquerda, respectivamente.

Utilizando as equações (\ref{Caucy_dir}) e (\ref{Caucy_esq}), a equação (\ref{filamento_dx}) pode ser escrita da seguinte forma
\begin{equation}
	\vec{dx} =  \textbf{F} \vec{dX} = \textbf{R} \textbf{U} \vec{dX} = \textbf{R} ( \textbf{U} \vec{dX}),
\end{equation}
onde pode-se afirmar que, inicialmente, o corpo está sofrendo uma deformação pura (\textit{Strech}) e, após, uma rotação. Mas, pela equação (\ref{Caucy_esq}), pode-se ter também que
\begin{equation}
	\vec{dx} =  \textbf{F} \vec{dX} = \textbf{V} \textbf{R} \vec{dX} = \textbf{V} ( \textbf{R} \vec{dX}),
\end{equation}
onde inicialmente o corpo sofre uma rotação e então uma deformação. Em ambos os casos, o resultado será sempre o mesmo, como pode ser observado na Figura \ref{fig:campodeform}.

Dos tensores \textbf{U} e \textbf{V} surgem conceitos de significativa importância nas engenharias. Para tanto, seja
\begin{equation}
	\textbf{C} = \textbf{U} ^{2},
\end{equation}
onde o tensor \textbf{C} é chamado de tensor deformação de Cauchy-Green à direita, pois na equação (\ref{Caucy_dir}) o tensor \textbf{U} encontra-se à direita. Como este tensor é simétrico e o tensor \textbf{R} é ortogonal, tem-se
\begin{equation} \label{CG_direita}
	\textbf{C} = \textbf{F} ^{T} \textbf{F}.
\end{equation}

As componentes do tensor de Cauchy-Green à direita, $ C_{ij} $, tomando como base o tensor \textbf{F}, representam uma razão quadrática da medida  de deformação entre dois filamentos $ \vec{dx} ^{ (1)} $ e $ \vec{dx} ^{ (2)} $, se $ i=j $. Caso $ i \neq j $, $ C_{ij} $ irá representar a medida de distorção angular entre os dois filamentos.

Se as deformação não forem mais infinitesimais, mas sim finitas, tem-se o tensor Lagrangeano de deformações, $ \textbf{E} ^{*}$, sendo
\begin{equation}
	\textbf{E} ^{*} = \dfrac{1}{2} [( { \nabla} \vec{u}) + ( { \nabla} \vec{u}) ^{T}] + \dfrac{1}{2} ( { \nabla} \vec{u}) ^{T} ( { \nabla} \vec{u})
\end{equation}  
ou
\begin{equation}
	\textbf{E} ^{*} = \dfrac{1}{2} ( \textbf{C} - \textbf{I}).
\end{equation}

Em notação indicial o tensor Lagrangeano de deformações é escrito como
\begin{equation}
	E_{ij} ^{*} = \dfrac{1}{2} \left[ \dfrac{ \partial u_{i}}{ \partial X_{j}} + \dfrac{ \partial u_{j}}{ \partial X_{i}} \right] + \dfrac{1}{2} \dfrac{ \partial u_{m}}{ \partial X_{i}}  \dfrac{ \partial u_{m}}{ \partial X_{j}},
\end{equation}
onde $m$ e $n$ resultam de vetores unitários não mutuamente perpendiculares.

Seja \begin{equation}
	\textbf{B} = \textbf{V} ^{2},
\end{equation}
então o tensor \textbf{B} é chamado de tensor de Cauchy-Green à esquerda, pois na equação (\ref{Caucy_esq}) o tensor \textbf{V} está à esquerda. 

Assim como na equação (\ref{CG_direita}), tem-se 
\begin{equation}
	\textbf{B} = \textbf{V} ^{2} = \textbf{F} \textbf{F} ^{T} = ( \textbf{I} + { \nabla} \vec{u}) ( \textbf{I} + { \nabla} \vec{u}) ^{T}.
\end{equation}
Logo,
\begin{equation}
	\textbf{B} = \textbf{I} + [ { \nabla} \vec{u} + ( { \nabla} \vec{u}) ^{T}] + ( { \nabla} \vec{u}) ( { \nabla} \vec{u}) ^{T}
\end{equation}
que, em notação indicial, torna-se
\begin{equation} \label{comp_CG_esquerda}
	B_{ij} = \delta _{ij} + \left[ \dfrac{ \partial u_{i}}{ \partial X_{j}} + \dfrac{ \partial u_{j}}{ \partial X_{i}} \right] +  \dfrac{ \partial u_{i}}{ \partial X_{m}}  \dfrac{ \partial u_{j}}{ \partial X_{m}}.
\end{equation}

Se na equação (\ref{comp_CG_esquerda}) $ i=j $, assim como ocorre no tensor de Cauchy-Green à direta, $ B_{ij} $ representará a razão quadrática da medida de deformação entre os filamentos $ \vec{dx} ^{ (1)} $ e  $ \vec{dx} ^{ (2)} $. Caso $ i \neq j $, $ B_{ij} $ representará o quanto o ângulo entre os filamentos deixa de ser reto.

Se as deformações não forem mais infinitesimais, mas sim finitas, tem-se o tensor Euleriano de deformações,  $ \textbf{e} ^{*}$, sendo  
\begin{equation}
	\textbf{e} ^{*} = \dfrac{1}{2} [( { \nabla} \vec{u}) + ( { \nabla} \vec{u}) ^{T}] - \dfrac{1}{2} ( { \nabla} \vec{u}) ^{T} ( { \nabla} \vec{u})
\end{equation}
ou
\begin{equation}
	\textbf{e} ^{*} = \dfrac{1}{2} ( \textbf{I} - \textbf{B} ^{-1}).
\end{equation}

Em notação indicial o vetor $ \textbf{e} ^{*}$ torna-se
\begin{equation}
	e_{ij} ^{*} = \dfrac{1}{2} \left[ \dfrac{ \partial u_{i}}{ \partial X_{j}} + \dfrac{ \partial u_{j}}{ \partial X_{i}} \right] - \dfrac{1}{2} \dfrac{ \partial u_{m}}{ \partial X_{i}}  \dfrac{ \partial u_{m}}{ \partial X_{j}}.
\end{equation}

Um dado corpo é dito em equilíbrio se a resultante das forças que atuam sobre ele é nula \cite{Malvern}. Em outras palavras,
\begin{equation}
	\sum \vec{F} = 0.
\end{equation}  

Seja $ S $ um plano que passa em algum ponto arbitrário $ P $ de um corpo que possui $  \vec{n} $ como seu vetor normal unitário. Ao ser aplicada uma força externa  $ \vec{F}$ neste corpo o ponto $ P $ sofrerá uma tensão (pressão) que é dada pela razão da decomposição da força $ \vec{F} $, em relação à normal $ \vec{n} $, pela variação de área. A medida que a área diminui atingi-se um valor limite da tensão, 
\begin{equation}
	\vec{t} = \lim_{ A_{S} \to 0} \dfrac{ \Delta \vec{F}}{ \Delta A_{S}}
\end{equation}  
onde $ \vec{t} $ é chamado de vetor tensão e é entendido como a força resultante no ponto $ P $ em uma área infinitesimal. Por esta característica, o vetor tensão $ \vec{t} $ tem dimensão de pressão, isto é, $ N/m^{2} $, $ kgf/cm^{2} $, $ lb/in^{2} $ \cite{Lai}.

\begin{figure}[H]%COLOCAR FIGURA PAG 155 LAI
	\centering
	\includegraphics[scale=1]{figuras/vetor_tensao.jpg}
	\caption{\textsc{Plano de formação do vetor de tensão}}
	\vspace{-0.1cm}
	\legend{FONTE: \citeonline{Lai}}
	\label{fig:vetortensao}
\end{figure}
%\begin{figure}%COLOCAR FIGURA PAG 155 LAI
%\caption{PLANO DE FORMAÇÃO DO VETOR DE TENSÃO}
%\label{fig:vetortensao}
%\centering
%\includegraphics[scale=1]{vetor_tensao.jpg}
%\end{figure}	

O vetor tensão $  \vec{t} $ depende tanto da posição do ponto $ P $ como também da normal do plano $ S $. Assim,
\begin{equation}
	\vec{t} = \vec{t} ( \vec{x}, t, \vec{n}) = \textbf{T} ( \vec{x}, t) \vec{n}
\end{equation} 
ou, simplesmente,
\begin{equation} \label{vetor_tensao}
	\vec{t} _{ \vec{n}} = \textbf{T} \vec{n}.
\end{equation}

\begin{figure}[H]%COLOCAR A FIGURA 4.2.1 PAG 157
	\centering
	\includegraphics[scale=1]{figuras/componentes_vetor_tensao.jpg}
	\caption{\textsc{Componentes do vetor de tensão}}
	\vspace{-0.1cm}
	\legend{FONTE: \citeonline{Lai}}
	\label{fig:compvetortensao}
\end{figure}
%\begin{figure}%COLOCAR A FIGURA 4.2.1 PAG 157
%\caption{COMPONENTES DO VETOR DE TENSÃO}
%\label{fig:compvetortensao}
%\centering
%\includegraphics[scale=1]{componentes_vetor_tensao.jpg}
%\end{figure}	

Tomando como base a Figura \ref{fig:compvetortensao} o vetor de tensões pode ser escrito em função das suas componentes como
\begin{equation}
	\vec{t} _{ \vec{n}} = n_{1} \vec{t} _{ \vec{e} _{1}} + n_{2} \vec{t} _{ \vec{e} _{2}} + n_{3} \vec{t} _{ \vec{e} _{3}}.
\end{equation}

Assim,
\begin{equation} \label{componente_vetor_tensao}
	\begin{array}{c}
		\vec{t} _{ \vec{e} _{1}} = T_{11} \vec{e} _{1} + T_{21} \vec{e} _{2} + T_{31} \vec{e} _{3}\\
		\vec{t} _{ \vec{e} _{2}} = T_{21} \vec{e} _{1} + T_{22} \vec{e} _{2} + T_{32} \vec{e} _{3}\\
		\vec{t} _{ \vec{e} _{3}} = T_{31} \vec{e} _{1} + T_{31} \vec{e} _{2} + T_{33} \vec{e} _{3}\\
	\end{array}
\end{equation}
ou, utilizando notação indicial,
\begin{equation}
	\vec{t} _{ \vec{e} _{i}} = T_{mi} \vec{e} _{m},
\end{equation}
sendo $ T_{mi} $ com $ m=i  $, chamado componente tangencial, $ \sigma _{ \vec{n}} $, e quando $ m \neq i  $, $ T_{mi} $ é chamado de componente cisalhante, $ \vec{\tau} _{S}$, cuja magnitude é calculada por
\begin{equation}
	\Vert \vec{\tau} _{S} \Vert = \sqrt{ \Vert \vec{t} _{ \vec{n}} \Vert ^{2} - \Vert \sigma _{ \vec{n}} \Vert ^{2}}.
\end{equation}

De acordo com as equações (\ref{vetor_tensao}) e (\ref{componente_vetor_tensao}), $ \textbf{T} $ é uma transformação linear sendo chamado de tensor de tensões ou tensor de tensões de Cauchy \cite{Lai}.

O tensor de tensões de Cauchy, de acordo com a sua formulação, está definido na configuração deformada $ B_{t} $, conforme pode ser observado nas Figuras \ref{fig:campodesl} e \ref{fig:campodeform}. No entanto, em certos problemas das engenharias, há a necessidade de se avaliar os estados de tensões na configuração $ B_{0} $. Para tanto, seja $ \vec{df} $ um vetor força definido em uma área infinitesimal $ dA $. Então, 
\begin{equation}
	\vec{df}  = \vec{t} dA
\end{equation}
onde, pela equação (\ref{vetor_tensao}), 
\begin{equation}
	\vec{t} = \textbf{T} \vec{n}.
\end{equation}

Se for possível escrever o vetor $ \vec{df} $ em função de um vetor tensão, $ \vec{t} _{0} $, em $ B_{0} $ e em relação a uma área indeformada, então
\begin{equation}
	\vec{df}  = \vec{t} dA \Longleftrightarrow \vec{df}  = \vec{t_{0}} dA_{0}.
\end{equation}
Logo,
\begin{equation}
	\vec{df}  = \vec{t} dA  = \vec{t_{0}} dA_{0},
\end{equation}
isto é,
\begin{equation}
	\vec{t_{0}} = \dfrac{dA}{dA_{0}} \vec{t}
\end{equation}
que, pela equação (\ref{vetor_tensao}),
\begin{equation} \label{vetor_tensao_B0}
	\vec{t}_{0} = \textbf{T}_{0} \vec{n}_{0} = \textbf{T} \dfrac{dA}{dA_{0}} \vec{n}. 
\end{equation}

Como, segundo \citeonline{Malvern}, 
\begin{equation}
	dA \vec{n} = dA_{0} \vert \textbf{F} \vert ( \textbf{F} ^{-1}) ^{T} \vec{n} _{0}
\end{equation}
a equação em (\ref{vetor_tensao_B0}) pode ser escrita como
\begin{equation}
	\vec{t}_{0}  \vec{n} _{0} = \textbf{T}  \vert \textbf{F} \vert ( \textbf{F} ^{-1}) ^{T} \vec{n} _{0}.
\end{equation}
Logo,
\begin{equation} \label{tensor_PK}
	\textbf{T} _{ \textbf{0}} = \textbf{T}  \vert \textbf{F} \vert ( \textbf{F} ^{-1}) ^{T},
\end{equation}
onde $ \textbf{T} _{ \textbf{0}} $ é conhecido como o primeiro tensor de Piolla-Kirchhoff.

Como o tensor de tensões de Cauchy \textbf{T} é simétrico e o tensor \textbf{F} não é simétrico, o primeiro tensor de Piolla-Kirchhoff não será simétrico. Para contornar tal problema pode-se considerar um  "pseudo" \ tensor força $ \tilde{df} $ na área $ dA_{0} $, onde
\begin{equation}
	\tilde{df} = \tilde{t} dA_{0}
\end{equation} 
sendo $ \tilde{t} $ um "pseudo" \ vetor tensão na área $ dA_{0} $, calculado por
\begin{equation}
	\tilde{t} =  \tilde{ \textbf{ T}} \vec{n} _{0}.
\end{equation}

Procedendo de forma análoga ao que foi feito com primeiro tensor de Piolla-Kirchhoff, obtêm-se
\begin{equation}
	\tilde{ \textbf{ T}}  = \textbf{F} ^{-1} \textbf{T} _{0}
\end{equation}
ou, pela equação (\ref{tensor_PK}),
\begin{equation}
	\tilde{ \textbf{T}} = \vert \textbf{F} \vert ( \textbf{F} ^{-1}) \textbf{T}   ( \textbf{F} ^{-1}) ^{T},
\end{equation}
sendo agora $ \tilde{ \textbf{T}} $ simétrico e chamado de segundo tensor de Piolla-Kirchhoff.




%  % ----------------------------------------------------------
\chapter{MODELO HÍBRIDO}

Modelo híbrido desenvolvido, em Pascal, para solucionar o problema da ruptura de barragens governado pelas equações de Águas Rasas unidimensional. 

\begin{verbatim}

	program rb02;
	
	uses
	wincrt;
	
	type
	vet = array[0..1000] of double;
	
	var
	az,au   : text;
	x       : ^vet;
	u,z     : array[1..2] of ^vet;
	t,ti,
	c0,zm,
	dx,dt,
	msdx,z0 : double;
	i,i2,
	nx1,
	nx2     : integer;
	
	const
	h  : double  = 1.0;
	lp : double  = 2.0;
	lg : double  = 4.0;
	g  : double  = 9.807;
	ep : double  = 0.00001;
	nx : integer = 400;
	ni : integer = 500;
	
	
	procedure sol_ex(t:double);
	
	var
	x1,x2,a : double;
	i1,i2,i : integer;
	
	begin
	
	x1 := h - t*c0;
	x2 := h + 2.0*t*c0;
	
	i1 := trunc(ep+(x1/dx));
	i2 := trunc(ep+(x2/dx));
	
	for i := i1+1 to i2-1 do
	begin
	
	a := (2.0*c0 + (h-x^[i])/t)/3.0;
	
	z[2]^[i] := a*a/g;
	u[2]^[i] := 2.0*(c0 - a);
	
	end;
	
	end;
	
	procedure sol_num(t:double);
	
	var
	dzdx,dudx,
	a,zx,ux,
	zmed      : double;
	i1,i2,i   : integer;
	
	begin
	
	z[1] := z[2];
	u[1] := u[2];
	
	for i := 1 to nx1 do
	begin
	
	i1 := i - 1;
	i2 := i + 1;
	
	zx := 0.5*(z[1]^[i1] + z[1]^[i2]);
	ux := 0.5*(u[1]^[i1] + u[1]^[i2]);
	
	dzdx := msdx*(z[1]^[i2] - z[1]^[i1]);
	dudx := msdx*(u[1]^[i2] - u[1]^[i1]);
	
	a := ux*dudx + g*dzdx;
	u[2]^[i] := ux - a*dt;
	
	a := zx*dudx + ux*dzdx;
	z[2]^[i] := zx - a*dt;
	
	end;
	
	z[2]^[0]  := z[2]^[1];
	z[2]^[nx] := z[2]^[nx-1];
	
	zmed := 0.0;
	for i := 0 to nx do
	zmed := zmed + z[2]^[i];
	
	a := nx*zm/zmed;
	for i := 0 to nx do
	z[2]^[i] := a*z[2]^[i];
	
	end;
	
	procedure gravar(t:double);
	
	var
	i : integer;
	
	begin
	
	writeln(az);
	write  (az,t);
	
	writeln(au);
	write  (au,t);
	
	i := 0;
	repeat
	
	write(az,' ',z[2]^[i]);
	write(au,' ',u[2]^[i]);
	
	i := i + nx2;
	
	until i > nx;
	(*
	write(az,' ',z[2]^[nx]);
	write(au,' ',u[2]^[nx]);
	*)
	end;
	
	begin
	
	new(x);
	for i := 1 to 2 do
	begin
	new(u[i]);
	new(z[i]);
	end;
	
	nx1  := nx - 1;
	nx2  := nx div 20;
	dx   := lg/nx;
	dt   := 0.01*dx;
	msdx := 0.5/dx;
	
	i2 := trunc((h/dx)+ep);
	
	for i := 0 to nx do
	begin
	x^[i]    := i*dx;
	u[2]^[i] := 0.0;
	end;
	
	for i := 0 to i2 do
	z[2]^[i] := lp;
	
	for i := i2+1 to nx do
	z[2]^[i] := 0.0;
	
	zm := lp*h/lg;
	c0 := sqrt(g*lp);
	
	writeln(zm);
	
	t  := h/c0;
	ti := 0.5*(lg-h)/c0;
	
	if ti > t
	then
	ti := t;
	
	writeln(ti);
	readln;
	
	assign(az,'rb02z.txt');
	assign(au,'rb02u.txt');
	
	rewrite(az);
	rewrite(au);
	
	writeln(az);
	write  (az,'        t(s)\x(m)      ');
	writeln(au);
	write  (au,'        t(s)\x(m)      ');
	
	i := 0;
	repeat
	
	write(az,' ',x^[i]);
	write(au,' ',x^[i]);
	
	i := i + nx2;
	
	until i > nx;
	(*
	write(az,' ',x^[nx]);
	write(au,' ',x^[nx]);
	*)
	t := ni*dt;
	repeat
	
	writeln(t);
	
	sol_ex(t);
	
	writeln(t,z[2]^[0],z[2]^[nx]);
	
	gravar(t);
	
	t := t + ni*dt;
	
	until t > ti;
	
	t := t - ni*dt;
	
	repeat
	
	z0 := z[2]^[0];
	
	for i := 1 to ni do
	begin
	
	t := t + dt;
	
	sol_num(t);
	
	end;
	
	gravar(t);
	
	writeln(t,z[2]^[0],z[2]^[nx]);
	
	until abs(z0-z[2]^[0]) < ep;
	
	close(az);
	close(au);
	
	end.
	
\end{verbatim}

% ----------------------------------------------------------
%\lipsum[55-57]
%\end{apendicesenv}

%\begin{anexosenv}% Anexos: inserir se necessário
%\partanexos
% \input{Extras/anexo1}
% \input{Extras/anexo2}
% \input{Extras/anexo3}
%\end{anexosenv}

\printindex % Indice Remissivo
\end{document}
