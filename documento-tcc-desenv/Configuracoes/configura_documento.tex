%----------------------------------------------------------------- 
% CAPA
\renewcommand{\imprimircapa}{%
  \begin{capa}
  	\begin{center} %Isso eu fiz
  		\large{FACULDADE ANGLO-AMERICANO DE FOZ DO IGUAÇU}
  	\end{center}
    \center
    %{\imprimirautor} %Dica da Tássia
        
    %{\ABNTEXchapterfont\large\imprimirautor} %Original
    {\ABNTEXchapterfont\imprimirautor} %Isso eu fiz

    \vspace*{8cm}
    %{\imprimirtitulo} Dica da Tássia
    \ABNTEXchapterfont\bfseries\large\imprimirtitulo
    \vfill
    
    \imprimirlocal

    \imprimirdata
    
    \vspace*{1cm}
  \end{capa}
}

\makeatletter
%----------------------------------------------------------------- 
% FOLHA DE ROSTO
\renewcommand{\folhaderostocontent}{
		\center	
		%{\large\textbf\imprimirautor} %Original
			{\large\imprimirautor} %Eu fiz
	
	    \vspace*{8cm}		
		
		
		{\large\textbf\imprimirtitulo}
	
		\vspace*{3cm}	

		\abntex@ifnotempty{\imprimirpreambulo}{
			\hspace{.3\textwidth}
			\begin{minipage}{.55\textwidth}
			
				\SingleSpacing
				\imprimirpreambulo				
				\SingleSpacing				
				{\imprimirorientadorRotulo~\imprimirorientador\par}
				\SingleSpacing	
		        \abntex@ifnotempty{\imprimircoorientador}{
		             {\imprimircoorientadorRotulo~\imprimircoorientador}
		}
		\end{minipage}
		
			\vspace*{\fill}
		}
		
		
		{\large\textbf\imprimirlocal}
		
		{\large\textbf\imprimirdata}		
}

%****************************************************
%outras configurações 
%***********************************************************
\setlength{\parindent}{1.3cm} % Tamanho do parágrafo
\setlength{\parskip}{0.2cm}  % Controle do espaçamento entre um parágrafo e outro
	% Altera o tamanho das fontes dos capítulos e dos apêndices
\renewcommand{\ABNTEXchapterfont}{\bfseries}
\renewcommand{\ABNTEXchapterfontsize}{\large}
\renewcommand{\ABNTEXsectionfontsize}{\normalfont}

% Separação de Palavras:
%\hyphenation{}
\hyphenation{teorias}
\hyphenation{baseados}
\hyphenation{MATLAB}
\hyphenation{Pyramid}
\hyphenation{Mar-cação}
\hyphenation{lin-guagem}
\hyphenation{matplotlib}
\hyphenation{sociali-zação}
\hyphenation{bi-bli-oteca}
\hyphenation{Notebook}
\hyphenation{profis-si-onais}
\hyphenation{ar-quivos}
\hyphenation{usu-ário}
\hyphenation{fer-ramentas}
\hyphenation{materi-al}
\hyphenation{re-almen-te}
\hyphenation{LinkedIn}
\hyphenation{proces-samento}

	% Configura layout para elementos textuais

\makepagestyle{abnt_ufpr}

\makeoddhead{abntchapfirst}{}{}{\ABNTEXfontereduzida\thepage}

\pagestyle{abnt_ufpr}



\newtheorem{theorem}{Teorema}[chapter]
\newtheorem{definition}[theorem]{Defini\c{c}\~{a}o}
\newtheorem{proposition}[theorem]{Proposi\c{c}\~{a}o}

\makeatother
%********************************************************************
%*********************************************************************
% Essa parte para baixo até o fim foi tudo copiado da TASSIA

%\newcommand{\quadroname}{Quadro}
%\newcommand{\listofquadrosname}{Lista de quadros}

%\newfloat[chapter]{quadro}{loq}{\quadroname}
%\newlistof{listofquadros}{loq}{\listofquadrosname}
%\newlistentry{quadro}{loq}{0}

% configurações para atender às regras da ABNT
%\counterwithout{quadro}{chapter}
%\renewcommand{\cftquadroname}{\quadroname\space}
%\renewcommand*{\cftquadroaftersnum}{\hfill--\hfill}

%****************************************************
%outras configurações
%***********************************************************
%\setlength{\parindent}{1.3cm} % Tamanho do parágrafo
%\setlength{\parskip}{0.0cm}  % Controle do espaçamento entre um parágrafo e outro

% Altera o tamanho das fontes dos capítulos e dos apêndices

\renewcommand{\ABNTEXchapterfont}{\normalfont\fontseries{b}\selectfont}
\renewcommand{\ABNTEXchapterfontsize}{\normalsize}
\renewcommand{\ABNTEXpartfont}{\fontseries{b}\selectfont\selectfont}
\renewcommand{\ABNTEXpartfontsize}{\normalsize}
\renewcommand{\ABNTEXsectionfont}{\normalfont\selectfont}
\renewcommand{\ABNTEXsectionfontsize}{\normalsize}
\renewcommand{\ABNTEXsubsectionfont}{\normalfont\selectfont}
\renewcommand{\ABNTEXsubsectionfontsize}{\normalsize}
\renewcommand{\ABNTEXsubsubsectionfont}{\normalfont\selectfont}
\renewcommand{\ABNTEXsubsubsectionfontsize}{\normalsize}
\renewcommand{\ABNTEXsubsubsubsectionfont}{\normalfont\itshape\selectfont}
\renewcommand{\ABNTEXsubsubsubsectionfontsize}{\normalsize}


% CONFIGURACAO DO SUMARIO

% Sumário
\renewcommand*{\cftsectionfont}{\normalfont}
\renewcommand*{\cftsubsubsectionfont}{\normalfont}
\renewcommand*{\cftsubsectionfont}{\normalfont}
\renewcommand*{\cftparagraphfont}{\normalfont\itshape}

% -----------------------------------------------------------------------------
% Modifica o espaçamento no sumário
% Nao ha espacos, para as entradas de capitulos
\setlength{\cftbeforeparagraphskip}{0pt}
\setlength{\cftbeforesubsectionskip}{0pt}
\setlength{\cftbeforesectionskip}{0pt}
\setlength{\cftbeforesubsubsectionskip}{0pt}
\setlength{\cftbeforechapterskip}{0pt}



\addto\captionsbrazil{
	\renewcommand{\bibname}{REFER\^ENCIAS}
}

\addto\captionsbrazil{\renewcommand{\listadesiglasname}{LISTA DE ABREVIATURAS}}
\addto\captionsbrazil{\renewcommand{\listfigurename}{LISTA DE ILUSTRAÇÕES}}
\addto\captionsbrazil{\renewcommand{\listtablename}{LISTA DE TABELAS}}
\addto\captionsbrazil{%
  \renewcommand*{\lstlistlistingname}{LISTA DE CÓDIGOS}%
  \renewcommand*{\lstlistingname}{CÓDIGO}%
}
\addto\captionsbrazil{\renewcommand{\contentsname}{SUMÁRIO}}
\addto\captionsbrazil{\renewcommand\appendixtocname{APÊNDICES}\renewcommand\appendixpagename{APÊNDICES}}
\addto\captionsbrazil{\renewcommand{\figurename}{FIGURA}}
\addto\captionsbrazil{\renewcommand{\tablename}{TABELA}}


% Configura layout para elementos textuais

%\makepagestyle{abnt_ufpr}


%\makeoddhead{abntchapfirst}{}{}{\ABNTEXfontereduzida\thepage}

%\pagestyle{abnt_ufpr}

%%%%%%%% AMBIENTES %%%%%%%%%%
%\newtheorem{teo}{Teorema}[chapter]
%\newtheorem{cor}[teo]{Corol\'{a}rio}
%\newtheorem{lem}[teo]{Lema}
%\newtheorem{prop}[teo]{Proposi\c{c}\~{a}o}
%\newtheorem{defn}[teo]{Defini\c{c}\~{a}o}
%\newtheorem{Ex}[teo]{Exemplo}
%\newtheorem{obs}[teo]{Observa\c{c}\~{a}o}
%\newtheorem{prob}[teo]{Problema}
%\newtheorem{conc}[teo]{Conclusão}
%\newenvironment{dem}{\smallskip \noindent{\bf Demonstra\c{c}\~{a}o}: }
%{\hfill $\Box$\hspace{0in}\medskip}

%%%%%%%% COMANDOS FRECUENTES %%%%%%%%
%\newcommand{\eq}{\begin{equation}}
%	\newcommand{\ee}{\end{equation}}
%\newcommand{\R}{{\mathbb R}}
%\newcommand{\N}{{\mathbb N}}
%\newcommand{\K}{{\mathbb K}}
%\newcommand{\Q}{{\mathbb Q}}
%\newcommand{\Z}{{\mathbb Z}}
%\newcommand{\V}{{\mathbb V}}
%\newcommand{\D}{{\mathcal{D}}}
%\newcommand{\C}{{\mathbb C}}
%\newcommand{\di} {\displaystyle}
%\newcommand{\I}{{\displaystyle\int_{0}^{T} \displaystyle\int_{0}^1 }}
%\newcommand{\Ia}{{\displaystyle\int_{0}^{1} \displaystyle\int_{0}^T }}
%\newcommand{\Ii}{{\displaystyle\int_{0}^{t} \displaystyle\int_{0}^1 }}
%\newcommand{\Ib}{{\displaystyle\int_{0}^{1} \displaystyle\int_{0}^t }}
%%%%%%%%%%%%%%%%%%%%%%%%%%%
%\makeatother 