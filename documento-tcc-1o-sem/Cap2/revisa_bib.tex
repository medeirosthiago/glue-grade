\chapter{REVISÃO BIBLIOGRÁFICA}\label{ch:rev-bibs}

Alguns trabalhos serviram como ajuda e inspiração para este estudo, porém durante o período de busca por bibliografias a respeito de \textit{data mining} pouco material de qualidade foi encontrado quanto a mineração de redes sociais.

Ainda sim, alguns estudos possuem um destaque e é necessário citá-los devido a utilização de ferramentas específicas em \textit{data mining} e também aos resultados obtidos neste processo.

\citeonline{credito-bancario} afirma que um dado se transforma em informação quando ganha um significado para seu utilizador, caso contrário, continua sendo simplesmente um dado.

Em seu estudo \citeonline{credito-bancario} aborda duas técnicas de \textit{data mining}: Árvores de Decisão e Redes Neurais, para realizar a análise de crédito bancário. Estas técnicas permitem fazer o reconhecimento de padrões e também diagnosticar novos casos. Através deste modo, os analistas têm condições de diagnosticar os novos clientes, quanto ao merecimento de crédito ou não. Estas técnicas de mineração de dados são ferramentas que podem ser utilizadas pelo especialista para auxiliá-lo nas tomadas de decisão, nunca porém poderão por si só, substituir a figura do especialista no contexto da análise de crédito.

Semelhante as técnicas abordadas para a análise de crédito bancário, \citeonline{diag-medico} realizaram um estudo na área médica, onde a posse e uso de ferramentas que auxiliem na tarefa de classificação de pacientes em prováveis ictéricos com câncer ou ictéricos com cálculo, pode ser crucial. Em uma tentativa de otimizar todo o processo do diagnóstico, minimizando riscos e custos e, por outro lado, maximizando a eficácia nos resultados, utilizaram técnicas de  \textit{data mining} como um processo para extração de informações valiosas. O trabalho consiste na análise de 118 históricos de pacientes utilizando Árvores de Decisão e Regras de Classificação como ferramenta. \citeonline{diag-medico} afirmam que os métodos de \textit{data mining} apresentam a vantagem de deixar claro ao usuário quais são os atributos que estão discriminando os padrões e de que forma a mesma está ocorrendo, isso é uma característica altamente desejável em qualquer técnica de reconhecimento de padrões.

O reconhecimento de padrões permite a obtenção de conhecimento da frequência com que determinadas seções de uma página \textit{web} são acessadas e quais são os serviços mais procurados. Isso possibilita que empresas consigam descobrir o perfil de seus usuários e, com base nesse acontecimento, ofertar serviços e atendimento personalizado. Essa afirmação foi resultado do estudo de \citeonline{logs-web}, onde realizaram a mineração de dados em \textit{logs} de acesso a servidores \textit{web}. Para o desenvolvimento do trabalho foram utilizadas regras de associação para a descoberta e representação de padrões frequentes em conjuntos de dados. Isso propiciou a identificação de padrões de comportamento de usuários da Internet ao navegarem por \textit{websites}. \citeonline{logs-web} também concluem que outras ferramentas de mineração podem ser aplicadas, visando aumentar a flexibilidade de manipulação dos atributos específicos para o ambiente \textit{Web}.

Como visto, o tema sobre \textit{data mining} possui uma abordagem comum a setores e áreas diferentes, o que caracteriza enorme quantidade de dados que ainda não se tem informação útil para seus utilizadores. Por mais que as técnicas utilizadas dentre essa diversidade de áreas sejam semelhantes, o desenvolvimento e aplicação das técnicas são particulares para cada caso. Neste contexto, esta dissertação modela, de uma forma mais simplificada, o reconhecimento de padrões para a predição dos dados extraídos da rede social \textit{LinkedIn}.











