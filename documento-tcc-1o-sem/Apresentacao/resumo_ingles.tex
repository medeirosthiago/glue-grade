\begin{resumo}[ABSTRACT]
 \begin{otherlanguage*}{english}
   It is natural to human beings the need to relate to other people and it is through social networks that these relationships become real worldwide. \textit{LinkedIn} is one of these social networks and its goal is to develop professional relationships. The interaction between people in these online environments end up generating a lot of data that at some point, will need to become useful information. It is then necessary to work with that data, find ways to automate the analysis, classification, summarization, discovery and characterization of them and also point out some anomalies. Data mining comes from this need. Through interdisciplinary, researchers from several areas, including statistics, engineering, artificial intelligence and machine learning, are contributing and creating tools for this field. Python has been used as a tool for data mining thanks to its strong power of programming and also its libraries that allow the analysis and data mining. Finally, this paper aims to mine the data from \textit{LinkedIn}, in order to find patterns in Information Technology professionals' profiles.
   
   
   \vspace{\onelineskip}
 
   \noindent 
   \textbf{Keywords}: Data. Data Mining. LinkedIn. Python.
 \end{otherlanguage*}
\end{resumo}
