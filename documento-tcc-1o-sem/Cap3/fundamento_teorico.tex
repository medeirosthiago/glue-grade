\chapter{FUNDAMENTAÇÃO TEÓRICA}\label{ch:fundaments-teorico}
A mineração de dados é um assunto totalmente interdisciplinar podendo ser definido de diversas maneiras. Até mesmo o termo \textit{data mining} não representa realmente todos os componentes desta área. \citeonline{han} exemplificam esta questão comentando sobre a mineração de ouro através da extração de rocha e areia, que é chamado de mineração de ouro e não mineração de rochas ou mineração de areia. Analogamente, a mineração de dados deveria se chamar "mineração de conhecimento através de dados", que infelizmente é um termo um tanto longo. Entretanto, uma referência mais curta, como "mineração de conhecimento", pode não enfatizar a mineração de uma enorme quantidade de dados. Apesar disso, a mineração é um termo que caracteriza o processo de encontrar uma pequena quantia de uma preciosa pepita em uma grande quantidade de matéria bruta. Nesse sentido, um termo impróprio contendo ambos \textit{"data"} e \textit{"mining"} se tornou popular e, como consequência, muitos outros nomes similares surgiram: \textit{knowledge mining from data}, \textit{knowledge extraction}, \textit{data/pattern analysis}, \textit{data archaeology} e \textit{data dredging}.

Na Seção~\ref{sec:kdd} será abordado os conceitos de descoberta de conhecimento em base de dados e a sua diferença em relação ao \textit{data mining}. Logo, na Seção~\ref{sec:data-mining}, é apresentado os métodos e a concepção de \textit{data mining}. Alguns desses métodos tem como prática o uso de aprendizado de máquina que será definido na Seção~\ref{sec:machine-learning}. A Seção~\ref{sec:python} irá caracterizar a linguagem de programação Python e qual a sua vantagem em utilizá-la para a mineração de dados.

%%
% KDD
\section{DESCOBERTA DE CONHECIMENTO EM BASE DE DADOS E \textit{DATA MINING}}\label{sec:kdd}
Muitas pessoas tratam a mineração de dados como um sinônimo para outro termo muito popular, descoberta de conhecimento em base de dados (\textit{knowledge discovery from data}) - KDD, enquanto outros referenciam \textit{data mining} como apenas uma etapa no processo de descoberta de conhecimento em base de dados. O processo de KDD é demonstrado através da  Figura~\ref{kdd-fig} como uma sequência interativa e iterativa dos seguintes passos: \\ \\ \\ \\ \\ \\

\begin{figure}
	\centering
	\includegraphics[width=1\textwidth]{Cap3/imagens/kdd}
	\caption{\textsc{Etapas do processo de KDD}}
	\vspace{-0.3cm}
	\legend{\small{FONTE: Adaptado de \citeonline{han}}}
	\label{kdd-fig}
\end{figure}

\begin{enumerate}
	\item \textit{Data cleaning} (Limpeza de dados);
	\item \textit{Data integration} (Integração de dados);
	\item \textit{Data selection} (Seleção de dados);
	\item \textit{Data transformation} (Transformação de dados);
	\item \textit{Data mining} (Mineração de dados);
	\item \textit{Pattern evaluation} (Avaliação de padrões);
	\item \textit{Knowledge presentation} (Apresentação de conhecimento).
\end{enumerate}


De acordo com \apudonline{brachman}{fayyad2}, as etapas são interativas porque envolvem a cooperação da pessoa responsável pela análise de dados, cujo conhecimento sobre o domínio orientará a execução do processo. Por sua vez, a interação deve-se ao fato de que, com frequência, esse processo não é executado de forma sequencial, mas envolve repetidas seleções de parâmetros e conjuntos de dados, aplicações das técnicas de \textit{data mining} e posterior análise dos resultados obtidos, a fim de refinar os conhecimentos extraídos.

KDD refere-se ao processo global de descobrimento de conhecimento útil em bases de dados. \textit{Data mining} é um passo particular neste processo aplicação de algoritmos específicos para extrair padrões (modelos) de dados. Os passos adicionais no processo KDD, como integração de dados, limpeza de dados, seleção de dados, incorporação de conhecimento anterior apropriado e interpretação formal dos resultados de mineração assegura aquele conhecimento útil que é derivado dos dados. A aplicação cega de métodos de \textit{data mining} pode ser uma atividade perigosa que conduz a descoberta de padrões sem sentido \cite{navega}. 

O KDD evoluiu e continua evoluindo da interseção de pesquisas em campos como bancos de dados, aprendizado de máquinas (\textit{machine learning}), reconhecimento de padrões, estatísticas, inteligência artificial, aquisição de conhecimento para sistemas especialistas, visualização de dados, descoberta científica, recuperação de informação e computação de alto-desempenho. Aplicações de KDD incorporam teorias, algoritmos e métodos de todos estes campos \cite{credito-bancario}.


%%
% Data Mining
\section{\textit{DATA MINING}}\label{sec:data-mining}
Apesar do conceito de \textit{data mining}, na maioria das vezes, ser utilizado pelas indústrias, mídias e centros de pesquisa para se referir ao processo de descoberta de conhecimento considerado em sua globalidade, o termo \textit{data mining} poderá ser usado também para indicar o quinto estágio do KDD, sendo um processo essencial na descoberta e extração de padrões de dados. \citeonline{han}, adotam uma visão mais abrangente para a funcionalidade de mineração de dados: \textit{data mining} é o processo de descoberta de padrões interessantes e conhecimentos de um vasto conjunto de dados. A fonte dos dados pode ser banco de dados, \textit{data warehouses}, a Internet, outros repositórios de informações, ou dados correntes em sistemas dinâmicos.

Uma das definições, talvez, mais importante de \textit{data mining} foi elaborada por \citeonline{fayyad} "...o processo não-trivial de identificar, em dados, padrões válidos, novos, potencialmente úteis e ultimamente compreensíveis".

\textit{Data mining} ou mineração de dados, pode ser entendido, então como o processo de extração de informações, sem conhecimento prévio, de algum conjunto de dados e seu uso para tomada de decisões. A mineração de dados se define através de processos automatizados de captura e análise deste conjunto de dados com a finalidade de extrair algum significado, podendo descrever características do passado, como também para predizer futuras tendências \cite{conceito-data-mining}.

Diversos métodos são usados em \textit{data mining} para encontrar respostas ou extrair conhecimento interessante. Esses podem ser obtidos através dos seguintes métodos:

\begin{itemize}
	\item Classificação: associa ou classifica um item a uma ou várias classes. Os objetivos dessa técnica envolvem a descrição gráfica ou algébrica das características diferenciais das observações de várias populações. A ideia principal é derivar uma regra que possa ser usada para classificar, de forma otimizada, uma nova observação a uma classe já rotulada;
	
	\item Modelos de Relacionamento entre Variáveis: associa um item a uma ou mais variáveis de predição de valores reais, conhecidas como variáveis independentes ou exploratórias. Nesta etapa se destacam algumas técnicas estatísticas como regressão linear simples, múltipla e modelos lineares por transformações, com o objetivo de verificar o relacionamento funcional entre duas variáveis quantitativas, ou seja, constatar se há uma relação funcional entre X e Y;
	
	\item Análise de Agrupamento (\textit{Cluster}): associa um item a uma ou várias classes (ou \textit{clusters}). Os \textit{clusters} são definidos por meio do agrupamento de dados baseados em modelos probabilísticos ou medidas de similaridade. Analisar \textit{clusters} é uma técnica com o objetivo de detectar a existência de diferentes grupos dentro de um determinado conjunto de dados e, caso exista, determinar quais são eles;
	
	\item Sumarização: determina uma descrição compacta para um determinado subconjunto, por exemplos, medidas de posição e variabilidade. Nesta etapa se aplica algumas funções mais sofisticadas envolvendo técnicas de visualização e a determinação de relações funcionais entre variáveis. Estas funções são usadas para a geração automatizada de relatórios, sendo responsáveis pela descrição compacta de um conjunto de dados;
	
	\item Modelo de Dependência: descreve dependências significativas entre variáveis. Estes modelos existem em dois níveis: estruturado e quantitativo. O nível estruturado demonstra, através de gráficos, quais variáveis são localmente dependentes. O nível quantitativo especifica o grau de dependência, utilizando alguma escala numérica;
	
	\item Regras de Associação: determinam relações entre campos de um banco de dados. Esta relação é a derivação de correlações multivariadas que permitam auxiliar as tomadas de decisão. Medidas estatísticas, como correlação e testes de hipóteses apropriados, revelam a frequência de uma regra no universo dos dados minerados;
	
	\item Análise de Séries Temporais: determina características sequênciais, como dados com dependência no tempo. Tem como objetivo modelar o estado do processo extraindo e registrando desvios e tendências no tempo. As séries são compostas por quatro padrões: tendência, variações cíclicas, variações sazonais e variações irregulares. Existem vários modelos estatísticos que podem ser aplicados a essas situações.
\end{itemize}

A maioria destes métodos são baseados em técnicas de aprendizado de máquina (\textit{machine learning}), reconhecimento de padrões e estatística. Essas técnicas vão desde estatística multivariada, como análise de agrupamentos e regressões, até modelos mais atuais de aprendizagem, como redes neurais, lógica difusa e algoritmos genéticos \cite{conceito-data-mining}.

Devido aos vários métodos estatísticos que são aplicados no processo de \textit{data mining}, \cite{fayyad} mostram uma relevância da estatística para o processo de extração de conhecimentos ao afirmar que essa ciência provê uma linguagem e uma estrutura para quantificar a incerteza resultante quando se tenta deduzir padrões de uma amostra a partir de uma população.

%%
% Machine Learning
\section{\textit{MACHINE LEARNING}}\label{sec:machine-learning}
Abstratamente, pode-se pensar em \textit{machine learning}, ou aprendizado de máquina, como um conjunto de ferramentas e métodos que tentam inferir padrões e extrair \textit{insights} de um porção daquilo que se é observado no mundo. Por exemplo, ao tentar ensinar um computador a reconhecer os códigos postais escritos nos envelopes, os dados podem consistir em fotografias dos envelopes, além de um registro do código postal a que cada envelope estava endereçado. Ou seja, dentro de um contexto, é possível selecionar um registro de ações de certos objetos, aprender com este registro e, em seguida, criar um modelo dessas atividades que irão informar a compreensão deste contexto futuramente \cite{machine-hacker}.

Na prática, isto requer dados e, em aplicações atuais, isso, muitas vezes, significa uma grande quantidade de dados (talvez vários \textit{terabytes}). A maioria das técnicas de aprendizagem automática considera a disponibilidade de tais dados como algo inquestionável, o que significa novas oportunidades para a sua aplicação, em função da quantidade de dados que são produzidos como um produto de administrar companhias modernas.

\textit{Machine learning} é a intersecção entre ciência da computação, engenharia, estatística e ainda outras disciplinas. É possível ser aplicada em várias áreas desde políticas a geociência. É uma ferramenta que pode ser utilizada para a solução de vários problemas. Qualquer campo que precisa interpretar e agir sobre dados pode ser beneficiar do uso de técnicas de \textit{machine learning} \cite{machine-hacker}.

A prática de engenharia está em utilizar a ciência para resolver um problema. Em engenharia, é comum resolver um problema determinista, em que a solução dada por humanos sempre resolve o problema. Se desenvolver um software para controlar uma máquina de venda automática, é melhor que esta trabalhe sempre, independentemente do dinheiro depositado ou dos botões pressionados. Muitos problemas existem quando a solução não é determinista. Isto é, ou não se sabe o suficiente sobre o problema ou não se tem poder computacional suficiente para delinear adequadamente o problema. Para esses problemas, precisa-se de estatísticas. 

Uma das tarefas de \textit{machine learning} é a classificação. Na classificação, o trabalho é prever em que classe deve uma porção de dados ser enquadrada. Outra tarefa é a regressão. A regressão é a previsão de um valor numérico. Classificação e regressão são exemplos de aprendizado supervisionado. Este conjunto de problemas é conhecido como supervisionado, porque está dizendo o que o algoritmo deve prever \cite{machine-learning}.

O oposto de aprendizagem supervisionada é um conjunto de tarefas conhecidas como aprendizado não supervisionado. Em aprendizado não supervisionado, não há nenhum rótulo ou valor alvo dado para os dados. A tarefa através da qual se agrupa itens semelhantes é conhecida como \textit{clustering}. Na aprendizagem não supervisionada, também pode-se querer encontrar valores estatísticos que descrevem os dados. Isso é conhecido como estimativa da densidade. Outra tarefa do aprendizado não supervisionado está em reduzir dados com várias funcionalidades até se chegar a um número reduzido, em que seja possível visualizá-lo em duas ou três dimensões \cite{machine-learning}.

%%
% Python
\section{PYTHON}\label{sec:python}
Python é uma linguagem de programação orientada a objetos, interpretada, interativa. Incorpora módulos, excessões, tipagem dinâmica alta. Possuindo uma sintaxe clara e simples, o que facilita o aprendizado para novos desenvolvedores, assim como a rápida leitura e interpretação para usuários mais experientes. A linguagem dispõe de interfaces para várias chamadas de sistemas (\textit{system calls}) e bibliotecas, também para vários sistemas de janelas, e é extensível a outras linguagens de programação como C ou C++. É também usada como uma linguagem de extensão para aplicações que precisam de uma interface programática \cite{python-doc}.

Outra característica de Python é a portabilidade, podendo ser utilizada em diversos sistemas operacionais como variantes do Unix, em sistemas Mac e também em PCs sob MS-DOS, Windows, Windows-NT, e OS/2.

Python é uma linguagem de programação de alto-nível que pode ser aplicada em soluções para diversas classes diferentes. A linguagem possui uma vasta quantidade de bibliotecas que atende a áreas como o processamento de \textit{strings} (expressões regulares, Unicode, cálculo de diferença entre arquivos), protocolos de Internet (HTTP, FTP, SMTP, XML-RPC, POP, IMAP, CGI \textit{programming}), engenharia de software (testes unitários, registro de logs, \textit{profiling}, análise de código Python), e interfaces para sistemas operacionais (\textit{system calls}, sistemas de arquivos, TCP/IP sockets) \cite{python-doc}.

A sintaxe bastante expressiva e a abundância de suas bibliotecas tornam Python uma ótima linguagem para se obter resultados em várias questões. Algumas de suas utilidades é apresentada conforme a seguinte lista:

\begin{itemize}
	\item Escrita de \textit{scripts}: Python é uma ótima linguagem para a criação de \textit{scripts}. É possível usar \textit{scripts} para analisar arquivos de texto, gerar amostra de entradas para testar programas, coletar conteúdos de páginas \textit{web} utilizando a biblioteca \textit{Beautiful Soup}, dentre outras atividades;
	\item Desenvolvimento \textit{backend} para aplicações \textit{web}: É possível criar APIs (\textit{Application Programming Interface}, apresentado no Capítulo~\ref{ch:materiais-metodos}) e interagir com banco de dados. \textit{Frameworks} mais utilizados inclui \textit{Django}, \textit{Flask} e \textit{Pyramid};
	\item Análise e visualização de dados: Conforme o foco deste trabalho, bibliotecas como \textit{pandas}, \textit{NumPy} e recursos semelhantes a outras ferramentas como R e MATLAB estão dispostas através da biblioteca \textit{SciPy};
	\item \textit{matplotlib} e \textit{Seaborn} são mecanismos que possibilitam a visualização dos dados.
\end{itemize}

\textit{Dictionaries} em Python são estruturas de dados similares ao JSON, que permitem ordenar dados através de um modelo chave-valor. Devido então a sintaxe intuitiva que a linguagem possui e seu excelente ecossistema de bibliotecas, é possível acessar APIs e manipular dados com mais facilidade \cite{mining-social-web}. 












