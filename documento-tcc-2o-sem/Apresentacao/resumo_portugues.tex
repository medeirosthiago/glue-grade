% ---
% RESUMO PORTUGUÊS
% ---
\begin{resumo}[RESUMO]	
A rede social \textit{Twitter} é caracterizada como um serviço de microblog que permite uma breve comunicação entre pessoas. Devido a este aspecto \textit{Twitter} é utilizado para troca de mensagens imediatas, onde uma enorme quantidade de dados é gerado por interações de usuários. A grande proporcionalidade destes dados é disponibilizado pela rede social permitindo, então, a sua análise com o intuito de gerar conhecimento útil. Com o intuito de conceber informações úteis, este trabalho utiliza técnicas e algoritmos de \textit{data mining} para análise e mineração de dados provenientes da rede social \textit{Twitter} e se beneficia dos recursos e bibliotecas que a linguagem de programação Python possui para a coleta e apresentação de gráficos e planilhas para a visualização e interpretação dos dados minerados. É apresentado uma solução utilizando Python, que dispõe da API do \textit{Twitter}, para coletar todas as publicações referentes a \textit{hashtag} \#ImpeachmentDay durante o dia de votação no Congresso Brasileiro, para a continuidade do processo de Impeachment da Presidente Dilma Rousseff. Após a coleta de dados, fez-se necessário a limpeza dos dados, onde foi eliminado informações desnecessárias como \textit{dirty data}. Através da implementação de filtros e técnicas de mineração de dados é possível encontrar padrões e informações proveniente das publicações. Os idiomas e os países que trazem mais \textit{tweets}, as \textit{hashtags} mais utilizadas e os nomes mais mencionados, são ilustrados através de gráficos onde se quantifica em números e porcentagens a disponibilidade dos \textit{tweets}. Os picos de horários em que houveram a maior ocorrência de \textit{tweets} e a localização geográfica de algumas dessas publicações também são apresentadas neste trabalho. A oportunidade de utilizar a linguagem Python desde a autenticação da rede social através da sua API, a coleta de \textit{tweets}, a normalização dos dados e posteriormente filtrá-los e apresentá-los de diversas maneiras para visualização, torna a linguagem uma excelente ferramenta para as tarefas de mineração de dados.



 \vspace{\onelineskip}
    
 \noindent
 \textbf{Palavras-chaves}: Dados. Data Mining. Twitter. Python.
 % 4 palavras separadas por . (ponto)
\end{resumo}
