\chapter{CONCLUSÕES E SUGESTÕES PARA TRABALHOS FUTUROS}\label{ch:conclusao}

A proposta dessa dissertação foi de utilizar técnicas e algoritmos de \textit{data mining} para analisar e minerar os dados provenientes da rede social \textit{Twitter}, utilizando os recursos e bibliotecas da linguagem de programação Python. Para tanto, foi necessário o estudo de conceitos de KDD e \textit{data mining}, implementar técnicas utilizando bibliotecas da linguagem em estudo e apresentar informações úteis para possíveis interpretações.

Durante o estudo de \textit{data mining}, foi determinado que a sua definição consiste em um processo de descoberta de padrões interessantes e conhecimentos de um vasto conjunto de dados. Sendo necessário, então, ferramentas que auxiliem na coleta, limpeza, seleção e apresentação dos dados.

Para a coleta e limpeza dos dados a linguagem Python se tornou extremamente útil, ao permitir o desenvolvimento de um código que se autenticou com o \textit{Twitter} e a utilização da API de \textit{streaming} para coletar os dados que continham a \textit{hashtag} \#ImpeachmentDay.




%- Resgatar o objetivo
%
%- Comentar as ferramentas estudadas
%
%- Comentar as ferramentas utilizadas
%
%- Breve resumo dos resultados
%
%- Pontos positivos e negativos (O fato de não ter o perfil real)
%
%- Pensar em outra API (Aplicação para previsão de resultados) ou rede social, baseando-se numa pergunta específica