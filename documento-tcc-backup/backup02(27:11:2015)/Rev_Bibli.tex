\chapter{REVISÃO BIBLIOGRÁFICA}
%Título Capítulo 2

O estudo da ruptura de barragens tem sido, por mais de cem anos, foco de pesquisadores das mais variadas áreas do conhecimento e seu tratamento depende da resposta que está sendo esperada, o que pode diferenciar a maneira de equacioná-lo. Alguns pesquisadores utilizam as equações de Euler como governantes do sistema. Outros, as equações de Navier-Stokes. Existem aqueles que adotam as equações de Saint Venant ou, até mesmo, as equações de Águas Rasas. 

O tratamento computacional do fenômeno da ruptura de barragens, representado por algumas dessas equações, poderá ser feito numa configuração euleriana, com a adoção de uma malha (\textit{grid}), ou por uma configuração lagrangeana, com o uso de partículas (\textit{meshfree}), ou mesmo, por uma configuração que adota a mistura dessas duas, ao qual denomina-se configuração híbrida. Percebe-se então, que essa simulação tem um significado de extrema relevância dentro do conhecimento científico. Assim, esta revisão literária faz um levantamento, não só daqueles trabalhos que são relevantes para esta dissertação, mas também daqueles que, de alguma forma, utilizam a ruptura de barragens como tema.

Iniciando por aqueles trabalhos que adotam a configuração euleriana destaca-se aquele realizado por \citeonline{Hirt}, que usaram a técnica das Diferenças Finitas, com a pressão e a velocidade como variáveis primitivas, para resolverem as equações de Navier-Stokes para um fluido incompressível. A técnica para modelar as equações foi baseada no método \textit{Marker-and-Cell}, que foi simplificada para facilitar o uso por pessoas com pouca ou nenhuma experiência em Dinâmica do Fluidos Computacional. Eles ainda descreveram um algoritmo em Fortran, intitulado de SOLA, e acrescentaram exemplos com o intuito de verificar a eficiência do código sendo que, entre esses, está a ruptura de uma barragem hipotética.

Verificando que muitos pesquisadores desenvolviam seus trabalhos em Diferenças Finitas, \citeonline{HirtVOF} desenvolveram um novo código baseado na técnica de Volume de Fluido (VOF), denominando-o de SOLA-VOF. Com o intuito de testar a eficiência e a precisão do novo código, modelaram a queda de uma coluna d'água, que estava em repouso, removendo repentinamente sua barreira de contenção. Desta forma, a água se deslocou sobre o outro lado da barreira que estava totalmente seco, ao que denominaram de ruptura de barragem de "leito seco". 

\citeonline{Bellos} estudaram a onda de inundação, gerada sobre um leito seco, ocasionada pela ruptura de uma barragem. Para tanto, utilizaram o Método das Diferenças Finitas, mas esta aplicação gerou dificuldades em modelar as características do escoamento próximo das fronteiras. Para a correção dessas dificuldades, transformaram o sistema gerado pelas equações usando o esquema explícito de McCormack que, segundo os autores, melhorou muito os resultados obtidos.

Adotando um equacionamento diferente, \citeonline{Glaister} resolveu o problema da ruptura de barragens considerando as equações de Saint Venant como governante do sistema. Para resolvê-las usou a decomposição numérica característica e linearizou o problema por meio do método de diferenciação \textit{Upwind}, o que resultou num problema escalar em conjunto com um fluxo limitador que foi obtido com um esquema de segunda ordem evitando o aparecimento de oscilações espúrias ou não-físicas.

Considerando um modelo real, \citeonline{Collis} modelaram a barragem de Ernestina, situada no Rio Jacuí, no Estado do Rio Grande do Sul. Usaram o modelo DAMBRK para simular o escoamento resultante da ruptura daquela barragem até a barragem de Passo Real e Maia Filho, situada no mesmo rio, $200 \ km$ à jusante. Os impactos do evento foram analisados por meio de geoprocessamento, o que determinou as áreas alagadas e as populações atingidas. Para a modelagem utilizaram as equações de Saint Venant acopladas com as equações de escoamento rapidamente variados, o que representa o escoamento por brechas e aterros. O sistema foi discretizado usando o Método das Diferenças Finitas. A forma como se comportou o escoamento pôde ser considerada como supercrítica ou subcrítica e, em algumas partes, como uma combinação dos dois. O fluido foi tratado como newtoniano, depois como não-newtoniano devido aos sedimentos transportados.

\citeonline{Mohapatra} analisaram a ruptura de barragens usando um modelo bidimensional das equações de Euler num plano vertical. Usaram uma aproximação geral do método \textit{Marker-and-cell} em combinação  com a aproximação do método VOF. Investigaram a evolução da profundidade à montante da barragem e as variações de pressão para os casos de leitos secos e com uma certa quantidade de água (leitos úmidos). Concluíram que, no caso de um leito úmido, a pressão não-hidrostática, imediatamente após a ruptura da barragem, não tem significância para os resultados que ocorrem com a evolução do tempo.

\citeonline{Wang} estudaram um método híbrido do esquema das Diferenças Finitas juntamente com o método da Diminuição Total da Variação (TVD) de segunda ordem para resolverem o problema da ruptura de barragens. Para tanto, basearam-se  no esquema de primeira ordem \textit{Upwind} associado ao esquema de Lax-Wendroff, usando um e dois parâmetros limitantes. Com o intuito de testar o método híbrido, usaram diferentes parâmetros limitantes no sistema governado pelas equações de Saint Venant, até que conseguiram determinar o melhor limitador. Assim, estenderam o problema da ruptura de barragens para o caso bidimensional sendo governado pelas equações de Águas Rasas, alcançado bons resultados em comparação com outras publicações.

Na pesquisa de \citeonline{Macchione}, os autores compararam o desempenho do método da viscosidade artificial TVD com outros métodos como: o TVD, o esquema \textit{Upwind} com difusão de fluxo e os esquemas centrados de primeira e segunda ordens. Em todos os casos observaram a onda formada pela ruptura de barragens em leitos secos, sem e com fricção, sem e com inclinação. Destacaram que, para o esquema TVD de segunda ordem, não houve diferenças significativas em comparação com os esquemas \textit{Upwind} e os centrados. Destacaram também que, para o problema de ruptura de barragens no plano horizontal sem atrito, os esquemas de segunda ordem são de duas a cinco vezes mais precisos que os esquemas de difusão e o esquema de Roe, sendo que estas diferenças são reduzidas quando há rugosidades (fricção) e inclinações nos leitos.

Com o intuito de verificar a qualidade dos esquemas numéricos explícitos, \citeonline{Zoppou} compararam vinte esquemas que resolvem as equações de Águas Rasas e simulam a ruptura de barragens em canais com leitos secos e profundidades finita. Constataram que muitos deles produzem resultados razoáveis para escoamentos subcríticos, mas no casos onde há uma transição de subcrítico para supercrítico, ocorre uma mistura de resultados. Verificaram que, para a maioria dos esquemas de primeira ordem, as soluções não satisfazem às condições de entropia, o que produz soluções não-físicas. Analisaram os esquemas de segunda ordem que evitam a geração de choques, mas afirmaram que para isso, alguma forma de limitador de inclinação ou fluxo deve ser usado para eliminar as oscilações existentes e que esta prática eleva o tempo computacional e a complexabilidade do algoritmo. No entanto, concluíram que os esquemas de segunda ordem são mais precisos.

No artigo desenvolvido por \citeonline{Fraccarollo}, os autores propuseram o uso do método de Godunov para computar o escoamento em canais abertos com condições de rápida erosão e intenso transporte de sedimentos. Para tanto, adotaram as equações de Águas Rasas como governante do sistema e as resolveram usando o esquema de Harten, Lax e Van Leer (HLL) juntamente com o método dos Volumes Finitos. Para validar a proposta usaram a erosão causada pela ruptura de barragens, comparando com medições experimentais e com  a solução semi-analítica de Riemann.

\citeonline{Yang} usaram a técnica VOF para resolver as equações de Euler e Navier-Stokes e simularam a interação de ondas de grande porte e estruturas tridimensionais. Para tanto, adotaram uma malha não-estruturada onde as incompressibilidades das equações foram resolvidas pelo Método dos Elementos Finitos (MEF). Para validar o código implementado, tomaram como exemplo a ruptura de uma barragem tridimensional. Após, num tanque bidimensional parcialmente cheio, utilizaram o código para simular a formação do \textit{sloshing}. Para comparar o experimento, os autores fizeram uso dos resultados encontrados por meio do método da Hidrodinâmica das Partículas Suavizadas (SPH). 

Nos estudos realizados por \citeonline{Uemura} foi proposto, por meio de um modelo unidimensional  baseado nas equações de Saint Venant, chamado de ClivPlus, a gestão de um plano de emergência para o caso da ruptura da barragem de Guarapiranga, no Estado de São Paulo, visando o estabelecimento de rotinas para avaliar os impactos causados por um possível acidente.

Com o intuito de desenvolver um método numérico robusto e, computacionalmente, eficiente para modelar o escoamento ocasionado pela ruptura de barragens, \citeonline{Ying} propuseram a formulação de um modelo baseado no Método dos Volumes Finitos (MVF), construindo uma malha triangular não-estruturada. Consideraram as equações de Águas Rasas como governante  do sistema, onde os termos de pressão e gravidade foram incluídos por meio de um termo fonte, o que significou a computação  e eliminação do desequilíbrio entre os termos fluxo e fonte. A precisão e eficiência computacional foram testados com experimentos laboratoriais e a modelagem do caso real ocorrido na represa de Malpasset, na França.

No estudo realizado por \citeonline{Dutykh}, os autores verificaram a pertinência das equação não-lineares de Águas Rasas (NSWE) para solucionar os processos de inundações. Para isso, resolvem as equações bidimensionais de Navier-Stokes e as NSWE, analítica e numericamente, usando um modelo de ruptura de barragens.

A simulação numérica da ruptura de barragens usando o método da Pseudo-Concentração por meio da performance do GLITE, que baseia-se na infraestrutura \textit{Baltic Grid} e é uma ferramenta do código chamado FEMTOOL desenvolvida em MEF, foi proposto no trabalho desenvolvido por \citeonline{Kavceniauskas}. Para tanto, modelaram do perfil de uma coluna d'água colapsando. Consideraram o comportamento do experimento como aceitável ao compararem os dados resultantes com aqueles obtidos em outros trabalhos que efetuaram a mesma simulação. 

\citeonline{Gao} usaram o método VOF para resolver as equações de Navier-Stokes e simular a inundação dos porões de um navio danificado, pois este tipo de problema pode prejudicar a navegabilidade. Validaram seus resultados com aqueles encontrados em experimentos e na bibliografia de ruptura de barragens bi e tridimensionais.

\citeonline{Kocaman} investigaram o comportamento da ruptura de barragens cujo deslocamento do escoamento sofre a influência da profundidade e da mudança de direção ao colidir com prédios e construções localizados a jusante da barragem. Para tanto, analisaram experimentalmente e numericamente o evento. Após, compararam os resultados com o  modelo gerado pelo software comercial FLOW3D, que usa as equações médias espaciais e temporais de Navier-Stokes (RANS) juntamente com o modelo de turbulência $k- \epsilon$. Segundo os autores, os resultados foram satisfatórios em todas as comparações.

\citeonline{Peng} propôs a modelagem da ruptura de barragem por meio das equações de Águas Rasas unidimensional e bidimensional. Resolveu o sistema formado usando MVF e o esquema de Roe e HLL. Como proposta de diferenciação, adotou o método Operador-Divisão, com o intuito de separar as equações governantes em equações hiperbólicas e termos fonte. Segundo o autor, o código ficou simples, de fácil implementação e pôde ser usado para outras computações, como fluidos não-newtonianos ou escoamentos de fluidos com multicamadas.

Por intermédio da ruptura de barragem e da formação do \textit{sloshing}, \citeonline{Carbajal} estudaram a dinâmica dos navios e plataformas \textit{offshore} que se encontram em condições indesejáveis. Para tanto, propuseram a criação de um modelo bidimensional de uma barragem usando o Método das Diferenças Finitas (MDF), onde solucionaram as equações de Euler por meio do esquema \textit{Upwind} TVD de Roe e Sweby. Testaram a versatilidade do código baseando-se em resultados da literatura.

\citeonline{Verol} criaram um modelo quase bidimensional, chamado de MODCEL, para simular a ruptura da barragem da usina de funil hipotética em comparação com aquela situada no Rio Paraíba do Sul, no Estado do Rio de Janeiro. Validaram o modelo com experimentos em laboratórios e foram capazes de predizer, com exatidão, o tempo de inundação e as profundidades do alagamento nas áreas de risco a jusante da barragem.

Dentre os trabalhos que adotam a configuração lagrangeana destaca-se o realizado por \citeonline{Koshizuka} que apresentaram o método \textit{Moving Particle Semi-Implicite} (MPS) para simular a fragmentação de fluidos incompressíveis. Propuseram um modelo determinístico de interação de partículas para representar gradientes, laplacianos e superfícies livres. Resolveram a equação de pressão de Poisson por meio Método do Gradiente Conjugado Incompleto de Cholesky. Com o intuito de verificar a eficiência de método desenvolvido, calcularam a evolução do colapso de uma coluna d'água simulando a ruptura de uma barragem.

Em seu trabalho, \citeonline{Colagrossi} implementaram o método da Hidrodinâmica das Partículas Suavizadas (SPH) com um tratamento interfacial, ou seja, campos de fluxos para diferentes fluidos com separação por interfaces acentuadas. Apresentaram a ruptura de uma barragem com aproximação bifásica e estudaram os efeitos da variação da razão de densidade. Afirmaram, como resultado, que a evolução dos fenômenos modelados, para um período de tempo relativamente maior, necessita da resolução das equações de Navier-Stokes, devido aos efeitos da viscosidade dinâmica das estruturas rotacionais geradas. No entanto, concluíram que efeitos de interesse das engenharias são aqueles que ocorrem em um curto intervalo de tempo, tais como os carregamentos causados pelos impactos e, neste caso, um passo importante seria uma análise tridimensional.

Como foco da sua pesquisa, \citeonline{Feldman} propuseram um refinamento no tratamento das partículas que estão próximas ao núcleo (partícula vizinhas), no método SPH. Para tanto, trataram essas partículas como "partículas filhas", propondo um refinamento maior para as partícula mais próximas. Com isso, através da solução do problema de minimização não-linear, obtiveram uma distribuição mais ideal de massa entre as partículas filhas, o que reduziu a introdução de erros na base do campo de densidade, conservando a massa do sistema. Validaram sua ideia através da simulação da ruptura de barragens e barreiras de contenção de enchentes, mostrando que o modelo é eficiente e reduz o tempo computacional.

\citeonline{Harada} propuseram a implementação total do método SPH em unidades de processamento gráfico, ou seja, a implementação direta nas placas gráficas (GPUs). Com isso, puderam explorar, massivamente, a capacidade de processamento dessas unidades, o que não acontecia anteriormente, pois somente a busca por partículas vizinhas era implementada nas GPUs. Como teste de processamento, simularam a ruptura de uma barragem contendo $4.000.000$ de partículas em uma GPU com $768 \ MB$ de memória. Os resultados obtidos, em alguns casos, foram $28$ vezes mais rápidos que a programação na unidade central de processamento (CPU).

No trabalho proposto por \citeonline{Neto} foi implementado um sistema capaz produzir animações interativas usando as equações de Águas Rasas para jogos, modelados por meio de método SPH. A vantagem deste modelo, segundo o autor, é que, por ser baseado em Águas Rasas, foi possível obter uma representação tridimensional do fluido através de uma simulação bidimensional. Com isso, foi evitado o uso do teste de colisão do fluido contra os objetos em cena. Em seus testes de validação constatou que, por ser um método mais direcionado para animação, não se comportou de forma satisfatória ao simular problemas de engenharia como a ruptura de barragens.

Com o objetivo de gerar um sistema capaz de avisar pessoas em áreas de risco, \citeonline{Capone} trabalhou com a complexidade da simulação numérica do deslizamento de terras causadas por ondas. Inicialmente, usou o SPH somente para considerar as ondas como um corpo rígido. Após, tratou-as como um corpo deformável e testou seu modelo simulando a ruptura de barragens hipotéticas, com e sem construções formando barreiras. 

\citeonline{Ryszard} simulou a ruptura de uma barragem hipotética para modelar o escoamento de um fluido e testar uma reformulação no tratamento das funções padrões usadas como núcleos suavizados no método SPH. Mostrou que as mudanças propostas foram úteis e que eram capazes de predizer as mudanças nas posições de avanço das frentes de onda. No entanto, constatou que estas melhorias elevaram o custo computacional, mas não afetaram a eficiência dos resultados.

Uma visão geral do método SPH foi realizada por \citeonline{Liu}, onde fizeram uma revisão de todos os procedimentos que eram necessários para a aplicação do método, desde a consistência e estabilidade das partículas até a formulação dos núcleos. Destacaram-se ainda por verificar todo que havia sido feito de modificações e melhorias no SPH até aquele ano, demonstrando estes avanços com o uso de exemplos clássicos, como a ruptura de barragens.

Em seu trabalho \citeonline{Kao} utilizaram o método SPH para modelar as equações de Águas Rasas bidimensionais que simulam a ruptura de barragens que influenciam em enchentes e inundações. No modelagem proposta constaram que ocorreram fenômenos complicados de serem trabalhados, tais como misturas de escoamentos trans-críticos, propagações de frentes de choques, reflexões parciais, saltos hidráulicos e interação de múltiplas ondas. No entanto, a implementação numérica foi capaz de simulá-los de maneira fiel e, assim, provaram que este tipo de modelagem é útil para problemas práticos da engenharia hidráulica.

Baseados no método SPH Fracamente Compressível (WCSPH), \citeonline{Zhang} propuseram uma nova equação de pressão para reduzir o tempo computacional, acelerando a convergência para pequenas variações da densidade, o que permitiu um maior intervalo de tempo de simulação. Juntamente com essa mudança, introduziram uma nova aproximação para o passo de tempo que, automaticamente, adapta o tempo de acordo com cada partícula, o que melhorou a eficiência computacional e contribuiu para a simulação de fluidos em tempo real. Para isso, combinaram um processamento em paralelo de CPU e GPU, ao que chamaram de Plataforma Multi-GPU. Com isso, conseguiram rodar a simulação da ruptura de uma barragem em escada com efeitos realísticos e em tempo real.

Como visto, o tema sobre ruptura de barragens tem sido uma ferramenta útil às simulações numéricas. Neste contexto, esta dissertação modela, de uma forma mais simplificada, esse experimento e trata o sistema de equações formado por meio de uma variação das equações de Euler, assim como foi feito no trabalho de \citeonline{Carbajal}. 



       

    

         

                                
