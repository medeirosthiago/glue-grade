% resumo em português
\begin{resumo}[RESUMO]	
É natural ao seres humanos a necessidade de se relacionar com outras pessoas e é através de redes sociais que esta relação se torna real no mundo inteiro. \textit{LinkedIn} é uma destas redes sociais e tem a finalidade de desenvolver relacionamentos profissionais. A interação entre as pessoas nesses ambientes \textit{online} acabam gerando muitos dados que, em algum momento, precisão se tornar informação útil. É necessário então trabalhar com esses dados, encontrar meios para automatizar a análise, classificação, sumarização, descobrimento e caracterização destes e também apontar algumas anomalias. \textit{Data mining} surge dessa necessidade. Através da interdisciplinaridade, pesquisadores de diversas áreas, incluindo estatística, engenharia, inteligência artificial e aprendizado de máquina, estão contribuindo e gerando ferramentas para este campo. Python tem sido utilizado como uma ferramenta para \textit{data mining} graças ao seu forte poder de programação e também de suas bibliotecas que permitem a análise e mineração de dados. Por fim, este trabalho tem como objetivo minerar os dados provenientes do \textit{LinkedIn}, com a finalidade de encontrar padrões em perfis de profissionais de Tecnologia da Informação.


 \vspace{\onelineskip}
    
 \noindent
 \textbf{Palavras-chaves}: Dados. Data Mining. LinkedIn. Python.
 % 4 palavras separadas por . (ponto)
\end{resumo}
