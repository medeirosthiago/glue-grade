% ---
% RESUMO PORTUGUÊS
% ---
\begin{resumo}[RESUMO]	
A rede social \textit{Twitter} é utilizada para troca de mensagens imediatas, em que uma enorme quantidade de dados é gerada a partir das interações de seus usuários. O grande volume destes dados é disponibilizado pela rede social permitindo, dessa forma, a sua análise com o intuito de gerar conhecimento útil. Este trabalho apresenta o estudo e implementação de técnicas de \textit{data mining} para análise e mineração dos dados coletados no dia 17 de abril de 2016, data da votação, no Congresso brasileiro, sobre a continuidade do processo de Impeachment da presidente Dilma Rousseff, e se beneficia dos recursos e bibliotecas que a linguagem de programação Python possui para a extração de dados e apresentação  destes em gráficos, permitindo, assim, a visualização e interpretação dos resultados obtidos. 

 \vspace{\onelineskip}
    
 \noindent
 \textbf{Palavras-chaves}: Dados. Data Mining. Twitter. Python.
 % 4 palavras separadas por . (ponto)
\end{resumo}
