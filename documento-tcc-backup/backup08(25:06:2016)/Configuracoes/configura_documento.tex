% !TeX encoding = UTF-8

\preambulo{Trabalho de conclus{\~ a}o de curso apresentado como requisito obrigat{\' o}rio para obten{\c c}{\~ a}o do t{\' i}tulo de Bacharel em Ci{\^ e}ncia da Computa{\c c}{\~ a}o da Faculdade Anglo-Americano de Foz do Igua{\c c}u.}
\local{Foz do Igua{\c c}u}
\instituicao{Faculdade Anglo-Americano de Foz do Igua{\c c}u}


% ---
% CAPA
% ---
\renewcommand{\imprimircapa}{%
  \begin{capa}
  	\begin{center}
  		\textbf{\large{\MakeUppercase{\imprimirinstituicao}}}
  	\end{center}
    \center

    {\normalsize\MakeUppercase{\imprimirautor}}

    \vspace*{8cm}
    \ABNTEXchapterfont\bfseries\large\MakeUppercase{\imprimirtitulo}
    \vfill
    
    {\normalfont\small\MakeUppercase{\imprimirlocal}}

    {\normalfont\small\imprimirdata}
    
    \vspace*{1cm}
  \end{capa}
}

\makeatletter
% ---
% FOLHA DE ROSTO
% ---
\renewcommand{\folhaderostocontent}{
		\center	
		{\normalsize\MakeUppercase{\imprimirautor}}
	
	    \vspace*{8cm}		
		
		{\large\MakeUppercase{\imprimirtitulo}}
	
		\vspace*{3cm}	

		\abntex@ifnotempty{\imprimirpreambulo}{
			\hspace{.3\textwidth}
			\begin{minipage}{.55\textwidth}
				{\footnotesize \SingleSpacing
				\imprimirpreambulo				
				\SingleSpacing				
				{\imprimirorientadorRotulo~\imprimirorientador\par}
				\SingleSpacing	
		        \abntex@ifnotempty{\imprimircoorientador}{
		             {\imprimircoorientadorRotulo~\imprimircoorientador} } }
			\end{minipage}
		
			\vspace*{\fill}
		}
		
		{\normalfont\small\MakeUppercase{\imprimirlocal}}
		
		{\normalfont\small\imprimirdata}		
}


% ---
% OUTRAS CONFIGURAÇÕES
% ---
\setlength{\parindent}{1.3cm} 		% Tamanho do parágrafo
\setlength{\parskip}{0.2cm}  		% Controle do espaçamento entre um parágrafo e outro

% Controle do espaçamento entre a legenda e a fonte (e nota) de figuras e tabelas}:
\setlength{\abovecaptionskip}{0 mm}
\setlength{\belowcaptionskip}{1 mm}

% Define os tamanhos das margens do documento
% ---
\setlrmarginsandblock{3cm}{2cm}{*}
\setulmarginsandblock{3cm}{2cm}{*}
\checkandfixthelayout

	
% Altera o tamanho das fontes dos capítulos e dos apêndices
% ---
\renewcommand{\ABNTEXchapterfont}{\bfseries}
\renewcommand{\ABNTEXchapterfontsize}{\large}
\renewcommand{\ABNTEXsectionfontsize}{\normalfont}

% ---
% SEPARAÇÃO DE PALAVRAS
% ---
%\hyphenation{}
\hyphenation{teorias}
\hyphenation{baseados}
\hyphenation{MATLAB}
\hyphenation{Pyramid}
\hyphenation{Mar-cação}
\hyphenation{lin-guagem}
\hyphenation{matplotlib}
\hyphenation{sociali-zação}
\hyphenation{bi-bli-oteca}
\hyphenation{Notebook}
\hyphenation{profis-si-onais}
\hyphenation{ar-quivos}
\hyphenation{usu-ário}
\hyphenation{fer-ramentas}
\hyphenation{materi-al}
\hyphenation{re-almen-te}
\hyphenation{LinkedIn}
\hyphenation{proces-samento}
\hyphenation{neces-sidades}
\hyphenation{relaci-onados}

% ---
% CONFIGURACAO DO ESTILO DE PÁGINA TEXTUAL 
% ---
% criar um novo estilo de cabeçalhos e rodapés
\makepagestyle{abnt-anglo}
  %%cabeçalhos
  \makeevenhead{abnt-anglo}{\ABNTEXfontereduzida\thepage}{}{}
  \makeoddhead{abnt-anglo}{}{}{\ABNTEXfontereduzida\thepage}
  %\makeheadrule{textualCC-Anglo}{\textwidth}{\normalrulethickness} %linha
  %% rodapé
  \makeevenfoot{abnt-anglo}{}{}{} 
  \makeoddfoot{abnt-anglo}{}{}{}
\pagestyle{abnt-anglo}

% ---
% utfpr-style
%\makepagestyle{abnt_ufpr}
%
%\makeoddhead{abntchapfirst}{}{}{\ABNTEXfontereduzida\thepage}
%
%\pagestyle{abnt_ufpr}
%
%\newtheorem{theorem}{Teorema}[chapter]
%\newtheorem{definition}[theorem]{Defini\c{c}\~{a}o}
%\newtheorem{proposition}[theorem]{Proposi\c{c}\~{a}o}

\makeatother

\usepackage{perpage} %the perpage package
\MakePerPage{footnote} %the perpage package command

% ---
% CONFIGURAÇÃO DE LISTAS DE CONTEÚDO
% ---
% lista de gráficos
% -
\newcommand{\graficoname}{GRÁFICO}
\newcommand{\listofgraficosname}{LISTA DE GRÁFICOS}

\newfloat[chapter]{grafico}{loq}{\graficoname}
\newlistof{listofgraficos}{loq}{\listofgraficosname}
\newlistentry{grafico}{loq}{0}

\counterwithout{grafico}{chapter}		% configurações para atender às regras da ABNT em listas
\renewcommand{\cftgraficoname}{\graficoname\space}
\renewcommand*{\cftgraficoaftersnum}{\hfill--\hfill}

% lista de códigos
% -
\renewcommand{\lstlistingname}{CÓDIGO}
\renewcommand{\lstlistlistingname}{LISTA DE CÓDIGOS}

\begingroup\makeatletter				% configurações para atender às regras da ABNT em listas
\let\newcounter\@gobble\let\setcounter\@gobbletwo
  \globaldefs\@ne \let\c@loldepth\@ne
  \newlistof{listings}{lol}{\lstlistlistingname}
  \newlistentry{lstlisting}{lol}{0}
\endgroup

\renewcommand{\cftlstlistingaftersnum}{\hfill--\hfill}

\let\oldlstlistoflistings\lstlistoflistings
\renewcommand{\lstlistoflistings}{%
   \begingroup%
   \let\oldnumberline\numberline%
   \renewcommand{\numberline}{\lstlistingname\space\oldnumberline}%
   \oldlstlistoflistings%
   \endgroup}

%\setlength{\parindent}{1.3cm} 			% Tamanho do parágrafo
%\setlength{\parskip}{0.0cm}  			% Controle do espaçamento entre um parágrafo e outro

% Altera o tamanho das fontes dos capítulos e dos apêndices
% -
\renewcommand{\ABNTEXchapterfont}{\normalfont\fontseries{b}\selectfont}
\renewcommand{\ABNTEXchapterfontsize}{\normalsize}
\renewcommand{\ABNTEXpartfont}{\fontseries{b}\selectfont\selectfont}
\renewcommand{\ABNTEXpartfontsize}{\normalsize}
\renewcommand{\ABNTEXsectionfont}{\normalfont\selectfont}
\renewcommand{\ABNTEXsectionfontsize}{\normalsize}
\renewcommand{\ABNTEXsubsectionfont}{\normalfont\selectfont}
\renewcommand{\ABNTEXsubsectionfontsize}{\normalsize}
\renewcommand{\ABNTEXsubsubsectionfont}{\normalfont\selectfont}
\renewcommand{\ABNTEXsubsubsectionfontsize}{\normalsize}
\renewcommand{\ABNTEXsubsubsubsectionfont}{\normalfont\itshape\selectfont}
\renewcommand{\ABNTEXsubsubsubsectionfontsize}{\normalsize}

% ---
% CONFIGURAÇÃO DO SUMÁRIO
% ---
% Sumário
\renewcommand*{\cftsectionfont}{\normalfont}
\renewcommand*{\cftsubsubsectionfont}{\normalfont}
\renewcommand*{\cftsubsectionfont}{\normalfont}
\renewcommand*{\cftparagraphfont}{\normalfont\itshape}

% ---
% Modifica o espaçamento no sumário
% Nao ha espacos, para as entradas de capitulos
% ---
\setlength{\cftbeforechapterskip}{\onelineskip}
\setlength{\cftbeforesectionskip}{0pt}
\setlength{\cftbeforesubsectionskip}{0pt}
\setlength{\cftbeforesubsubsectionskip}{0pt}
\setlength{\cftbeforeparagraphskip}{0pt}

%\setlength{\cftbeforeparagraphskip}{0pt}
%\setlength{\cftbeforesubsectionskip}{0pt}
%\setlength{\cftbeforesectionskip}{0pt}
%\setlength{\cftbeforesubsubsectionskip}{0pt}
%\setlength{\cftbeforechapterskip}{0pt}

% ---
% CAPITALIZAÇÃO DE LISTAS
% ---
\renewcommand{\fontename}{FONTE}

\addto\captionsbrazil{
	\renewcommand{\bibname}{REFER\^ENCIAS}
}
\addto\captionsbrazil{\renewcommand{\listadesiglasname}{LISTA DE ABREVIATURAS}}
\addto\captionsbrazil{\renewcommand{\listfigurename}{LISTA DE ILUSTRAÇÕES}}
\addto\captionsbrazil{\renewcommand{\listtablename}{LISTA DE TABELAS}}
%\addto\captionsbrazil{%
%  \renewcommand*{\lstlistlistingname}{LISTA DE CÓDIGOS}%
%  \renewcommand*{\lstlistingname}{CÓDIGO}%
%}
\addto\captionsbrazil{\renewcommand{\contentsname}{SUMÁRIO}}
\addto\captionsbrazil{\renewcommand\appendixtocname{APÊNDICES}\renewcommand\appendixpagename{APÊNDICES}}
\addto\captionsbrazil{\renewcommand{\figurename}{FIGURA}}
\addto\captionsbrazil{\renewcommand{\tablename}{TABELA}}

% ---
% LAYOUT PARA ELEMENTOS TEXTUAIS
% ---
%\makepagestyle{abnt_ufpr}

%\makeoddhead{abntchapfirst}{}{}{\ABNTEXfontereduzida\thepage}

%\pagestyle{abnt_ufpr}

% ---
% AMBIENTES
% ---
%\newtheorem{teo}{Teorema}[chapter]
%\newtheorem{cor}[teo]{Corol\'{a}rio}
%\newtheorem{lem}[teo]{Lema}
%\newtheorem{prop}[teo]{Proposi\c{c}\~{a}o}
%\newtheorem{defn}[teo]{Defini\c{c}\~{a}o}
%\newtheorem{Ex}[teo]{Exemplo}
%\newtheorem{obs}[teo]{Observa\c{c}\~{a}o}
%\newtheorem{prob}[teo]{Problema}
%\newtheorem{conc}[teo]{Conclusão}
%\newenvironment{dem}{\smallskip \noindent{\bf Demonstra\c{c}\~{a}o}: }
%{\hfill $\Box$\hspace{0in}\medskip}

% ---
% COMANDOS FREQUENTES
% ---
%\newcommand{\eq}{\begin{equation}}
%	\newcommand{\ee}{\end{equation}}
%\newcommand{\R}{{\mathbb R}}
%\newcommand{\N}{{\mathbb N}}
%\newcommand{\K}{{\mathbb K}}
%\newcommand{\Q}{{\mathbb Q}}
%\newcommand{\Z}{{\mathbb Z}}
%\newcommand{\V}{{\mathbb V}}
%\newcommand{\D}{{\mathcal{D}}}
%\newcommand{\C}{{\mathbb C}}
%\newcommand{\di} {\displaystyle}
%\newcommand{\I}{{\displaystyle\int_{0}^{T} \displaystyle\int_{0}^1 }}
%\newcommand{\Ia}{{\displaystyle\int_{0}^{1} \displaystyle\int_{0}^T }}
%\newcommand{\Ii}{{\displaystyle\int_{0}^{t} \displaystyle\int_{0}^1 }}
%\newcommand{\Ib}{{\displaystyle\int_{0}^{1} \displaystyle\int_{0}^t }}

%\makeatother 